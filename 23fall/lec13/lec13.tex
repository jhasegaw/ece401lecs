\documentclass{beamer}
\usepackage{tikz,amsmath,hyperref,graphicx,stackrel,animate,media9}
\usetikzlibrary{positioning,shadows,arrows,shapes,calc}
\newcommand{\argmax}{\operatornamewithlimits{argmax}}
\newcommand{\argmin}{\operatornamewithlimits{argmin}}
\mode<presentation>{\usetheme{Frankfurt}}
\AtBeginSection[]
{
  \begin{frame}<beamer>
    \frametitle{Outline}
    \tableofcontents[currentsection,currentsubsection]
  \end{frame}
}
\title{Lecture 13: Frequency Response}
\author{Mark Hasegawa-Johnson\\These slides are in the public domain}
\date{ECE 401: Signal and Image Analysis, Fall 2023}  
\begin{document}

% Title
\begin{frame}
  \maketitle
\end{frame}

% Title
\begin{frame}
  \tableofcontents
\end{frame}

%%%%%%%%%%%%%%%%%%%%%%%%%%%%%%%%%%%%%%%%%%%%
\section[Review]{Review: Convolution and Fourier Series}
\setcounter{subsection}{1}

\begin{frame}
  \frametitle{What is Signal Processing, Really?}

  \begin{itemize}
  \item When we process a signal, usually, we're trying to
    enhance the meaningful part, and reduce the noise.
  \item {\bf Spectrum} helps us  to understand which part is
    meaningful, and which part is noise.
  \item {\bf Convolution} (a.k.a. filtering) is the tool we use to
    perform the enhancement.
  \item {\bf Frequency Response} of a filter tells us exactly which
    frequencies it will enhance, and which it will reduce.
  \end{itemize}
\end{frame}

\begin{frame}
  \frametitle{Review: Convolution}
  \begin{itemize}
  \item A {\bf convolution} is exactly the same thing as a {\bf weighted local average}.
    We give it a special name, because we will use it very often.  It's defined as:
    \[
    y[n] = \sum_m h[m] f[n-m] = \sum_m h[n-m] f[m]
    \]
  \item 
    We use the symbol $\ast$ to mean ``convolution:''
    \[
    y[n]=h[n]\ast f[n] = \sum_m h[m] f[n-m] = \sum_m h[n-m] f[m]
    \]
  \end{itemize}
\end{frame}

\begin{frame}
  \frametitle{Review: Spectrum}

  The {\bf spectrum} of $x(t)$ is the set of frequencies, and their
  associated phasors,
  \[
  \mbox{Spectrum}\left( x(t) \right) =
  \left\{ (f_{-N},a_{-N}), \ldots, (f_0,a_0), \ldots, (f_N,a_N) \right\}
  \]
  such that
  \[
  x(t) = \sum_{k=-N}^N a_ke^{j2\pi f_kt}
  \]
\end{frame}

\begin{frame}
  \frametitle{Review: Fourier Series}

  One reason the spectrum is useful is that {\bf\em any} periodic
  signal can be written as a sum of cosines.  Fourier's theorem says that
  any $x(t)$ that is periodic, i.e.,
  \[
  x(t+T_0) = x(t)
  \]
  can be written as
  \[
  x(t) = \sum_{k=-\infty}^\infty X_k e^{j2\pi k F_0 t}
  \]
  which is a special case of the spectrum for periodic signals:
  $f_k=kF_0$, and $a_k=X_k$, and
  \[
  F_0 = \frac{1}{T_0}
  \]
\end{frame}

\begin{frame}
  \begin{itemize}
  \item {\bf Fourier Series Analysis}  (finding the spectrum, given the waveform):
    \[
    X_k = \frac{1}{T_0}\int_0^{T_0} x(t)e^{-j2\pi kt/T_0}dt
    \]
  \item {\bf Fourier Series Synthesis}  (finding the waveform, given the spectrum):
    \[
    x(t) = \sum_{k=-\infty}^\infty X_k e^{j2\pi kt/T_0}
    \]
  \end{itemize}
\end{frame}  

%%%%%%%%%%%%%%%%%%%%%%%%%%%%%%%%%%%%%%%%%%%%
\section[Frequency Response]{Frequency Response}
\setcounter{subsection}{1}

\begin{frame}
  \begin{block}{Frequency Response}
    The {\bf frequency response}, $H(\omega)$, of a filter $h[n]$, is
    its output in response to a pure tone, expressed as a function of
    the frequency of the tone.
  \end{block}
\end{frame}

\begin{frame}
  \frametitle{Calculating the Frequency Response}
  \begin{itemize}
  \item {\bf Output of the filter:}
    \begin{align*}
      y[n] &= h[n]\ast x[n]\\
      &= \sum_m h[m] x[n-m]
    \end{align*}
  \item {\bf in response to a pure tone:}
    \[
    x[n] = e^{j\omega n}
    \]
  \end{itemize}
\end{frame}

\begin{frame}
  \frametitle{Calculating the Frequency Response}

  \noindent{\bf Output of the filter in response  to a pure tone:}
  
  \begin{align*}
    y[n] &= \sum_m h[m] x[n-m]\\
    &= \sum_m h[m] e^{j\omega (n-m)}\\
    &= e^{j\omega n} \left(\sum_m h[m] e^{-j\omega m}\right)
  \end{align*}
  Notice that the part inside the parentheses is not a function of
  $n$.  It is not a function of $m$, because the $m$ gets summed over.
  It is only a function of $\omega$.  It is called the frequency
  response of the filter, $H(\omega)$.
\end{frame}

\begin{frame}
  \begin{block}{Frequency Response}
    When the input to a filter is a pure tone,
    \[
    x[n] =e^{j\omega n},
    \]
    then its output is the same pure tone, scaled and phase shifted by  a  complex number
    called the {\bf frequency response} $H(\omega)$:
    \[
    y[n] = H(\omega) e^{j\omega n}
    \]
    The frequency response is related to the impulse response as
    \[
    H(\omega) = \sum_m h[m]e^{-j\omega m}
    \]
  \end{block}
\end{frame}

%%%%%%%%%%%%%%%%%%%%%%%%%%%%%%%%%%%%%%%%%%%%
\section[Example]{Example: First Difference}
\setcounter{subsection}{1}

\begin{frame}
  \frametitle{Example: First Difference}

  \centerline{\includegraphics[height=1in]{exp/differenced_unitstep.png}}
  
  Remember that taking the difference between samples can be written as a convolution:
  \[ y[n] = x[n]-x[n-1]= h[n]\ast x[n],\]
  where
  \[
  h[n]=\begin{cases}1 & n=0\\-1&n=1\\0&\mbox{otherwise}\end{cases}
  \]
\end{frame}

\begin{frame}
  \frametitle{Example: First Difference}

  Suppose that the input is a pure tone:
  \[
  x[n] = e^{j\omega n}
  \]
  Then the output will be
  \begin{align*}
    y[n] &= x[n]-x[n-1]\\
    &= e^{j\omega n} -e^{j\omega (n-1)}\\
    &= H(\omega) e^{j\omega n},
  \end{align*}
  where
  \[
  H(\omega) = 1-e^{-j\omega}
  \]
\end{frame}

\begin{frame}
  \frametitle{Example: First Difference}

  So we have some pure-tone input,
  \[
  x[n]=e^{j\omega n}
  \]
  \ldots and we send it through a first-difference system:
  \[
  y[n] = x[n]-x[n-1]
  \]
  \ldots and what we get, at the output, is a pure tone, scaled by
  $|H(\omega)|$, and phase-shifted by $\angle H(\omega)$:
  \[
  y[n] = H(\omega) e^{j\omega n}
  \]
\end{frame}
  
\begin{frame}
  \frametitle{Example: First Difference}

  \begin{itemize}
  \item How much is the scaling?
  \item How much is the phase shift?
  \end{itemize}
  Let's find out.
  \begin{align*}
    H(\omega) &= 1-e^{-j\omega}\\
    &= e^{-j\frac{\omega}{2}} \left(e^{j\frac{\omega}{2}}-e^{-j\frac{\omega}{2}}\right)\\
    &= e^{-j\frac{\omega}{2}} \left(2j\sin\left(\frac{\omega}{2}\right)\right)\\
    &= \left(2\sin\left(\frac{\omega}{2}\right)\right)\left(e^{j\left(\frac{\pi-\omega}{2}\right)}\right)
  \end{align*}
  So the magnitude and phase response are:
  \begin{align*}
    |H(\omega)| &= 2\sin\left(\frac{\omega}{2}\right)\\
    \angle H(\omega) &= \frac{\pi-\omega}{2}
  \end{align*}
\end{frame}

\begin{frame}
  \frametitle{First Difference: Magnitude Response}

  \begin{align*}
    |H(\omega)| &= 2\sin\left(\frac{\omega}{2}\right)
  \end{align*}

  \centerline{\includegraphics[height=2in]{exp/firstdiffonly_magnitude.png}}
\end{frame}

\begin{frame}
  \frametitle{First Difference Filter at $\omega=0$}

  \centerline{\includegraphics[height=2in]{exp/firstdiffonly_magnitude.png}}

  Suppose we put in the signal $x[n]=e{j\omega n}$, but at the frequency $\omega=0$.
  At that frequency, $x[n]=1$.  So
  \[
  y[n] = x[n]-x[n-1] = 0
  \]
\end{frame}


\begin{frame}
  \frametitle{First Difference Filter at $\omega=\pi$}
  \centerline{\includegraphics[height=1.5in]{exp/firstdiffonly_magnitude.png}}
  Frequency $\omega=\pi$ means the input is $(-1)^n$:
  \[
  x[n] = e^{j\pi n} = (-1)^n= \begin{cases}
    1 & n~\mbox{is even}\\
    -1 & n~\mbox{is odd}
  \end{cases}
  \]
  So
  \[
  y[n]=x[n]-x[n-1] = 2x[n]
  \]
\end{frame}

\begin{frame}
  \frametitle{First Difference Filter at $\omega=\frac{\pi}{2}$}
  \centerline{\includegraphics[height=1in]{exp/firstdiffonly_magnitude.png}}
  Frequency $\omega=\frac{\pi}{2}$ means the input is $j^n$:
  \[
  x[n] = e^{j\frac{\pi n}{2}} = j^n
  \]
  The frequency response is:
  \[
  G\left(\frac{\pi}{2}\right) = 1-e^{-j\frac{\pi}{2}} = 1 - \left(\frac{1}{j}\right),
  \]
  The output is
  \[
  y[n] = x[n]-x[n-1] = j^n - j^{n-1} = \left(1-\frac{1}{j}\right)j^n
  \]
\end{frame}

%%%%%%%%%%%%%%%%%%%%%%%%%%%%%%%%%%%%%%%%%%%%
\section[Superposition]{Superposition and the Frequency Response}
\setcounter{subsection}{1}

\begin{frame}
  \begin{block}{Superposition and the Frequency Response}
    The frequency response obeys the principle of {\bf superposition}, meaning that
    if the input is the sum of two pure tones:
    \[
    x[n] = e^{j\omega_1 n} + e^{j\omega_2 n},
    \]
    then the output is the sum of the same two tones, each scaled by
    the corresponding frequency response:
    \[
    y[n] = H(\omega_1)e^{j\omega_1 n}+H(\omega_2)e^{j\omega_2 n}
    \]
  \end{block}
\end{frame}


\begin{frame}
  \frametitle{Response to a Cosine}
  There are no  complex exponentials in the real world.  Instead, we'd like to know
  the output in response to a cosine input.  Fortunately, a cosine
  is the sum of two complex exponentials:
  \[
  x[n] = \cos(\omega n) = \frac{1}{2}e^{j\omega n}+\frac{1}{2}e^{-j\omega n},
  \]
  therefore,
  \[
  y[n] = \frac{H(\omega)}{2}e^{j\omega n}+\frac{H(-\omega)}{2}e^{-j\omega n}
  \]
\end{frame}


\begin{frame}
  \frametitle{Response to a Cosine}
  What is $H(-\omega)$?  Remember the definition:
  \[
  H(\omega) = \sum_m h[m] e^{-j\omega m}
  \]
  Replacing every $\omega$ with a  $-\omega$ gives:
  \[
  H(-\omega) = \sum_m h[m] e^{j\omega m}.
  \]
  Notice that $h[m]$ is real-valued, so the only complex number on the RHS  is $e^{j\omega m}$.  But
  \[
  e^{j\omega m}=\left(e^{-j\omega m}\right)^*
  \]
  so
  \[
  H(-\omega) = H^*(\omega)
  \]
\end{frame}

\begin{frame}
  \frametitle{Response to a Cosine}
  \[
  y[n] = \frac{H(\omega)}{2}e^{j\omega n}+\frac{H^*(\omega)}{2}e^{-j\omega n}
  \]
  \[
  = \frac{|H(\omega)|}{2}e^{j\angle H(\omega)}e^{j\omega n} + 
  \frac{|H(\omega)|}{2}e^{-j\angle H(\omega)}e^{-j\omega n}
  \]
  \[
  = |H(\omega)|\cos\left(\omega n+\angle H(\omega)\right)
  \]
\end{frame}


\begin{frame}
  \begin{block}{Response to a Cosine}
    If
    \[
    x[n] = \cos(\omega n)
    \]
    \ldots then \ldots
    \[
    y[n] = |H(\omega)|\cos\left(\omega n+\angle H(\omega)\right)
    \]
  \end{block}
  \begin{block}{Magnitude and Phase Responses}
    \begin{itemize}
    \item The {\bf Magnitude Response} $|H(\omega)|$ tells you by how
      much a pure tone at $\omega$ will be scaled.
    \item The {\bf Phase Response} $\angle H(\omega)$ tells you by how much
      a pure tone at $\omega$ will be advanced in phase.
    \end{itemize}
  \end{block}
\end{frame}


%%%%%%%%%%%%%%%%%%%%%%%%%%%%%%%%%%%%%%%%%%%%
\section[Example]{Example: First Difference}
\setcounter{subsection}{1}

\begin{frame}
  \frametitle{Example: First Difference}

  Remember that the first difference, $y[n]=x[n]-x[n-1]$, is supposed
  to sort of approximate a derivative operator:
  \[
  y(t) \approx \frac{d}{dt} x(t)
  \]
  If the input is a cosine, what is the output?
  \[
  \frac{d}{dt} \cos\left(\omega t\right) = -\omega \sin\left(\omega t\right)
  = \omega \cos\left(\omega t+\frac{\pi}{2}\right)
  \]
  Does the first-difference operator behave the same way (multiply by
  a magnitude of $|H(\omega)|=\omega$, phase shift by $+\frac{\pi}{2}$
  radians so that cosine turns into negative sine)?
\end{frame}

\begin{frame}
  \frametitle{Example: First Difference}

  Freqeuncy response of the first difference filter is
  \[
  H(\omega) = 1-e^{-j\omega}
  \]
  Let's try to convert it to polar form, so we can find its magnitude and phase:
  \begin{align*}
    H(\omega) &= e^{-j\frac{\omega}{2}} \left(e^{j\frac{\omega}{2}}-e^{-j\frac{\omega}{2}}\right)\\
    &= e^{-j\frac{\omega}{2}} \left(2j\sin\left(\frac{\omega}{2}\right)\right)\\
    &= \left(2\sin\left(\frac{\omega}{2}\right)\right)\left(e^{j\left(\frac{\pi-\omega}{2}\right)}\right)
  \end{align*}
  So the magnitude and phase response are:
  \begin{align*}
    |H(\omega)| &= 2\sin\left(\frac{\omega}{2}\right)\\
    \angle H(\omega) &= \frac{\pi-\omega}{2}
  \end{align*}
\end{frame}

\begin{frame}
  \frametitle{First Difference: Magnitude Response}
  Taking the derivative of a cosine scales it by $\omega$.  The first-difference
  filter scales it by $|H(\omega)|=2\sin(\omega/2)$, which is almost the same, but not quite:
  \centerline{\includegraphics[height=2in]{exp/firstdiff_magnitude.png}}
\end{frame}


\begin{frame}
  \frametitle{First Difference: Phase Response}

  Taking the derivative of a cosine shifts it, in phase, by
  $+\frac{\pi}{2}$ radians, so that the cosine turns into a negative
  sine.  The first-difference filter shifts it by $\angle
  H(\omega)=\frac{\pi-\omega}{2}$, which is the same at very low frequencies,
  but very different at high frequencies.
  \centerline{\includegraphics[height=2in]{exp/firstdiff_phase.png}}
\end{frame}

\begin{frame}
  \frametitle{First Difference: All Together}
  Putting it all together, if the input is $x[n]=\cos(\omega n)$, the output is
  \[
  y[n]=|H(\omega)|\cos\left(\omega n+\angle H(\omega)\right)
  =2\sin\left(\frac{\omega}{2}\right)\cos\left(\omega n+\frac{\pi-\omega}{2}\right)
  \]
\end{frame}

\begin{frame}
  \frametitle{First Difference: All Together}
  \[
  y[n]=2\sin\left(\frac{\omega}{2}\right)\cos\left(\omega n+\frac{\pi-\omega}{2}\right)
  \]
  At very low frequencies, the output is almost $-\sin(\omega n)$, but
  with very low amplitude:
  \centerline{\includegraphics[height=2in]{exp/firstdiff_tonesweep10.png}}
\end{frame}


\begin{frame}
  \frametitle{First Difference: All Together}
  \[
  y[n]=2\sin\left(\frac{\omega}{2}\right)\cos\left(\omega n+\frac{\pi-\omega}{2}\right)
  \]
  At intermediate frequencies, the phase shift between the input and output is reduced:
  \centerline{\includegraphics[height=2in]{exp/firstdiff_tonesweep75.png}}
\end{frame}

\begin{frame}
  \frametitle{First Difference: All Together}
  \[
  y[n]=2\sin\left(\frac{\omega}{2}\right)\cos\left(\omega n+\frac{\pi-\omega}{2}\right)
  \]
  At very high frequencies, the phase shift between input and output
  is eliminated -- the output is a cosine, just like the input:
  \centerline{\includegraphics[height=2in]{exp/firstdiff_tonesweep150.png}}
\end{frame}


%%%%%%%%%%%%%%%%%%%%%%%%%%%%%%%%%%%%%%%%%%%%
\section[Example]{Written Examples}
\setcounter{subsection}{1}

\begin{frame}
  \frametitle{Written Example}
  Consider the following system:
  \[
  y[n] = x[n-n_0]
  \]
  This can be written as a convolution, with impulse response
  \[
  h[n] = \delta[n-n_0]
  \]
  What is $H(\omega)$?  Using $H(\omega)$, can you find the output of this system if
  $x[n]=\cos(\omega n)$?
\end{frame}


%%%%%%%%%%%%%%%%%%%%%%%%%%%%%%%%%%%%%%%%%%%%
\section[Summary]{Summary}
\setcounter{subsection}{1}

\begin{frame}
  \frametitle{Summary}
  \begin{itemize}
  \item {\bf Tones in $\rightarrow$ Tones out}
    \begin{align*}
      x[n]=e^{j\omega n} &\rightarrow y[n]=H(\omega)e^{j\omega n}\\
      x[n]=\cos\left(\omega n\right)
      &\rightarrow y[n]=|H(\omega)|\cos\left(\omega n+\angle H(\omega)\right)\\
      x[n]=A\cos\left(\omega n+\theta\right)
      &\rightarrow y[n]=A|H(\omega)|\cos\left(\omega n+\theta+\angle H(\omega)\right)
    \end{align*}
  \item where the {\bf Frequency Response} is given by
    \[
    H(\omega) = \sum_m h[m]e^{-j\omega m}
    \]
  \end{itemize}
\end{frame}  
        
\end{document}
