\documentclass{beamer}
\usepackage{tikz,amsmath,hyperref,graphicx,stackrel,animate}
\usetikzlibrary{positioning,shadows,arrows,shapes,calc,dsp,chains}
\newcommand{\argmax}{\operatornamewithlimits{argmax}}
\newcommand{\argmin}{\operatornamewithlimits{argmin}}
\mode<presentation>{\usetheme{Frankfurt}}
\AtBeginSection[]
{
  \begin{frame}<beamer>
    \frametitle{Outline}
    \tableofcontents[currentsection,currentsubsection]
  \end{frame}
}
\title{Lecture 11: Linearity and Shift-Invariance}
\author{Mark Hasegawa-Johnson\\These slides are in the public domain.}
\date{ECE 401: Signal and Image Analysis, Fall 2023}  
\begin{document}

% Title
\begin{frame}
  \maketitle
\end{frame}

% Title
\begin{frame}
  \tableofcontents
\end{frame}

%%%%%%%%%%%%%%%%%%%%%%%%%%%%%%%%%%%%%%%%%%%%
\section[Systems]{Systems}
\setcounter{subsection}{1}

\begin{frame}
  \frametitle{What is a System?}

  A {\bf system} is anything that takes one signal as input, and
  generates another signal as output.   We can write
  \[
  x[n] \stackrel{\mathcal H}{\longrightarrow} y[n]
  \]
  which means
  \centerline{
    \begin{tikzpicture}
      \node[dspnodeopen,dsp/label=right] (y) at (5,0) {$y[n]$};
      \node[dspsquare] (f) at (2.5,0) {${\mathcal H}$} edge[dspconn](y);
      \node[dspnodeopen,dsp/label=left] (x) at (0,0) {$x[n]$} edge[dspconn](f);
    \end{tikzpicture}}
\end{frame}

\begin{frame}
  \frametitle{Example: Averager}

  For example, a weighted local averager is a system.  Let's call it
  system ${\mathcal A}$.
  \[
  x[n]\stackrel{\mathcal A}{\longrightarrow} y[n]=\sum_{m=0}^6 g[m] x[n-m]
  \]
\end{frame}

\begin{frame}
  \frametitle{Example: Time-Shift}

  A time-shift is a system.  Let's call it
  system ${\mathcal T}$.
  \[
  x[n]\stackrel{\mathcal T}{\longrightarrow} y[n]=x[n-1]
  \]
\end{frame}

\begin{frame}
  \frametitle{Example: Square}

  If you calculate the square of a signal, that's also a system.
  Let's call it system ${\mathcal S}$:
  \[
  x[n]\stackrel{\mathcal S}{\longrightarrow} y[n]=x^2[n]
  \]
  
\end{frame}

\begin{frame}
  \frametitle{Example: Add a Constant}

  If you add a constant to a signal, that's also a system.  Let's call
  it system ${\mathcal C}$:
  \[
  x[n]\stackrel{\mathcal C}{\longrightarrow} y[n]=x[n]+1
  \]
  
\end{frame}

\begin{frame}
  \frametitle{Example: Window}

  If you chop off all elements of a signal that are before time $0$ or
  after time $N-1$ (for example, because you want to put it into an
  image), that is a system:
  \[
  x[n]\stackrel{\mathcal W}{\longrightarrow} y[n]=\begin{cases}x[n]&0\le n\le N-1\\0&\mbox{otherwise}
  \end{cases}
  \]
  
\end{frame}



%%%%%%%%%%%%%%%%%%%%%%%%%%%%%%%%%%%%%%%%%%%%
\section[Linearity]{Linearity}
\setcounter{subsection}{1}

\begin{frame}
  \frametitle{Linearity}
  A system is {\bf linear} if these two algorithms compute the same thing:
  
  \begin{columns}
    \begin{column}{0.5\textwidth}
      \begin{center}
        \begin{tikzpicture}
          \node[dspnodeopen,dsp/label=above] (y) at (4.5,0) {?};
          \node[dspadder] (add) at (3,0) {} edge[dspconn](y);
          \node[dspsquare] (f) at (1.5,1.5) {${\mathcal H}$} edge[dspflow](add);
          \node[dspnodeopen,dsp/label=left] (x) at (0,1.5) {$x_1[n]$} edge[dspconn](f);
          \node[dspsquare] (f) at (1.5,0) {${\mathcal H}$} edge[dspflow](add);
          \node[dspnodeopen,dsp/label=left] (x) at (0,0) {$x_2[n]$} edge[dspconn](f);
          \node[dspsquare] (f) at (1.5,-1.5) {$\mathcal H$} edge[dspflow](add);
          \node[dspnodeopen,dsp/label=left] (x) at (0,-1.5) {$x_3[n]$} edge[dspconn](f);
        \end{tikzpicture}
      \end{center}
    \end{column}
    \begin{column}{0.5\textwidth}
      \begin{center}
        \begin{tikzpicture}
          \node[dspnodeopen,dsp/label=above] (y) at (4.5,0) {?};
          \node[dspsquare] (f) at (3,0) {$\mathcal H$} edge[dspconn](y);
          \node[dspadder] (add) at (1.5,0) {} edge[dspconn](f);
          \node[dspnodeopen,dsp/label=left] (x) at (0,1.5) {$x_1[n]$} edge[dspconn](add);
          \node[dspnodeopen,dsp/label=left] (x) at (0,0) {$x_2[n]$} edge[dspconn](add);
          \node[dspnodeopen,dsp/label=left] (x) at (0,-1.5) {$x_3[n]$} edge[dspconn](add);
        \end{tikzpicture}
      \end{center}
    \end{column}
  \end{columns}
\end{frame}


\begin{frame}
  \frametitle{Linearity}

  A system ${\mathcal H}$ is said to be {\bf linear} if and only if,
  for any $x_1[n]$ and $x_2[n]$,
  \begin{align*}
    x_1[n] &\stackrel{\mathcal H}{\longrightarrow} y_1[n]\\
    x_2[n] &\stackrel{\mathcal H}{\longrightarrow} y_2[n]
  \end{align*}
  implies that
  \[
  x[n]=x_1[n]+x_2[n] \stackrel{\mathcal H}{\longrightarrow} y[n]=y_1[n]+y_2[n]
  \]
  In words: a system is {\bf linear} if and only if, for every pair of
  inputs $x_1[n]$ and $x_2[n]$, (1) adding the inputs and then passing
  them through the system gives exactly the same effect as (2) passing
  both inputs through the system, and {\bf then} adding them.
\end{frame}


\begin{frame}
  \frametitle{Special case of linearity: Scaling}

  Notice, a special case of linearity is the case when $x_1[n]=x_2[n]$:
  \begin{align*}
    x_1[n] &\stackrel{\mathcal H}{\longrightarrow} y_1[n]\\
    x_1[n] &\stackrel{\mathcal H}{\longrightarrow} y_1[n]
  \end{align*}
  implies that
  \[
  x[n]=2x_1[n] \stackrel{\mathcal H}{\longrightarrow} y[n]=2y_1[n]
  \]
  So if a system is linear, then {\bf scaling the input} also {\bf
    scales the output}.
\end{frame}


\begin{frame}
  \frametitle{Example: Averager}

  Let's try it with the weighted averager.
  \begin{align*}
  x_1[n] &\stackrel{\mathcal A}{\longrightarrow} y_1[n]=\sum_{m=0}^6 g[m] x_1[n-m]\\
  x_2[n] &\stackrel{\mathcal A}{\longrightarrow} y_2[n]=\sum_{m=0}^6 g[m] x_2[n-m]
  \end{align*}
  Then:
  \begin{align*}
    x[n]=x_1[n]+x_2[n] 
    &=\sum_{m=0}^6 g[m] \left(x_1[n-m]+x_2[n-m]\right)\\
    &=\left(\sum_{m=0}^6 g[m]x_1[n-m]\right)+\left(\sum_{m=0}^6g[m]x_2[n-m]\right)\\
    &=y_1[n]+y_2[n]
  \end{align*}
  \ldots so a weighted averager is a {\bf linear system}.
\end{frame}

\begin{frame}
  \frametitle{Example: Square}

  A squarer is just obviously nonlinear, right?  Let's see if that's true:
  \begin{align*}
  x_1[n] &\stackrel{\mathcal S}{\longrightarrow} y_1[n]=x_1^2[n]\\
  x_2[n] &\stackrel{\mathcal S}{\longrightarrow} y_2[n]=x_2^2[n]
  \end{align*}
  Then:
  \begin{align*}
    x[n]=x_1[n]+x_2[n] &\stackrel{\mathcal A}{\longrightarrow} y[n]=x^2[n]\\
    &= \left(x_1[n]+x_2[n]\right)^2\\
    &=x_1^2[n]+2x_1[n]x_2[n] + x_2^2[n]\\
    &\ne y_1[n]+y_2[n]
  \end{align*}
  \ldots so a squarer is a {\bf nonlinear system}.
\end{frame}

\begin{frame}
  \frametitle{Example: Add a Constant}

  This one is tricky.  Adding a constant seems like it ought to be
  linear, but it's actually {\bf nonlinear}. Adding a constant is
  what's called an {\bf affine} system, which is not necessarily linear.
  \begin{align*}
  x_1[n] &\stackrel{\mathcal C}{\longrightarrow} y_1[n]=x_1[n]+1\\
  x_2[n] &\stackrel{\mathcal C}{\longrightarrow} y_2[n]=x_2[n]+1
  \end{align*}
  Then:
  \begin{align*}
    x[n]=x_1[n]+x_2[n] &\stackrel{\mathcal A}{\longrightarrow} y[n]=x[n]+1\\
    &= x_1[n]+x_2[n]+1\\
    &\ne y_1[n]+y_2[n]
  \end{align*}
  \ldots so adding a constant is a {\bf nonlinear system}.
\end{frame}

\begin{frame}
  \frametitle{What about the real world?}

  Suppose you're showing people images $x[n]$, and measuring their
  brain activity $y[n]$ as a result.  How can you tell if this system
  is linear?
  \begin{itemize}
  \item Show them one image, call it $x_1[n]$.  Measure the resulting
    brain activity, $y_1[n]$.
  \item Show them another image, $x_2[n]$.  Measure the brain
    activity, $y_2[n]$.
  \item Show them $x[n]=x_1[n]+x_2[n]$.  Measure $y[n]$.  Is it equal
    to $y_1[n]+y_2[n]$?
  \item Repeat this experiment with lots of different images, and
    their sums, until you are convinced that the system is linear (or not).
  \end{itemize}
\end{frame}



%%%%%%%%%%%%%%%%%%%%%%%%%%%%%%%%%%%%%%%%%%%%
\section[Shift Invariance]{Shift Invariance}
\setcounter{subsection}{1}

\begin{frame}
  \frametitle{Shift Invariance}

  A system ${\mathcal H}$ is {\bf shift-invariant} if these two
  algorithms compute the same thing (here ${\mathcal T}$ means ``time
  shift''):
  
  \begin{center}
    \begin{tikzpicture}
      \node[dspnodeopen,dsp/label=above] (y) at (8,0) {?};
      \node[dspsquare] (h) at (6,0) {${\mathcal H}$} edge[dspconn](y);
      \node[dspnodeopen] (s) at (4,0) {$x[n-1]$} edge[dspconn](h);
      \node[dspsquare] (t) at (2,0) {${\mathcal T}$} edge[dspconn](s);
      \node[dspnodeopen,dsp/label=left] (x) at (0,0) {$x[n]$} edge[dspconn](t);
    \end{tikzpicture}
  \end{center}
  \begin{center}
    \begin{tikzpicture}
      \node[dspnodeopen,dsp/label=above] (y) at (8,0) {$y[n-1]$};
      \node[dspsquare] (t) at (6,0) {${\mathcal T}$} edge[dspconn](y);
      \node[dspnodeopen] (s) at (4,0) {$y[n]$} edge[dspconn](t);
      \node[dspsquare] (h) at (2,0) {${\mathcal H}$} edge[dspconn](s);
      \node[dspnodeopen,dsp/label=left] (x) at (0,0) {$x[n]$} edge[dspconn](h);
    \end{tikzpicture}
  \end{center}
\end{frame}


\begin{frame}
  \frametitle{Shift Invariance}

  A system ${\mathcal H}$ is said to be {\bf shift-invariant} if and
  only if, for every $x_1[n]$,
  \begin{align*}
    x_1[n] &\stackrel{\mathcal H}{\longrightarrow} y_1[n]
  \end{align*}
  implies that
  \[
  x[n]=x_1[n-n_0] \stackrel{\mathcal H}{\longrightarrow} y[n]=y_1[n-n_0]
  \]
  In words: a system is {\bf shift-invariant} if and only if, for any
  input $x_1[n]$, (1) shifting the input by some number of samples
  $n_0$, and then passing it through the system, gives exactly the
  same result as (2) passing it through the system, and then shifting
  it.
\end{frame}

\begin{frame}
  \frametitle{Example: Averager}

  Let's try it with the weighted averager.
  \begin{align*}
  x_1[n] &\stackrel{\mathcal A}{\longrightarrow} y_1[n]=\sum_{m=0}^6 g[m] x_1[n-m]
  \end{align*}
  Then:
  \begin{align*}
    x[n]=x_1[n-n_0] &\stackrel{\mathcal A}{\longrightarrow} y[n]=\sum_{m=0}^6 g[m] x[n-m]\\
    &=\sum_{m=0}^6 g[m] x_1\left[(n-m)-n_0\right]\\
    &=\sum_{m=0}^6 g[m] x_1\left[(n-n_0)-m\right]\\
    &=y_1[n-n_0]
  \end{align*}
  \ldots so a weighted averager is a {\bf shift-invariant system}.
\end{frame}

\begin{frame}
  \frametitle{Example: Square}

  Squaring the input is a nonlinear operation, but is it shift-invariant?
  Let's find out:
  \begin{align*}
  x_1[n] &\stackrel{\mathcal S}{\longrightarrow} y_1[n]=x_1^2[n]
  \end{align*}
  Then:
  \begin{align*}
    x[n]=x_1[n-n_0] &\stackrel{\mathcal A}{\longrightarrow} y[n]=x^2[n]\\
    &= \left(x_1[n-n_0]\right)^2\\
    &= x_1^2[n-n_0]\\
    &=y_1[n-n_0]
  \end{align*}
  \ldots so computing the square is a {\bf shift-invariant system}.
\end{frame}


\begin{frame}
  \frametitle{Example: Windowing}

  How about windowing, e.g., in order to create an image?
  \begin{align*}
    x_1[n] &\stackrel{\mathcal W}{\longrightarrow} y_1[n]=
    \begin{cases}
      x_1[n] & 0\le n\le N-1\\
      0 & \mbox{otherwise}
    \end{cases}
  \end{align*}
  If we shift the {\bf output}, we get
  \[
  y_1[n-n_0] = 
  \begin{cases}
    x_1[n-n_0] & n_0\le n\le N-1+n_0\\
    0 & \mbox{otherwise}
  \end{cases}
  \]
  \ldots but if we shift the {\bf input} ($x[n]=x_1[n-n_0]$), we get
  \begin{align*}
  y[n]&=
  \begin{cases}
    x[n] & 0\le n\le N-1\\
    0 & \mbox{otherwise}
  \end{cases}
  =
  \begin{cases}
    x_1[n-n_0] & 0\le n\le N-1\\
    0 & \mbox{otherwise}
  \end{cases}\\
  &\ne y_1[n-n_0]
  \end{align*}
  \ldots so windowing is a {\bf shift-varying system} (not shift-invariant).
\end{frame}

\begin{frame}
  \frametitle{How about the real world?}

  Suppose you're showing images $x[n]$, and measuring the neural
  response $y[n]$.  How do you determine if this system is
  shift-invariant?
  \begin{itemize}
  \item Show an image $x_1[n]$, and measure the neural response $y_1[n]$.
  \item Shift the image by $n_0$ columns to the right, to get the
    image $x[n]=x_1[n-n_0]$.  Show people $x[n]$.
  \item Is the resulting neural response exactly the same, but shifted
    to a slightly different set of neurons (shifted ``to the right?'')
    If so, then the system may be shift-invariant!
  \item Keep doing that, with many different images and many different
    shifts, until you're convinced the system is shift-invariant.
  \end{itemize}
\end{frame}

%%%%%%%%%%%%%%%%%%%%%%%%%%%%%%%%%%%%%%%%%%%%
\section[Example]{Written Example}
\setcounter{subsection}{1}

\begin{frame}
  \frametitle{Written Example}

  Prove that differentiation, $y(t)=\frac{dx}{dt}$, is a linear
  shift-invariant system (in terms of $t$ as the time index, instead
  of $n$).
\end{frame}

%%%%%%%%%%%%%%%%%%%%%%%%%%%%%%%%%%%%%%%%%%%%
\section[Summary]{Summary}
\setcounter{subsection}{1}

\begin{frame}
  \frametitle{Summary}
  \begin{itemize}
  \item A system is {\bf linear} if and only if, for any two inputs
    $x_1[n]$ and $x_2[n]$ that produce outputs $y_1[n]$ and $y_2[n]$,
    \[
    x[n]=x_1[n]+x_2[n] \stackrel{\mathcal H}{\longrightarrow}  y[n]=y_1[n]+y_2[n]
    \]
  \item A system is {\bf shift-invariant} if and only if, for any input
    $x_1[n]$ that produces output $y_1[n]$,
    \[
    x[n]=x_1[n-n_0] \stackrel{\mathcal H}{\longrightarrow}  y[n]=y_1[n-n_0]
    \]
  \end{itemize}
\end{frame}
    

\end{document}
