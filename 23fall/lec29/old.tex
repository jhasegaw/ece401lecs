\begin{frame}
  \begin{block}{General form of an FIR filter}
    \[
    y[n] = \sum_{k=0}^{M} b_k x[n-k]
    \]
    This filter has an impulse response ($h[n]$) that is $M+1$ samples
    long.
  \end{block}
  \begin{block}{General form of an IIR filter}
    \[
    \sum_{\ell=1}^N a_\ell y[n-\ell] = \sum_{k=0}^{M} b_k x[n-k]
    \]
  \end{block}
\end{frame}

\begin{frame}
  \frametitle{Feedback and Feedforward Coefficients}

  \begin{itemize}
  \item The general form of an FIR filter is $y[n] = \sum_{k=0}^{M}
    b_k x[n-k]$.  The $b_k$'s are called {\bf feedforward}
    coefficients, because they feed $x[n]$ forward into $y[n]$.  The
    impulse response has a length of exactly $M+1$ samples; in fact,
    it's given by
    \[h[n] = \begin{cases} b_n & 0\le n\le M\\0& \mbox{otherwise}\end{cases}\]
  \item The general form of an IIR filter is $\sum_{\ell=1}^N a_\ell
    y[n-\ell] = \sum_{k=0}^{M} b_k x[n-k]$.  The $a_\ell$'s are caled
    {\bf feedback} coefficients, because they feed $y[n]$ back into
    itself.  The impulse response is infinite length.  In order to
    find its general form, we need a bit more math.
  \end{itemize}
\end{frame}


%%%%%%%%%%%%%%%%%%%%%%%%%%%%%%%%%%%%%%%%%%%%
\section[Second-Order]{Impulse Response and Transfer Function of a Second-Order Filter}
\setcounter{subsection}{1}

\begin{frame}
  \frametitle{Second-Order Filter: Example}

  Let's take an example of a second-order filter (a filter with two
  feedback terms):
  \[
  y[n] = x[n] + y[n-1] - \frac{1}{2}y[n-2]
  \]
  Remember the transfer function of this filter is
  \[
  H(z) = \frac{1}{1-z^{-1}+\frac{1}{2}z^{-2}}
  \]
\end{frame}

\begin{frame}
  \frametitle{Factoring the Transfer Function}

  Notice that we can write the transfer function (just multiply both
  top and bottom by $z^2$):
  \[
  H(z) = \frac{z^2}{z^2-z+\frac{1}{2}}
  \]
  Now, notice that this polynomial can be factored:
  \[
  z^2-z+\frac{1}{2} = (z-p_1)(z-p_2),
  \]
  where $p_1$ and $p_2$ are the zeros of the polynomial.  Using the
  quadratic formula, we can find that they are
  \[
  p_1,p_2 = \frac{1}{2}\pm \frac{j}{2} = \frac{\sqrt{2}}{2}e^{\pm j\frac{\pi}{4}}
  \]
\end{frame}

\begin{frame}
  \frametitle{Factoring the Transfer Function}

  So the transfer function is:
  \[
  H(z) = \frac{z^2}{(z-p_1)(z-p_2)}
  \]
  where $p_1$ and $p_2$ are
  \[
  p_1,p_2 = \frac{1}{2}\pm \frac{j}{2} = \frac{\sqrt{2}}{2}e^{\pm j\frac{\pi}{4}}
  \]
\end{frame}

\begin{frame}
  \frametitle{From Second Order to First Order}

  So the transfer function is:
  \[
  H(z) = \frac{z^2}{(z-p_1)(z-p_2)} = \frac{1}{(1-p_1z^{-1})(1-p_2z^{-1})}
  \]
  On the other hand, if it was just a first-order fraction, like $\frac{1}{1-p_1z^{-1}}$, then
  we would already know the solution:
  \[
  \frac{1}{1-p_1z^{-1}} \leftrightarrow p_1^n u[n]
  \]
  Is there any way that we can break the second-order transfer function down into
  two first-order functions?  Yes, in fact, there are two different ways to do it.
\end{frame}

\begin{frame}
  \frametitle{From Second Order to First Order}

  \begin{enumerate}
  \item {\bf Factoring:}  We could factor $H(z)$, as
    \[
    H(z) = H_1(z)H_2(z) \leftrightarrow h_1[n]\ast h_2[n]
    \]
    The problem with this method is that we would need to convolve $h_1[n]\ast h_2[n]$, which
    is kind of time-consuming.  It works, but it takes some effort.
  \item {\bf Partial Fraction Expansion:} A method that works faster is to write
    \[
    \frac{1}{(1-p_1z^{-1})(1-p_2z^{-1})} = \frac{c_1}{1-p_1z^{-1}}+\frac{c_2}{1-p_2z^{-1}}.
    This is called a {\bf partial fraction expansion}.
  \end{enumerate}
\end{frame}

\begin{frame}
  \frametitle{Factoring the Transfer Function}

  So the transfer function is:
  \[
  H(z) = \frac{z^2}{(z-p_1)(z-p_2)}
  \]
  where $p_1$ and $p_2$ are
  \[
  p_1,p_2 = \frac{1}{2}\pm \frac{j}{2} = \frac{\sqrt{2}}{2}e^{\pm j\frac{\pi}{4}}
  \]
\end{frame}

\begin{frame}
  \frametitle{Partial Fraction Expansion}

  Now here's a trick you may not have seen before.  This is called
  ``partial fraction expansion:''
  \[
  \frac{z^2}{(z-z_1)(z-z_2)} = \frac{zc_1}{z-z_1} + \frac{zc_2}{z-z_2}
  \]
  Partial fraction expansion always works, regardless of what $z_1$
  and $z_2$ are.  You just need to solve for the constants, $c_1$ and
  $c_2$.  You solve for them by multiplying both sides of the equation
  by the polynomial $(z-z_1)(z-z_2)$, giving
  \[
  z^2 = c_1z(z-z_1) + c_2z(z-z_2) = (c_1+c_2)z^2 - (c_1z_1+c_2z_2)z
  \]
  Obviously, this is satisfied as long as
  \begin{align*}
    c_1 + c_2 &= 1\\
    c_1z_1+c_2z_2 &= 0
  \end{align*}
  
  where $z_1$ and $z_2$ are
  \[
  z_1,z_2 = \frac{1}{2}\pm \frac{j}{2} = \frac{\sqrt{2}}{2}e^{\pm j\frac{\pi}{4}}
  \]
\end{frame}







\begin{frame}
  \frametitle{Feedback-Only Filter}

  Let's move to a slightly more difficult case, the case of a filter
  with an arbitrary amount of feedback, but with no extra feedforward
  terms besides $x[n]$.  Here's the general form:
  \[
  \sum_{\ell=0}^N a_\ell y[n-\ell] = x[n]
  \]
  Of course, the way you'd actually implement it, in a program, would be
  like this:
  \[
  y[n] = \frac{1}{a_0}\left(x[n] - \sum_{\ell=1}^N a_\ell y[n-\ell]\right)
  \]
  $y[n]$ is equal to $x[n]/a_0$, minus a weighted sum of past values
  of $y[n]$.  For convenience, let's assume $a_0=1$, so that
  \[
  y[n] = x[n] - \sum_{\ell=1}^N a_\ell y[n-\ell]
  \]
\end{frame}

\begin{frame}
  \frametitle{Transfer Function}

  Taking the Z transform of each term in the equation, we get
  \[
  \sum_{\ell=0}^N a_\ell z^{-\ell} Y(z) = X(z)
  \]
  Solving for $H(z)$, we find
  \[
  H(z) = \frac{Y(z)}{X(z)} = \frac{1}{\sum_{\ell=0}^N a_\ell z^{-\ell}}
  \]
\end{frame}

\begin{frame}
  \frametitle{Factoring the Transfer Function}

  Taking the Z transform of each term in the equation, we get
  \[
  Y(z) + \sum_{\ell=1}^N a_\ell z^{-\ell} Y(z) = X(z)
  \]
  Solving for $H(z)$, we find
  \[
  H(z) = \frac{Y(z)}{X(z)} = \frac{1}{1-\sum_{\ell=1}^N a_\ell z^{-\ell}}
  \]
\end{frame}
