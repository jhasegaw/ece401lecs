\documentclass{beamer}
\usepackage{tikz,amsmath,hyperref,graphicx,stackrel,animate,amssymb}
\usetikzlibrary{positioning,shadows,arrows,shapes,calc}
\newcommand{\argmax}{\operatornamewithlimits{argmax}}
\newcommand{\argmin}{\operatornamewithlimits{argmin}}
\mode<presentation>{\usetheme{Frankfurt}}
\AtBeginSection[]
{
  \begin{frame}<beamer>
    \frametitle{Outline}
    \tableofcontents[currentsection,currentsubsection]
  \end{frame}
}
\title{Lecture 23: Circular Convolution}
\author{Mark Hasegawa-Johnson}
\date{ECE 401: Signal and Image Analysis}
\begin{document}

% Title
\begin{frame}
  \maketitle
\end{frame}

% Title
\begin{frame}
  \tableofcontents
\end{frame}

%%%%%%%%%%%%%%%%%%%%%%%%%%%%%%%%%%%%%%%%%%%%
\section[Review]{Review: DTFT and DFT}
\setcounter{subsection}{1}

\begin{frame}
  \frametitle{Review: DTFT}

  The DTFT (discrete time Fourier transform) of any signal is
  $X(\omega)$, given by
  \begin{align*}
    X(\omega) &= \sum_{n=-\infty}^\infty x[n]e^{-j\omega n}\\
    x[n] &= \frac{1}{2\pi}\int_{-\pi}^\pi X(\omega)e^{j\omega n}d\omega
  \end{align*}
  Particular useful examples include:
  \begin{align*}
    f[n]=\delta[n] &\leftrightarrow F(\omega)=1\\
    g[n]=\delta[n-n_0] &\leftrightarrow G(\omega)=e^{-j\omega n_0}
  \end{align*}
\end{frame}

\begin{frame}
  \frametitle{Properties of the DTFT}

  Properties worth knowing  include:
  \begin{enumerate}
    \setcounter{enumi}{-1}
  \item Periodicity: $X(\omega+2\pi)=X(\omega)$
  \item Linearity:
    \[z[n]=ax[n]+by[n]\leftrightarrow Z(\omega)=aX(\omega)+bY(\omega)
    \]
  \item Time Shift: $x[n-n_0]\leftrightarrow e^{-j\omega n_0}X(\omega)$
  \item Frequency Shift: $e^{j\omega_0 n}x[n]\leftrightarrow X(\omega-\omega_0)$
  \item Filtering is Convolution:
    \[
    y[n]=h[n]\ast x[n]\leftrightarrow Y(\omega)=H(\omega)X(\omega)
    \]
  \end{enumerate}
\end{frame}


\begin{frame}
  \frametitle{Review: DFT}

  The DFT (discrete Fourier transform) of any signal is
  $X[k]$, given by
  \begin{align*}
    X[k] &= \sum_{n=0}^{N-1} x[n]e^{-j\frac{2\pi kn}{N}}\\
    x[n] &= \frac{1}{N}\sum_0^{N-1} X[k]e^{j\frac{2\pi kn}{N}}
  \end{align*}
  Particular useful examples include:
  \begin{align*}
    f[n]=\delta[n] &\leftrightarrow F[k]=1\\
    g[n]=\delta\left[(\!(n-n_0)\!)_N\right] &\leftrightarrow G[k]=e^{-j\frac{2\pi kn_0}{N}}
  \end{align*}
\end{frame}

\begin{frame}
  \frametitle{Properties of the DTFT}

  Properties worth knowing  include:
  \begin{enumerate}
    \setcounter{enumi}{-1}
  \item Periodicity: $X[k+N]=X[k]$
  \item Linearity:
    \[z[n]=ax[n]+by[n]\leftrightarrow Z[k]=aX[k]+bY[k]
    \]
  \item Circular Time Shift: $x\left[(\!(n-n_0)\!)_N\right]\leftrightarrow e^{-j\frac{2\pi kn_0}{N}}X(\omega)$
  \item Frequency Shift: $e^{j\frac{2\pi k_0 n}{N}}x[n]\leftrightarrow X[k-k_0]$
  \item Filtering is Circular Convolution:
    \[
      y[n]=h[n]\circledast x[n]\leftrightarrow Y[k]=H[k]X[k],
      \]
  \end{enumerate}
\end{frame}

%%%%%%%%%%%%%%%%%%%%%%%%%%%%%%%%%%%%%%%%%%%%
\section[Periodic in Time]{Sampled in Frequency $\leftrightarrow$ Periodic in Time}
\setcounter{subsection}{1}

\begin{frame}
  \frametitle{Two different ways to think about the DFT}
  \begin{block}{1. $x[n]$ is finite length; DFT is samples of DTFT}
    \begin{displaymath}
      x[n]=0,n<0~\text{or}~n\ge N~~~\leftrightarrow~~~X[k]=\left.X(\omega)\right|_{\omega=\frac{2\pi k}{N}}
    \end{displaymath}
  \end{block}
  \begin{block}{2. $x[n]$ is periodic; DFT is scaled version of Fourier series}
    \begin{displaymath}
      x[n]=x[n+N]~~~\leftrightarrow~~~X[k]=N X_k
    \end{displaymath}
  \end{block}
\end{frame}

\begin{frame}
  \frametitle{1. $x[n]$ finite length, DFT is samples of DTFT}

  If $x[n]$ is nonzero only for $0\le n\le N-1$, then
  \begin{displaymath}
    X(\omega) = \sum_{n=-\infty}^\infty x[n]e^{-j\omega n}
    = \sum_{n=0}^{N-1} x[n]e^{-j\omega n},
  \end{displaymath}
  and
  \begin{displaymath}
    X[k] = \left. X(\omega)\right|_{\omega=\frac{2\pi k}{N}}
  \end{displaymath}
  
\end{frame}

\begin{frame}
  \frametitle{2. $x[n]$ periodic, $X[k]=NX_k$}

  If $x[n]=x[n+N]$, then its Fourier series is
  \begin{align*}
    X_k &= \frac{1}{N}\sum_{n=1}^{N-1} x[n]e^{-j\frac{2\pi kn}{N}}\\
    x[n] &= \sum_{k=0}^{N-1} X_k e^{j\frac{2\pi kn}{N}},
  \end{align*}
  and its DFT is
  \begin{align*}
    X[k] &= \sum_{n=1}^{N-1} x[n]e^{-j\frac{2\pi kn}{N}}\\
    x[n] &= \frac{1}{N}\sum_{k=0}^{N-1} X[k] e^{j\frac{2\pi kn}{N}}
  \end{align*}
\end{frame}
  
\begin{frame}
  \frametitle{Delayed impulse wraps around}

  \begin{displaymath}
    \delta\left[(\!(n-n_0)\!)_N\right] \leftrightarrow e^{-j\frac{2\pi kn_0}{N}}
  \end{displaymath}
  
  \centerline{\animategraphics[loop,controls,width=\textwidth]{10}{exp/delaycircular}{0}{49}}

\end{frame}

\begin{frame}
  \frametitle{Delayed impulse is really periodic impulse}

  \begin{displaymath}
    \delta\left[(\!(n-n_0)\!)_N\right] \leftrightarrow e^{-j\frac{2\pi kn_0}{N}}
  \end{displaymath}
  
  \centerline{\animategraphics[loop,controls,width=\textwidth]{10}{exp/delayperiodic}{0}{49}}

\end{frame}

\begin{frame}
  \frametitle{Principal Phase}

  \begin{itemize}
  \item Something weird going on: how can the phase keep getting
    bigger and bigger, but the signal wraps around?
  \item It's because the phase wraps around too!
    \begin{displaymath}
      \angle X[k] = -\omega_k (N+n) = -\omega_k n,~~~\mbox{because}~\omega_k=\frac{2\pi k}{N}
    \end{displaymath}
  \item {\bf Principal phase =} add $\pm 2\pi$ to the phase, as
    necessary, so that $-\pi< \angle X[k]\le \pi$
  \item {\bf Unwrapped phase = } let the phase be as large as
    necessary so that it is plotted as a smooth function of $\omega$
  \end{itemize}
\end{frame}

\begin{frame}
  \frametitle{Unwrapped phase vs. Principal phase}

  \begin{displaymath}
    \delta\left[(\!(n-n_0)\!)_N\right] \leftrightarrow e^{-j\frac{2\pi kn_0}{N}}
  \end{displaymath}
  
  \centerline{\animategraphics[loop,controls,width=\textwidth]{10}{exp/principalphase}{0}{49}}

\end{frame}

\begin{frame}
  \frametitle{Summary: Two different ways to think about the DFT}
  \begin{block}{1. $x[n]$ is finite length; DFT is samples of DTFT}
    \begin{displaymath}
      x[n]=0,n<0~\text{or}~n\ge N~~~\leftrightarrow~~~X[k]=\left.X(\omega)\right|_{\omega=\frac{2\pi k}{N}}
    \end{displaymath}
  \end{block}
  \begin{block}{2. $x[n]$ is periodic; DFT is scaled version of Fourier series}
    \begin{displaymath}
      x[n]=x[n+N]~~~\leftrightarrow~~~X[k]=N X_k
    \end{displaymath}
  \end{block}
\end{frame}

%%%%%%%%%%%%%%%%%%%%%%%%%%%%%%%%%%%%%%%%%%%%
\section[Circular Convolution]{Circular Convolution}
\setcounter{subsection}{1}

\begin{frame}
  \frametitle{Multiplying two DFTs: what we think we're doing}

  \centerline{\includegraphics[width=\textwidth]{exp/convolutionlinear.png}}
\end{frame}

\begin{frame}
  \frametitle{Multiplying two DFTs: what we're actually doing}

  \centerline{\includegraphics[width=\textwidth]{exp/convolutionperiodic.png}}
\end{frame}

\begin{frame}
  \frametitle{Circular convolution}

  Suppose $Y[k]=H[k]X[k]$, then
  \begin{align*}
    y[n] &= \frac{1}{N}\sum_{k=0}^{N-1} H[k]X[k]e^{j\frac{2\pi kn}{N}}\\
    &= \frac{1}{N}\sum_{k=0}^{N-1} H[k]\left(\sum_{m=0}^{N-1} x[m]e^{-j\frac{2\pi km}{N}}\right)
    e^{j\frac{2\pi kn}{N}}\\
    &= \sum_{m=0}^{N-1}x[m]\left(\frac{1}{N}\sum_{k=0}^{N-1} H[k]e^{-j\frac{2\pi k(n-m)}{N}}\right)\\
    &= \sum_{m=0}^{N-1} x[m] h\left[(\!(n-m)\!)_N\right]
  \end{align*}
  The last line is because $\frac{2\pi k(n-m)}{N}=\frac{2\pi k(\!(n-m)\!)_N}{N}$.
\end{frame}

\begin{frame}
  \frametitle{Circular convolution}

  Multiplying the DFT means {\bf circular convolution} of the time-domain signals:
  \begin{displaymath}
    y[n]=h[n]\circledast x[n] \leftrightarrow Y[k] = H[k]X[k],
  \end{displaymath}
  
  Circular convolution ($h[n]\circledast x[n]$) is defined like this:
  \begin{displaymath}
    h[n]\circledast x[n] = \sum_{m=0}^{N-1}x[m]h\left[(\!(n-m)\!)_N\right]
    = \sum_{m=0}^{N-1}h[m]x\left[(\!(n-m)\!)_N\right]
  \end{displaymath}
\end{frame}

\begin{frame}
  \frametitle{Circular convolution example}

  \centerline{\animategraphics[loop,controls,width=\textwidth]{10}{exp/circularsquares}{0}{49}}
\end{frame}

\begin{frame}
  \frametitle{Circular convolution example}

  \centerline{\animategraphics[loop,controls,width=\textwidth]{10}{exp/circularexps}{0}{49}}
\end{frame}


%%%%%%%%%%%%%%%%%%%%%%%%%%%%%%%%%%%%%%%%%%%%
\section[Zero-Padding]{Zero-Padding}
\setcounter{subsection}{1}

\begin{frame}
  \frametitle{How long is $h[n]\ast x[n]$?}

  If $x[n]$ is $M$ samples long, and $h[n]$ is $L$ samples long, then
  their linear convolution, $y[n] = x[n]\ast h[n]$, is $M+L-1$ samples long.
  
  \centerline{\includegraphics[width=\textwidth]{exp/convolutionlengths.png}}
\end{frame}

\begin{frame}
  \frametitle{Zero-padding turns circular convolution into linear convolution}

  How it works:
  \begin{itemize}
  \item $h[n]$ is length-$L$
  \item $x[n]$ is length-$M$
  \item As long as they are both zero-padded to length $N\ge L+M-1$, then
  \item $y[n] = h[n]\circledast x[n]$ is the same as $h[n]\ast x[n]$.
  \end{itemize}
\end{frame}
  
\begin{frame}
  \frametitle{Zero-padding turns circular convolution into linear convolution}

  Why it works:  Either\ldots
  \begin{itemize}
  \item $n-m$ is a positive number, between $0$ and $N-1$.  Then
    $(\!(n-m)\!)_N=n-m$, and therefore
    \begin{displaymath}
      x[m]h\left[(\!(n-m)\!)_N\right]=x[m]h[n-m]
    \end{displaymath}
  \item $n-m$ is a negative number, between $0$ and $-(L-1)$.  Then
    $(\!(n-m)\!)_N=N+n-m\ge N-(L-1)>M-1$, so
    \begin{displaymath}
      x[m]h\left[(\!(n-m)\!)_N\right]=0
    \end{displaymath}
  \end{itemize}
\end{frame}
  
\begin{frame}
  \frametitle{Case \#1: $n-m$ is positive, so circular
    convolution is the same as linear convolution}

  \centerline{\includegraphics[width=\textwidth]{exp/circandlinear50.png}}
\end{frame}
  
\begin{frame}
  \frametitle{Case \#2: $n-m$ is negative, so it wraps around, but $N$
    is long enough so that the wrapped part of
    $h\left[(\!(n-m)\!)_N\right]$ doesn't overlap with $x[m]$}
  
  \centerline{\includegraphics[width=\textwidth]{exp/circandlinear0.png}}
\end{frame}
  
\begin{frame}
  \frametitle{Zero-padding turns circular convolution into linear convolution}

  \centerline{\animategraphics[loop,controls,width=\textwidth]{10}{exp/circandlinear}{0}{68}}
\end{frame}



%%%%%%%%%%%%%%%%%%%%%%%%%%%%%%%%%%%%%%%%%%%%
\section[Summary]{Summary}
\setcounter{subsection}{1}

\begin{frame}
  \frametitle{Summary: Two different ways to think about the DFT}
  \begin{block}{1. $x[n]$ is finite length; DFT is samples of DTFT}
    \begin{displaymath}
      x[n]=0,n<0~\text{or}~n\ge N~~~\leftrightarrow~~~X[k]=\left.X(\omega)\right|_{\omega=\frac{2\pi k}{N}}
    \end{displaymath}
  \end{block}
  \begin{block}{2. $x[n]$ is periodic; DFT is scaled version of Fourier series}
    \begin{displaymath}
      x[n]=x[n+N]~~~\leftrightarrow~~~X[k]=N X_k
    \end{displaymath}
  \end{block}
\end{frame}

\begin{frame}
  \frametitle{Circular convolution}

  Multiplying the DFT means {\bf circular convolution} of the time-domain signals:
  \begin{displaymath}
    y[n]=h[n]\circledast x[n] \leftrightarrow Y[k] = H[k]X[k],
  \end{displaymath}
  
  Circular convolution ($h[n]\circledast x[n]$) is defined like this:
  \begin{displaymath}
    h[n]\circledast x[n] = \sum_{m=0}^{N-1}x[m]h\left[(\!(n-m)\!)_N\right]
    = \sum_{m=0}^{N-1}h[m]x\left[(\!(n-m)\!)_N\right]
  \end{displaymath}

  Circular convolution is the same as linear convolution if and only if $N\ge L+M-1$.
\end{frame}


\end{document}
