\documentclass{beamer}
\usepackage{tikz,amsmath,hyperref,graphicx,stackrel,animate}
\usetikzlibrary{positioning,shadows,arrows,shapes,calc,dsp,chains}
\newcommand{\argmax}{\operatornamewithlimits{argmax}}
\newcommand{\argmin}{\operatornamewithlimits{argmin}}
\mode<presentation>{\usetheme{Frankfurt}}
\AtBeginSection[]
{
  \begin{frame}<beamer>
    \frametitle{Outline}
    \tableofcontents[currentsection,currentsubsection]
  \end{frame}
}
\title{Lecture 12: Impulse Response}
\author{Mark Hasegawa-Johnson\\These slides are in the public domain.}
\date{ECE 401: Signal and Image Analysis, Fall 2023}  
\begin{document}

% Title
\begin{frame}
  \maketitle
\end{frame}

% Title
\begin{frame}
  \tableofcontents
\end{frame}

%%%%%%%%%%%%%%%%%%%%%%%%%%%%%%%%%%%%%%%%%%%%
\section[Review]{Review: Linearity and Shift Invariance}
\setcounter{subsection}{1}

\begin{frame}
  \frametitle{What is a System?}

  A {\bf system} is anything that takes one signal as input, and
  generates another signal as output.   We can write
  \[
  x[n] \stackrel{\mathcal H}{\longrightarrow} y[n]
  \]
  which means
  \centerline{
    \begin{tikzpicture}
      \node[dspnodeopen,dsp/label=right] (y) at (5,0) {$y[n]$};
      \node[dspsquare] (f) at (2.5,0) {${\mathcal H}$} edge[dspconn](y);
      \node[dspnodeopen,dsp/label=left] (x) at (0,0) {$x[n]$} edge[dspconn](f);
    \end{tikzpicture}}
\end{frame}

\begin{frame}
  \frametitle{Linearity and Shift Invariance}
  \begin{itemize}
  \item A system is {\bf linear} if and only if, for any two inputs
    $x_1[n]$ and $x_2[n]$ that produce outputs $y_1[n]$ and $y_2[n]$,
    \[
    x[n]=x_1[n]+x_2[n] \stackrel{\mathcal H}{\longrightarrow}  y[n]=y_1[n]+y_2[n]
    \]
  \item A system is {\bf shift-invariant} if and only if, for any input
    $x_1[n]$ that produces output $y_1[n]$,
    \[
    x[n]=x_1[n-n_0] \stackrel{\mathcal H}{\longrightarrow}  y[n]=y_1[n-n_0]
    \]
  \end{itemize}
\end{frame}


%%%%%%%%%%%%%%%%%%%%%%%%%%%%%%%%%%%%%%%%%%%%
\section[Convolution]{Convolution}
\setcounter{subsection}{1}

\begin{frame}
  \frametitle{LSI Systems and Convolution}

  We care about linearity and shift-invariance because of the
  following remarkable result:
  \begin{block}{LSI Systems and Convolution}
    Let ${\mathcal H}$ be any system,
    \[
    x[n]\stackrel{H}{\longrightarrow} y[n]
    \]
    If ${\mathcal H}$ is linear and shift-invariant, then whatever
    processes it performs can be equivalently replaced by a convolution:
    \[
    y[n] = \sum_{m=-\infty}^\infty h[m] x[n-m]
    \]
  \end{block}
\end{frame}

\begin{frame}
  \frametitle{Impulse Response}

  \[
  y[n] = \sum_{m=-\infty}^\infty h[m] x[n-m]
  \]
  The weights $h[m]$ are called the ``impulse response'' of the system.  We can
  measure them, in the real world, by putting the following signal into the system:
  \[
  \delta[n] = \begin{cases}
    1 & n=0\\
    0 & \mbox{otherwise}
  \end{cases}
  \]
  and measuring the response:
  \[
  \delta[n] \stackrel{H}{\longrightarrow} h[n]
  \]
\end{frame}



\begin{frame}
  \frametitle{Convolution: Proof}
  \begin{enumerate}
  \item $h[n]$ is the impulse response.
    \[
    \delta[n] \stackrel{H}{\longrightarrow} h[n]
    \]
  \item The system is {\bf shift-invariant}, therefore
    \[
    \delta[n-m] \stackrel{H}{\longrightarrow} h[n-m]
    \]
  \item The system is {\bf linear}, therefore {\bf scaling the input
    by a constant} results in {\bf scaling the output by the same
    constant}:
    \[
    x[m]\delta[n-m] \stackrel{H}{\longrightarrow} x[m]h[n-m]
    \]
  \item The system is {\bf linear}, therefore {\bf adding input
    signals} results in {\bf adding the output signals}:
    \[
    \sum_{m=-\infty}^\infty x[m]\delta[n-m] \stackrel{H}{\longrightarrow} \sum_{m=-\infty}^\infty x[m]h[n-m]
    \]
  \end{enumerate}
\end{frame}

\begin{frame}
  \frametitle{Convolution: Proof (in Words)}

  \begin{itemize}
  \item The input signal, $x[n]$, is just a bunch of samples.
  \item Each one of those samples is a scaled impulse, so each one of
    them produces a scaled impulse response at the output.
  \item Convolution = add together those scaled impulse responses.
  \end{itemize}
\end{frame}


\begin{frame}
  \frametitle{Convolution: Proof (in Pictures)}

  \centerline{\includegraphics[width=\textwidth]{exp/convolutionproof.png}}
\end{frame}

%%%%%%%%%%%%%%%%%%%%%%%%%%%%%%%%%%%%%%%%%%%%
\section[Example]{Written Example}
\setcounter{subsection}{1}

\begin{frame}
  \frametitle{Written Example}

  Consider a system that computes the summation of all of its inputs:
  \[
  y[n] = \sum_{m=-\infty}^n x[m]
  \]
  What is the impulse response of this system?  Show that this system can be implemented using
  $y[n]=h[n]\ast x[n]$ for an appropriate $h[n]$.
\end{frame}

%%%%%%%%%%%%%%%%%%%%%%%%%%%%%%%%%%%%%%%%%%%%
\section[Summary]{Summary}
\setcounter{subsection}{1}

\begin{frame}
  \frametitle{Summary}
  \begin{itemize}
  \item A system is {\bf linear} if and only if, for any two inputs
    $x_1[n]$ and $x_2[n]$ that produce outputs $y_1[n]$ and $y_2[n]$,
    \[
    x[n]=x_1[n]+x_2[n] \stackrel{\mathcal H}{\longrightarrow}  y[n]=y_1[n]+y_2[n]
    \]
  \item A system is {\bf shift-invariant} if and only if, for any input
    $x_1[n]$ that produces output $y_1[n]$,
    \[
    x[n]=x_1[n-n_0] \stackrel{\mathcal H}{\longrightarrow}  y[n]=y_1[n-n_0]
    \]
  \item If a system is {\bf linear and shift-invariant} (LSI), then it
    can be implemented using convolution:
    \[
    y[n] = h[n]\ast x[n]
    \]
    where $h[n]$ is the impulse response:
    \[
    \delta[n] \stackrel{\mathcal H}{\longrightarrow}  h[n]
    \]
  \end{itemize}
\end{frame}
    

\end{document}
