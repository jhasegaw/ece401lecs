\documentclass{beamer}
\usepackage{tikz,amsmath,hyperref,graphicx,stackrel,animate,media9}
\usetikzlibrary{positioning,shadows,arrows,shapes,calc}
\newcommand{\argmax}{\operatornamewithlimits{argmax}}
\newcommand{\argmin}{\operatornamewithlimits{argmin}}
\mode<presentation>{\usetheme{Frankfurt}}
\AtBeginSection[]
{
  \begin{frame}<beamer>
    \frametitle{Outline}
    \tableofcontents[currentsection,currentsubsection]
  \end{frame}
}
\title{Lecture 15: Causality and Stability}
\author{Mark Hasegawa-Johnson\\These slides are in the public domain}
\date{ECE 401: Signal and Image Analysis, Fall 2023}  
\begin{document}

% Title
\begin{frame}
  \maketitle
\end{frame}

% Title
\begin{frame}
  \tableofcontents
\end{frame}

%%%%%%%%%%%%%%%%%%%%%%%%%%%%%%%%%%%%%%%%%%%%
\section[Review]{Review: Impulse Response and Frequency Response}
\setcounter{subsection}{1}

\begin{frame}
  \frametitle{Impulse Response and Convolution}

  The impulse response of a system is its response to an impulse:
  \begin{displaymath}
    \delta[n] \stackrel{\mathcal H}{\longrightarrow} h[n]
  \end{displaymath}
  If a system is linear and shift-invariant, then its output, in
  response to {\bf any} input, can be commputed  using convolution:
  \begin{displaymath}
    x[n] \stackrel{\mathcal H}{\longrightarrow} y[n]=h[n]\ast x[n]
  \end{displaymath}
\end{frame}

\begin{frame}
  \frametitle{Frequency Response}

  The frequency response of a system is its response to a pure tone:
  \begin{align*}
    x[n]=e^{j\omega n} &\rightarrow y[n]=H(\omega)e^{j\omega n}\\
    x[n]=\cos\left(\omega n\right)
    &\rightarrow y[n]=|H(\omega)|\cos\left(\omega n+\angle H(\omega)\right)\\
  \end{align*}
  The frequency response is related to the impulse response by:
  \[
  H(\omega) = \sum_m h[m]e^{-j\omega m}
  \]
\end{frame}  
        

%%%%%%%%%%%%%%%%%%%%%%%%%%%%%%%%%%%%%%%%%%%%
\section[Causality]{Causality = The future is unknown}
\setcounter{subsection}{1}

\begin{frame}
  \frametitle{Causality}

  {\bf Definition:} A {\bf causal} system is a system whose output at
  time $n$, $y[n]$, depends on inputs $x[m]$ only for $m\le n$.
\end{frame}
  
\begin{frame}
  \frametitle{What systems are causal?  What systems are non-causal?}

  \begin{itemize}
  \item A real-time system must be causal.
  \item If $n$ is time, but the system is operating in batch mode,
    then it doesn't need to be causal.
  \item If $n$ is space (e.g., rows or columns of an image), the
    system doesn't need to be causal.
  \end{itemize}
\end{frame}

\begin{frame}
  \frametitle{Causal system $\Leftrightarrow$ Right-sided impulse response}

  \begin{displaymath}
    y[n] = \sum_m h[m]x[n-m]
  \end{displaymath}
  \begin{itemize}
    \item This system is {\bf causal} iff $y[n]$ depends on $x[n-m]$
      only for $n-m\le n$.
    \item In other words, the system is causal iff $h[n]=0$ for all $n<0$.
  \end{itemize}
\end{frame}

\begin{frame}
  \frametitle{Causal system $\Leftrightarrow$ Right-sided impulse response}

  \begin{center}
    \includegraphics[width=0.8\textwidth]{exp/664px-Illustration_of_causal_and_non-causal_filters.svg.png}

    {\tiny Public domain image, \url{https://commons.wikimedia.org/wiki/File:Illustration_of_causal_and_non-causal_filters.svg}}
  \end{center}
\end{frame}

\begin{frame}
  \frametitle{Variations on the word ``causal''}

  \begin{itemize}
  \item A {\bf causal} system is one that depends only on the present
    and the past, thus $h[n]=0$ for $n<0$.
  \item A {\bf non-causal} system is one that's not causal.
  \item A {\bf anti-causal} system is one that depends only on the
    present and the future, thus $h[n]=0$ for $n>0$.
  \end{itemize}
\end{frame}

\begin{frame}
  \frametitle{Causality $\Leftrightarrow$ Non-Positive Phase Response}

  If you put a cosine into a system, you get a cosine advanced by
  $\angle H(\omega)$:
  \begin{displaymath}
    \cos(\omega n) \stackrel{\mathcal H}{\longrightarrow} |H(\omega)|
    \cos\left(\omega n+\angle H(\omega)\right)
  \end{displaymath}
  \begin{itemize}
  \item If $\angle H(\omega)>0$, it means that the output is happening {\bf before} the input!
  \item It turns out that for any causal system, $\angle H(\omega)\le 0$
  \end{itemize}
\end{frame}

\begin{frame}
  \frametitle{Causality $\Leftrightarrow$ Non-Positive Phase Response}

  Remember how we can calculate $H(\omega)$:
  \begin{displaymath}
    H(\omega) = \sum_{m=-\infty}^\infty h[m] e^{-j\omega m}
  \end{displaymath}
  \begin{itemize}
  \item If the system is causal, then the only nonzero terms in that sum are terms with
    non-positive phase ($e^{-j\omega m}$)
  \item Therfore causal systems have $\angle H(\omega)\le 0$
  \end{itemize}
\end{frame}

%%%%%%%%%%%%%%%%%%%%%%%%%%%%%%%%%%%%%%%%%%%%
\section[Stability]{Stability = All finite inputs produce finite outputs}
\setcounter{subsection}{1}

\begin{frame}
  \frametitle{Stability}

  {\bf Definition:} A system is {\bf stable} if and only if {\bf every}
  bounded $x[n]$ (every signal such that $|x[n]|<\infty$ for all $n$) produces a bounded output
  ($|y[n]|<\infty$ for all $n$).
\end{frame}

\begin{frame}
  \frametitle{Why Stability Matters}

  \begin{itemize}
  \item If your system is stable, then every now and then, you'll
    get an inexplicable bug:
  \item all of the samples of $y[n]$ will be FLT\_MAX!
  \item That's very hard to debug.  If you view it as an image, or
    listen to it, it will sound like you just didn't generate the
    samples, so you will be looking for the error in the wrong place!
  \end{itemize}
    
\end{frame}

\begin{frame}
  \frametitle{Magnitude-summable impulse response $\Rightarrow$ Stable system}

  \begin{displaymath}
    y[n] = \sum_{m=-\infty}^\infty h[m] x[n-m]
  \end{displaymath}
  Suppose we know that $|x[n]|<M$, for some finite $M$, for all $n$.  Then
  \begin{displaymath}
    |y[n]| \le  M \sum_{m=-\infty}^\infty |h[m]|
  \end{displaymath}
  So
  \begin{displaymath}
    \sum_{m=-\infty}^\infty |h[m]|~\mbox{is finite}~\Rightarrow~\mbox{System is stable}
  \end{displaymath}
\end{frame}

\begin{frame}
  \frametitle{Stable system $\Rightarrow$ Magnitude-summable impulse response}

  On the other hand, suppose that 
  \begin{displaymath}
    \sum_{m=-\infty}^\infty |h[m]|=\infty
  \end{displaymath}
  Does that mean that the system is {\bf unstable}?  Yes!  Yes, it
  does!  Consider the ``worst-case'' input
  \begin{displaymath}
    x[n] = \mbox{sign}\left(h[-n]\right)
  \end{displaymath}
  Then $y[0]$ is
  \begin{displaymath}
    y[0] = \sum_{m=-\infty}^\infty h[m] x[-m] = \sum_{m=-\infty}^\infty |h[m]|=\infty
  \end{displaymath}
\end{frame}

\begin{frame}
  \frametitle{Example: Weighted Average}

  For example, consider a 7-tap weighted average:
  \begin{displaymath}
    y[n] = \sum_{m=-3}^3 h[m] x[n-m]
  \end{displaymath}
  As long as all of the weights are finite ($|h[m]|<\infty$ for all
  $m$), then $\sum_{m=-3}^3 |h[m]|$ is also finite, so the system is
  stable
\end{frame}

\begin{frame}
  \frametitle{Example: Weighted Average}

  For any finite input, the output is finite:
  \centerline{\animategraphics[loop,controls,height=0.8\textheight]{10}{exp/weightedaverage}{0}{49}}
\end{frame}

\begin{frame}
  \frametitle{Example: Summation}

  For example, consider summation:
  \begin{displaymath}
    y[n] = \sum_{m=0}^\infty x[n-m]
  \end{displaymath}
  This is an {\bf unstable} system!!
  \begin{align*}
    h[n] &= \left\{\begin{array}{ll}1 & n\ge 0\\0 & n<0\end{array}\right.\\
    \sum_{n=-\infty}^\infty |h[n]] &=\sum_{n=0}^\infty 1 = \infty
  \end{align*}
\end{frame}

\begin{frame}
  \frametitle{Example: Summation}

  For example, if the input is a unit step, the output is unbounded:
  \centerline{\animategraphics[loop,controls,height=0.8\textheight]{10}{exp/stepfunction}{0}{49}}
\end{frame}

\begin{frame}
  \frametitle{Example: Obviously Unstable System}

  Finally, some systems are just obviously unstable.  Consider
  \begin{displaymath}
    h[n] = (1.1)^n u[n]
  \end{displaymath}
  This is obviously unstable.  In fact, not only does $|h[n]|$ sum to
  infinity --- it even goes to infinity if the input is just a delta function!
\end{frame}

\begin{frame}
  \frametitle{Example: Obviously Unstable System}

  For example, if the input is a unit step, the output is unbounded:
  \centerline{\animategraphics[loop,controls,height=0.8\textheight]{10}{exp/exponentialstepresponse}{0}{49}}
\end{frame}

\begin{frame}
  \frametitle{Example: Obviously Unstable System}

  Even if the input is a delta function, the output is unbounded:
  \centerline{\animategraphics[loop,controls,height=0.8\textheight]{10}{exp/exponentialresponse}{0}{49}}
\end{frame}

\begin{frame}
  \frametitle{Relationship to Frequency Response}

  How about the frequency response of stable versus unstable systems?
  Guess what:
  \begin{itemize}
  \item A stable system has a finite magnitude response.
  \item An unstable system usually has an infinite-magnitude (undefined) frequency response.
  \end{itemize}

  Proof:
  \begin{align*}
    |H(\omega)| &= \left|\sum_{m=-\infty}^\infty h[n]e^{-j\omega n}\right|\\
    &\le \sum_{m=-\infty}^\infty \left|h[n]\right|
  \end{align*}
\end{frame}

%%%%%%%%%%%%%%%%%%%%%%%%%%%%%%%%%%%%%%%%%%%%
\section[Summary]{Summary}
\setcounter{subsection}{1}

\begin{frame}
  \frametitle{Summary}

  \begin{itemize}
  \item A system is causal if and only if $h[n]$ is right-sided.
    \begin{itemize}
    \item A causal system has a negative phase response.
    \end{itemize}
  \item A system is stable if and only if $h[n]$ is
    magnitude-summable.
    \begin{itemize}
    \item A stable system has a finite magnitude response.
    \end{itemize}
  \end{itemize}
\end{frame}


\end{document}
