\documentclass{beamer}
\usepackage{tikz,amsmath,hyperref,graphicx,stackrel,animate}
\usetikzlibrary{positioning,shadows,arrows,shapes,calc}
\newcommand{\argmax}{\operatornamewithlimits{argmax}}
\newcommand{\argmin}{\operatornamewithlimits{argmin}}
\mode<presentation>{\usetheme{Frankfurt}}
\AtBeginSection[]
{
  \begin{frame}<beamer>
    \frametitle{Outline}
    \tableofcontents[currentsection,currentsubsection]
  \end{frame}
}
\title{Lecture 21: Frequency Domain Convolution Examples}
\author{Mark Hasegawa-Johnson}
\date{ECE 401: Signal and Image Analysis, Fall 2022}  
\begin{document}

% Title
\begin{frame}
  \maketitle
\end{frame}

% Title
\begin{frame}
  \tableofcontents
\end{frame}

%%%%%%%%%%%%%%%%%%%%%%%%%%%%%%%%%%%%%%%%%%%%
\section[Examples]{Frequency Domain Convolution Examples}
\setcounter{subsection}{1}

\begin{frame}
  \frametitle{Frequency Domain Convolution}

  Remember that the Fourier transform of windowing is convolution in
  frequency:
  \begin{displaymath}
    h[n] = w[n]h_i[n] \leftrightarrow H(\omega) = \frac{1}{2\pi} H_i(\omega)\ast W(\omega),
  \end{displaymath}
  where
  \begin{displaymath}
    H_i(\omega)\ast W(\omega) = \int_{-\pi}^\pi H_i(\theta)W(\omega-\theta)d\theta
  \end{displaymath}
\end{frame}

\begin{frame}
  \frametitle{Ideal LPF Convolved in Frequency with the DTFT of a Rectangular Window}
  \centerline{\animategraphics[loop,controls,width=0.8\textwidth]{15}{exp/ideal_with_rect}{0}{126}}
\end{frame}

\begin{frame}
  \frametitle{Ideal LPF Convolved in Frequency with the DTFT of a Hamming Window}
  \centerline{\animategraphics[loop,controls,width=0.8\textwidth]{15}{exp/ideal_with_hamming}{0}{126}}
\end{frame}

%%%%%%%%%%%%%%%%%%%%%%%%%%%%%%%%%%%%%%%%%%%%
\section[Summary]{Summary}
\setcounter{subsection}{1}

\begin{frame}
  \frametitle{Summary}
  \begin{itemize}
  \item The sidelobes of the Hamming window are tiny, therefore the
    stop-band ripple of the Hamming-windowed filter is tiny.
  \item The filter's transition band equals the main-lobe width of the
    window spectrum, which is $2\times \frac{2\pi}{N}=\frac{4\pi}{N}$
    for the rectangular window, $2\times \frac{4\pi}{N}=\frac{8\pi}{N}$ for the
    Hamming window.
  \end{itemize}
\end{frame}


\end{document}
