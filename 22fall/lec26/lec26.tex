\documentclass{beamer}
\usepackage{tikz,amsmath,hyperref,graphicx,stackrel,animate,amssymb}
\usetikzlibrary{positioning,shadows,arrows,shapes,calc}
\newcommand{\argmax}{\operatornamewithlimits{argmax}}
\newcommand{\argmin}{\operatornamewithlimits{argmin}}
\mode<presentation>{\usetheme{Frankfurt}}
\AtBeginSection[]
{
  \begin{frame}<beamer>
    \frametitle{Outline}
    \tableofcontents[currentsection,currentsubsection]
  \end{frame}
}
\title{Lecture 26: DTFT of a Sinusoid}
\author{Mark Hasegawa-Johnson\\All content~\href{https://creativecommons.org/licenses/by-sa/4.0/}{CC-SA 4.0} unless otherwise specified.}
\date{ECE 401: Signal and Image Analysis, Fall 2022}  
\begin{document}

% Title
\begin{frame}
  \maketitle
\end{frame}

% Title
\begin{frame}
  \tableofcontents
\end{frame}

%%%%%%%%%%%%%%%%%%%%%%%%%%%%%%%%%%%%%%%%%%%%
\section[Review]{Review: DFT, DTFT, and Fourier Series}
\setcounter{subsection}{1}

\begin{frame}
  \frametitle{Review: DFT, DTFT, and Fourier Series}

  Magnitude-summable signals have a DTFT:
  \begin{displaymath}
    X(\omega) = \sum_{n=-\infty}^\infty x[n]e^{-j\omega n},~~~\Leftrightarrow~~~
    x[n] = \frac{1}{2\pi}\int_{-\pi}^\pi X(\omega)e^{j\omega n}d\omega
  \end{displaymath}
  Periodic signals have a Fourier series:
  \begin{displaymath}
    X_k = \frac{1}{N}\sum_{n=1}^{N-1} x[n]e^{-j\frac{2\pi kn}{N}},~~~\Leftrightarrow~~~
    x[n] = \sum_{k=0}^{N-1} X_k e^{j\frac{2\pi kn}{N}}
  \end{displaymath}
  Finite-length or periodic signals have a DFT:
  \begin{displaymath}
    X[k] = \sum_{n=1}^{N-1} x[n]e^{-j\frac{2\pi kn}{N}},~~~\Leftrightarrow~~~
    x[n] = \frac{1}{N}\sum_{k=0}^{N-1} X[k] e^{j\frac{2\pi kn}{N}}
  \end{displaymath}
\end{frame}

\begin{frame}
  \frametitle{Review: DFT of a Sinusoid}

  To find the DFT of a sinusoid, we use the frequency-shift property
  of the DFT:
  \begin{align*}
    x[n]&=\cos(\omega_0 n)w[n] = 
    \left(\frac{1}{2}w[n]e^{j\omega_0 n}+\frac{1}{2}w[n]e^{-j\omega_0 n}\right)\\
    &\leftrightarrow\\
    X[k] &= \frac{1}{2}W\left(\frac{2\pi k}{N}-\omega_0\right) +
    \frac{1}{2}W\left(\frac{2\pi k}{N}+\omega_0\right)
  \end{align*}
  where $W(\omega)$ is the DTFT of the window.
\end{frame}

\begin{frame}
  \frametitle{Today's Questions}

  Today's questions are:
  \begin{enumerate}
  \item Can we use the frequency-shift property to find the DTFT of a
    windowed sinusoid?
  \item Can we use something like that to find the DTFT of a
    non-windowed, infinite length sinusoid?
  \end{enumerate}
\end{frame}

%%%%%%%%%%%%%%%%%%%%%%%%%%%%%%%%%%%%%%%%%%%%
\section[Windowed]{DTFT of a Windowed Sinusoid}
\setcounter{subsection}{1}

\begin{frame}
  \frametitle{DTFT of a Windowed Sinusoid}

  First, let's find the DTFT of a windowed sinusoid.  This is easy;
  it's the same as the DFT.  Since 
  \begin{align*}
    x[n]&=\cos(\omega_0 n)w[n] = 
    \left(\frac{1}{2}w[n]e^{j\omega_0 n}+\frac{1}{2}w[n]e^{-j\omega_0 n}\right)
  \end{align*}
  We can just use the frequency-shift property of the DTFT to get
  \begin{align*}
    X(\omega) &= \frac{1}{2}W\left(\omega-\omega_0\right) +
    \frac{1}{2}W\left(\omega+\omega_0\right)
  \end{align*}
\end{frame}

\begin{frame}
  \frametitle{DFT of a Cosine}

  Here are the DTFT and DFT of
  \[
  x[n] = \cos\left(\frac{2\pi 20.3}{N}n\right) w[n]
  \]
  
  \centerline{\includegraphics[width=\textwidth]{exp/dft_of_cosine1.png}}
\end{frame}

\begin{frame}
  \frametitle{DFT of a Cosine}

  Here are the DTFT and DFT of a cosine at a frequency that's a multiple of $2\pi k/N$.
  \centerline{\includegraphics[width=\textwidth]{exp/dft_of_cosine2.png}}
\end{frame}

%%%%%%%%%%%%%%%%%%%%%%%%%%%%%%%%%%%%%%%%%%%%
\section[Non-Windowed]{DTFT of a Non-Windowed Sinusoid}
\setcounter{subsection}{1}

\begin{frame}
  \frametitle{DTFT of a Non-Windowed Sinusoid}

  \begin{itemize}
  \item 
    How about $x[n]=\cos(\omega_0 n)$, with no windows?  Does it have a
    DTFT?
  \item
    It's not magnitude-summable!
    \begin{displaymath}
      \sum_{n=-\infty}^\infty |x[n]| = \infty
    \end{displaymath}
    Therefore, there's no guarantee that it has a valid DTFT.
  \item
    In fact, we will need to make up some new math in order to find
    the DTFT of this signal.
  \end{itemize}
\end{frame}

\begin{frame}
  \frametitle{The Dirac Delta Function}

  The Dirac delta function, $\delta(\omega)$, is defined as:
  \begin{itemize}
  \item $\delta(\omega)=0$ for all $\omega$ other than $\omega=0$.
  \item $\delta(0)=\infty$
  \item The integral of $\delta(\omega)$, from any negative $\omega$
    to any positive $\omega$, is exactly 1:
    \begin{displaymath}
      \int_{-\epsilon}^\epsilon \delta(\omega) d\omega = 1
    \end{displaymath}
  \end{itemize}
\end{frame}

\begin{frame}
  \begin{columns}
    \begin{column}{0.5\textwidth}
      It's useful to imagine the Dirac delta function as a tall, thin
      function --- a Gaussian, a rectangle, or whatever --- with zero
      width, infinite height, and an area of exactly 1.
    \end{column}
    \begin{column}{0.5\textwidth}
      \begin{center}
        \animategraphics[loop,controls,width=0.9\textwidth]{5}{exp/Dirac_function_approximation-}{0}{9}
        
        {\tiny CC-SA 4.0, \url{https://commons.wikimedia.org/wiki/File:Dirac_function_approximation.gif}}
      \end{center}
    \end{column}
  \end{columns}
\end{frame}

\begin{frame}
  \begin{columns}
    \begin{column}{0.5\textwidth}
      We usually draw it like this.  The arrow has zero width,
      infinite height, and an area of exactly 1.0.
    \end{column}
    \begin{column}{0.5\textwidth}
      \begin{center}
        \includegraphics[width=\textwidth]{exp/Dirac_distribution.png}
        
        {\tiny CC-SA 2.0, \url{https://commons.wikimedia.org/wiki/File:Dirac_distribution_PDF.svg}}
      \end{center}
    \end{column}
  \end{columns}
\end{frame}

\begin{frame}
  \frametitle{Integrating a Dirac Delta}

  The key use of a Dirac delta is that, when we multiply it by any
  function and integrate,
  \begin{itemize}
  \item All the values of that function at $\omega\ne 0$ are multiplied by $\delta(\omega)=0$
  \item The value at $\omega=0$ is multiplied by $+\infty$, in such a
    way that the integral is exactly:
    \begin{displaymath}
      \int_{-\pi}^\pi f(\omega)\delta(\omega)d\omega = f(0)
    \end{displaymath}
  \end{itemize}
\end{frame}

\begin{frame}
  \frametitle{Integrating a Shifted Dirac Delta}

  The delta function can also be shifted, to some frequency
  $\omega_0$.  This is written as
  $\delta(\omega-\omega_0)$.
  \begin{itemize}
  \item All the values of that function at $\omega\ne \omega_0$ are
    multiplied by $\delta(\omega-\omega_0)=0$
  \item The value at $\omega=\omega_0$ is multiplied by $+\infty$, in such a
    way that the integral is exactly:
    \begin{displaymath}
      \int_{-\pi}^\pi f(\omega)\delta(\omega-\omega_0)d\omega = f(\omega_0)
    \end{displaymath}
  \end{itemize}
\end{frame}

\begin{frame}
  \frametitle{Inverse DTFT of a Shifted Dirac Delta}

  Thus, for example,

  \begin{displaymath}
    \frac{1}{2\pi}\int_{-\pi}^\pi \delta(\omega-\omega_0) e^{j\omega n}d\omega =
    \frac{1}{2\pi}e^{j\omega_0 n}
  \end{displaymath}

  In other words, the inverse DTFT of $Y(\omega)=\delta(\omega-\omega_0)$ is
  $y[n]= \frac{1}{2\pi}e^{j\omega_0 n}$, a complex exponential.
\end{frame}

\begin{frame}
  \frametitle{DTFT Pairs}

  By the linearity of the DTFT, we therefore have the following useful
  DTFT pairs:

  \begin{displaymath}
    e^{j\omega_0 n} ~~~\leftrightarrow ~~~ 2\pi\delta(\omega-\omega_0),
  \end{displaymath}
  and
  \begin{displaymath}
    \cos(\omega_0 n) ~~~\leftrightarrow ~~~ \pi\delta(\omega-\omega_0)+\pi\delta(\omega+\omega_0)
  \end{displaymath}
\end{frame}

\begin{frame}
  \frametitle{Why This Answer Makes Sense}

  Suppose we were to try to find the DTFT of $x[n]=e^{j\omega_0 n}$ directly:
  \begin{displaymath}
    X(\omega) = \sum_{n=-\infty}^\infty x[n]e^{-j\omega n} = \sum_{n=-\infty}^\infty e^{j(\omega-\omega_0)n}
  \end{displaymath}
  \begin{itemize}
  \item At frequencies $\omega\ne\omega_0$, we would be adding the
    samples of a sinusoid, which would give us $X(\omega)=0$.
  \item At $\omega=\omega_0$, the summation becomes
    \begin{displaymath}
      X(\omega_0) = \sum_{n=-\infty}^\infty 1 = \infty
    \end{displaymath}
  \item So $X(\omega_0)=\infty$, and $X(\omega)=0$ everywhere else.
    So it's a Dirac delta!  The only thing the forward transform {\bf
      doesn't} tell us is: {\bf what kind of infinity?}
  \end{itemize}
\end{frame}

\begin{frame}
  \frametitle{Why This Answer Makes Sense}
  \begin{itemize}
  \item So $X(\omega_0)=\infty$, and $X(\omega)=0$ everywhere else.
    So it's a Dirac delta!  The only thing the forward transform {\bf
      doesn't} tell us is: {\bf what kind of infinity?}
  \item The inverse DTFT gives us the answer.  It needs to be the kind
    of infinity such that
    \begin{displaymath}
      \frac{1}{2\pi}\int_{-\pi}^\pi X(\omega) e^{j\omega n}d\omega = e^{j\omega_0 n},
    \end{displaymath}
    and the solution is $X(\omega)=2\pi\delta(\omega-\omega_0)$
  \end{itemize}
\end{frame}

%%%%%%%%%%%%%%%%%%%%%%%%%%%%%%%%%%%%%%%%%%%%
\section[Windowing]{Windowing in Time = Convolution in Frequency}
\setcounter{subsection}{1}

\begin{frame}
  \frametitle{Windowing in Time = Convolution in Frequency}

  Remember that windowing in time = convolution in frequency:

  \begin{displaymath}
    y[n]=x[n]w[n] ~~~\leftrightarrow~~~Y(\omega) = \frac{1}{2\pi} X(\omega)\ast W(\omega).
  \end{displaymath}

  But if $x[n]=\cos(\omega_0 n)$, we already know that
  \begin{displaymath}
    y[n]=\cos(\omega_0 n)w[n] ~~~\leftrightarrow~~~ Y(\omega)=
    \frac{1}{2}W(\omega-\omega_0) + \frac{1}{2}W(\omega+\omega_0)
  \end{displaymath}
  Can we reconcile these two facts?
\end{frame}

\begin{frame}
  \frametitle{Convolving with a Dirac delta function}

  The delta function is defined by this sampling property:
  \begin{displaymath}
    \int_{-\pi}^\pi \delta(\omega-\omega_0) f(\omega) d\omega = f(\omega_0)
  \end{displaymath}

  What does that mean about convolution?  Let's try it:
  \begin{align*}
    \delta(\omega-\omega_0)\ast W(\omega)
    &= \int_{-\pi}^\pi \delta(\theta-\omega_0)W(\omega-\theta)d\theta\\
    &= W(\omega-\omega_0
  \end{align*}
\end{frame}
    
\begin{frame}
  \frametitle{Convolving with a Dirac delta function}

  So we see that:
  \begin{align*}
    \delta(\omega-\omega_0)\ast W(\omega)
    &= W(\omega-\omega_0)
  \end{align*}
  This is just like the behavior of impulses in the time domain:
  \begin{center}
    \includegraphics[width=\textwidth]{exp/two_pulses_with_impulse_response.png}

    {\tiny Public domain, \url{https://commons.wikimedia.org/wiki/File:Convolution_of_two_pulses_with_impulse_response.svg}}
  \end{center}
\end{frame}

\begin{frame}
  \frametitle{DTFT of a Windowed Cosine}

  So if:
  \begin{displaymath}
    \cos(\omega_0 n)~~~\leftrightarrow~~~\pi\delta(\omega-\omega_0) + \pi\delta(\omega+\omega_0),
  \end{displaymath}
  and
  \begin{displaymath}
    y[n]=x[n]w[n] ~~~\leftrightarrow~~~Y(\omega) = \frac{1}{2\pi} X(\omega)\ast W(\omega),
  \end{displaymath}
  then
  \begin{align*}
    \cos(\omega_0 n)w[n] ~~~
    &\leftrightarrow~~~
    \left(\frac{1}{2}\delta(\omega-\omega_0)\ast W(\omega)+
    \frac{1}{2}\delta(\omega+\omega_0)\ast W(\omega)\right)\\
    &= \left(\frac{1}{2}W(\omega-\omega_0)+
    \frac{1}{2}W(\omega+\omega_0)\right)
  \end{align*}
\end{frame}  
  
\begin{frame}
  \frametitle{DFT of a Cosine}

  So again, we discover that:
  \[
  x[n] = \cos\left(\frac{2\pi 20.3}{N}n\right) w[n]
  \]
  
  \centerline{\includegraphics[width=\textwidth]{exp/dft_of_cosine1.png}}
\end{frame}

%%%%%%%%%%%%%%%%%%%%%%%%%%%%%%%%%%%%%%%%%%%%
\section[Summary]{Summary}
\setcounter{subsection}{1}

\begin{frame}
  \frametitle{Summary}

  \begin{itemize}
  \item DTFT of a complex exponential is a delta function:
    \begin{displaymath}
      e^{j\omega_0 n}~~\leftrightarrow~~~2\pi\delta(\omega-\omega_0)
    \end{displaymath}
  \item DTFT of a cosine is two delta functions:
    \begin{displaymath}
      \cos(\omega_0 n)~~\leftrightarrow~~~\pi\delta(\omega-\omega_0)+\pi\delta(\omega+\omega_0)
    \end{displaymath}
  \item DTFT of a windowed cosine is frequency-shifted window functions:
    \begin{displaymath}
      \cos(\omega_0 n)w[n]~~\leftrightarrow~~~\frac{1}{2}W(\omega-\omega_0)+\frac{1}{2}W(\omega+\omega_0)
    \end{displaymath}
  \end{itemize}
\end{frame}

%%%%%%%%%%%%%%%%%%%%%%%%%%%%%%%%%%%%%%%%%%%%
\section[Example]{Written Example}
\setcounter{subsection}{1}

\begin{frame}
  \frametitle{Written Example}

  Consider the function
  \[
  x[n] = A \cos(\omega_0 n+\theta)
  \]
  What is $X(\omega)$?

  How about $y[n]=w[n]x[n]$.  What is $Y(\omega)$?
\end{frame}


\end{document}
