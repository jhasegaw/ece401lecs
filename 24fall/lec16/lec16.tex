\documentclass{beamer}
\usepackage{tikz,amsmath,hyperref,graphicx,stackrel,animate,media9}
\usetikzlibrary{positioning,shadows,arrows,shapes,calc}
\newcommand{\argmax}{\operatornamewithlimits{argmax}}
\newcommand{\argmin}{\operatornamewithlimits{argmin}}
\mode<presentation>{\usetheme{Frankfurt}}
\AtBeginSection[]
{
  \begin{frame}<beamer>
    \frametitle{Outline}
    \tableofcontents[currentsection,currentsubsection]
  \end{frame}
}
\title{Lecture 16: Discrete-Time Fourier Transform (DTFT)}
\author{Mark Hasegawa-Johnson}
\date{ECE 401: Signal and Image Analysis}  
\begin{document}

% Title
\begin{frame}
  \maketitle
\end{frame}

% Title
\begin{frame}
  \tableofcontents
\end{frame}

%%%%%%%%%%%%%%%%%%%%%%%%%%%%%%%%%%%%%%%%%%%%
\section[Review]{Review: Frequency Response}
\setcounter{subsection}{1}

\begin{frame}
  \frametitle{Response of LSI System to Periodic Inputs}
  Suppose we compute $y[n]=x[n]\ast h[n]$, where
  \begin{align*}
  x[n] &= \frac{1}{N}\sum_{k=0}^{N-1} X[k] e^{j2\pi kn/N},~\mbox{and}\\
  y[n] &= \frac{1}{N}\sum_{k=0}^{N-1} Y[k] e^{j2\pi kn/N}.
  \end{align*}
  The relationship between $Y[k]$ and $X[k]$ is given by the frequency
  response:
  \[
  Y[k] = H(k\omega_0) X[k]
  \]
  where
  \[
  H(\omega) = \sum_{n=-\infty}^\infty h[n]e^{-j\omega n}
  \]
\end{frame}

\begin{frame}
  \frametitle{Response of LSI System to Aperiodic Inputs}

  But what about signals that never repeat themselves?

  Can we still write something like
  \[
  Y(\omega)=H(\omega)X(\omega)?
  \]

\end{frame}
  
%%%%%%%%%%%%%%%%%%%%%%%%%%%%%%%%%%%%%%%%%%%%
\section[DTFT]{Discrete Time Fourier Transform}
\setcounter{subsection}{1}

\begin{frame}
  \frametitle{Aperiodic}
  
  An ``aperiodic signal'' is a signal that is not periodic.
  \begin{itemize}
  \item Music: strings, woodwinds, and brass are periodic, drums and rain sticks are aperiodic.
  \item Speech: vowels and nasals are periodic, plosives and fricatives are aperiodic.
  \item Images: stripes are periodic, clouds are aperiodic.
  \item Bioelectricity: heartbeat is periodic, muscle contractions are aperiodic.
  \end{itemize}
\end{frame}

\begin{frame}
  \frametitle{Periodic}

  The spectrum of a periodic signal is given by its Fourier series.
  In discrete time, that's:
  \begin{align*}
    X_k &= \frac{1}{N_0}\sum_{n=-\frac{N_0}{2}}^{\frac{N_0-1}{2}} x[n] e^{-j\frac{2\pi kn}{N_0}}\\
    &= \frac{1}{N_0}\sum_{n=-\frac{N_0}{2}}^{\frac{N_0-1}{2}} x[n] e^{-j\omega n}
  \end{align*}
  and that gives the frequency content of the signal, at the frequency
  $\omega=\frac{2\pi k}{N_0}$.

  Here I'm using  $n\in\left\{-\frac{N_0}{2},\ldots,\frac{N_0-1}{2}\right\}$,
  but the sum could be over any sequence of $N_0$ continuous samples.
\end{frame}

\begin{frame}
  \frametitle{Aperiodic}

  An aperiodic signal is one that {\bf never} repeats itself.  So we
  want something like the limit, as $N_0\rightarrow\infty$, of the
  Fourier series.  Here is the simplest such thing that is useful:
  \begin{block}{Discrete-Time Fourier Transform (DTFT)}
  \[
  X(\omega) = \sum_{n=-\infty}^\infty x[n]e^{-j\omega n}
  \]
  \end{block}
\end{frame}
    
\begin{frame}
  \frametitle{Fourier Series vs. Fourier Transform}

  The Fourier Series coefficients are:
  \begin{align*}
    X_k &= \frac{1}{N_0}\sum_{n=-\frac{N_0}{2}}^{\frac{N_0-1}{2}} x[n] e^{-j\omega n}
  \end{align*}
  The Fourier transform is:
  \[
  X(\omega) = \sum_{n=-\infty}^\infty x[n]e^{-j\omega n}
  \]
  Notice that, besides taking the limit as $N_0\rightarrow\infty$, we
  also got rid of the $\frac{1}{N_0}$ factor.  So we can think of the
  DTFT as
  \[
  X(\omega) = \lim_{N_0\rightarrow\infty,\omega=\frac{2\pi k}{N_0}} N_0 X_k
  \]
  where the limit is: as $N_0\rightarrow\infty$, and
  $k\rightarrow\infty$, but $\omega=\frac{2\pi k}{N_0}$ remains
  constant.
\end{frame}

\begin{frame}
  \frametitle{Inverse DTFT}

  In order to convert $X(\omega)$ back to $x[n]$, we'll take advantage
  of orthogonality:
  \[
  \int_{-\pi}^\pi e^{j\omega(m-n)} d\omega =
  \begin{cases}
    2\pi & m=n\\
    0 & (m-n)=\mbox{any nonzero integer}
  \end{cases}
  \]
\end{frame}
  
\begin{frame}
  \frametitle{Inverse DTFT}

  Taking advantage of orthogonality, we can see that
  \begin{align*}
    \frac{1}{2\pi}\int_{-\pi}^\pi X(\omega)e^{j\omega m}d\omega\\
    &=\frac{1}{2\pi}\int_{-\pi}^{\pi}
    \left(\sum_{n=-\infty}^\infty x[n]e^{-j\omega n}\right)e^{j\omega m}d\omega\\
    &=\frac{1}{2\pi}\sum_{n=-\infty}^\infty x[n]\int_{-\pi}^\pi e^{j\omega(m-n)}d\omega\\
    = x[m]
  \end{align*}
\end{frame}

  
\begin{frame}
  \frametitle{Fourier Series and Fourier Transform}
  Discrete-Time Fourier Series (DTFS):
  \begin{align*}
    X_k &= \frac{1}{N_0}\sum_{n=0}^{N_0-1} x[n] e^{-j\frac{2\pi kn}{N_0}}\\
    x[n] &= \sum_{k=0}^{N_0-1} X_ke^{j\frac{2\pi kn}{N_0}}
  \end{align*}
  Discrete-Time Fourier Transform (DTFT):
  \begin{align*}
    X(\omega) &= \sum_{n=-\infty}^{\infty} x[n] e^{-j\omega n}\\
    x[n] &= \frac{1}{2\pi}\int_{-\pi}^\pi X(\omega)e^{j\omega n}d\omega
  \end{align*}
  
\end{frame}

\begin{frame}
  \frametitle{Quiz}

  Go to the course webpage, try the quiz!
\end{frame}

%%%%%%%%%%%%%%%%%%%%%%%%%%%%%%%%%%%%%%%%%%%%%%%%%%%%%%%%%%%%%%%%%%%%%%%%%%%%%%%%%%%%
\section[DTFT Properties]{Properties of the DTFT}
\setcounter{subsection}{1}

\begin{frame}
  \frametitle{Properties of the DTFT}

  In order to better understand the DTFT, let's discuss these properties:
  \begin{enumerate}
    \setcounter{enumi}{-1}
  \item Periodicity
  \item Linearity
  \item Time Shift
  \item Frequency Shift
  \item Filtering is Convolution
  \end{enumerate}
  Property \#4 is actually the reason why we invented the DTFT in the first place.
  Before we discuss it, though, let's talk about the others.
\end{frame}

\begin{frame}
  \frametitle{0. Periodicity}

  The DTFT is periodic with a  period of $2\pi$.  That's just because  $e^{j2\pi}=1$:
  \begin{align*}
    X(\omega) &= \sum_n x[n]e^{-j\omega n}\\
    X(\omega+2\pi) &= \sum_n x[n]e^{-j(\omega+2\pi) n} = \sum_n x[n]e^{-j\omega n} = X(\omega)\\
    X(\omega-2\pi) &= \sum_n x[n]e^{-j(\omega-2\pi) n} = \sum_n x[n]e^{-j\omega n} = X(\omega)
  \end{align*}
  For example, the inverse DTFT can be defined in two different ways:
  \[
  x[n]=\frac{1}{2\pi}\int_{-\pi}^\pi X(\omega)e^{j\omega n}d\omega =
  \frac{1}{2\pi}\int_{0}^{2\pi} X(\omega)e^{j\omega n}d\omega
  \]
  Those two integrals are equal because $X(\omega+2\pi)=X(\omega)$.
\end{frame}

\begin{frame}
  \frametitle{1. Linearity}

  The DTFT is linear:
  \[
  z[n] = ax[n]+by[n]~~~\leftrightarrow~~~
  Z(\omega)=aX(\omega)+bY(\omega)
  \]
  {\bf Proof:}
  \begin{align*}
    Z(\omega) &= \sum_n z[n]e^{-j\omega n}\\
    &= a\sum_n x[n]e^{-j\omega n} + b\sum_n y[n]e^{-j\omega n}\\
    &= aX(\omega) + bY(\omega)
  \end{align*}
\end{frame}

\begin{frame}
  \frametitle{2. Time Shift Property}

  Shifting in time is the same as multiplying by a  complex exponential in frequency:
  \[
  z[n] = x[n-n_0]~~~\leftrightarrow~~~
  Z(\omega)=e^{-j\omega n_0}X(\omega)
  \]
  {\bf Proof:}
  \begin{align*}
    Z(\omega) &= \sum_{n=-\infty}^{\infty} x[n-n_0]e^{-j\omega n}\\
    &= \sum_{m=-\infty}^{\infty} x[m]e^{-j\omega (m+n_0)}~~~\left(\mbox{where}~m=n-n_0\right)\\
    &= e^{-j\omega n_0} X(\omega)
  \end{align*}
\end{frame}

\begin{frame}
  \frametitle{3. Frequency Shift Property}

  Shifting in frequency is the same as multiplying by a complex exponential in time:
  \[
  z[n] = x[n]e^{j\omega_0 n}~~~\leftrightarrow~~~
  Z(\omega)=X(\omega-\omega_0)
  \]
  {\bf Proof:}
  \begin{align*}
    Z(\omega) &= \sum_{n=-\infty}^{\infty} x[n]e^{j\omega_0 n}e^{-j\omega n}\\
    &= \sum_{n=-\infty}^{\infty} x[n]e^{-j(\omega-\omega_0) n}\\
    &= X(\omega-\omega_0)
  \end{align*}
\end{frame}

\begin{frame}
  \frametitle{4. Convolution Property}

  Convolving in time is the same as multiplying in frequency:
  \[
  y[n]=h[n]\ast x[n]~~~\leftrightarrow
  Y(\omega)=H(\omega)X(\omega)
  \]
  {\bf Proof:}
  Remember that $y[n]=h[n]\ast x[n]$ means that $y[n]=\sum_{m=-\infty}^\infty h[m]x[n-m]$.  Therefore,
  \begin{align*}
    Y(\omega) &= \sum_{n=-\infty}^{\infty}\left(\sum_{m=-\infty}^\infty h[m]x[n-m]\right)e^{-j\omega n}\\
    &= \sum_{m=-\infty}^{\infty}\sum_{n=-\infty}^\infty\left(h[m]x[n-m]\right)e^{-j\omega m}e^{-j\omega (n-m)}\\
    &= \left(\sum_{m=-\infty}^{\infty}h[m]e^{-j\omega m}\right)
    \left(\sum_{(n-m)=-\infty}^\infty  x[n-m]e^{-j\omega (n-m)}\right)\\
    &= H(\omega)X(\omega)
  \end{align*}
\end{frame}

%%%%%%%%%%%%%%%%%%%%%%%%%%%%%%%%%%%%%%%%%%%%
\section[Examples]{Examples}
\setcounter{subsection}{1}

\begin{frame}
  \frametitle{Impulse and Delayed Impulse}

  For our  examples today, let's consider different combinations of these three signals:
  \begin{align*}
    f[n] &= \delta[n]\\
    g[n] &= \delta[n-3]\\
    h[n] &= \delta[n-6]
  \end{align*}
  Remember from last time what these mean:
  \begin{align*}
    f[n] &= \begin{cases}1&n=0\\0&\mbox{otherwise}\end{cases}\\
    g[n] &= \begin{cases}1&n=3\\0&\mbox{otherwise}\end{cases}\\
    h[n] &= \begin{cases}1&n=6\\0&\mbox{otherwise}\end{cases}\\
  \end{align*}
\end{frame}

\begin{frame}
  \frametitle{DTFT of an Impulse}

  First, let's find the DTFT of an impulse:
  \begin{align*}
    f[n] &= \begin{cases}1&n=0\\0&\mbox{otherwise}\end{cases}\\
    F(\omega) &= \sum_{n=-\infty}^\infty f[n]e^{-j\omega n}\\
    &= 1\times e^{-j\omega 0}\\
    &= 1
  \end{align*}
  So we get that $f[n]=\delta[n]\leftrightarrow F(\omega)=1$.  That
  seems like it might be important.
\end{frame}

\begin{frame}
  \frametitle{DTFT of a Delayed Impulse}

  Second, let's find the DTFT of a delayed impulse:
  \begin{align*}
    g[n] &= \begin{cases}1&n=3\\0&\mbox{otherwise}\end{cases}\\
    G(\omega) &= \sum_{n=-\infty}^\infty g[n]e^{-j\omega n}\\
    &= 1\times e^{-j\omega 3}
  \end{align*}
  So we get that
  \[
  g[n]=\delta[n-3]\leftrightarrow G(\omega)=e^{-j3\omega}
  \]
  Similarly, we could show that
  \[
  h[n]=\delta[n-6]\leftrightarrow H(\omega)=e^{-j6\omega}
  \]
\end{frame}

\begin{frame}
  \frametitle{Impulse and Delayed Impulse}

  So our signals are:
  \begin{align*}
    f[n] = \delta[n] &\leftrightarrow F(\omega)=1\\
    g[n] = \delta[n-3] &\leftrightarrow G(\omega)=e^{-3j\omega}\\
    h[n] = \delta[n-6] &\leftrightarrow H(\omega)=e^{-6j\omega}
  \end{align*}
\end{frame}


\begin{frame}
  \frametitle{Time Shift Property}

  Notice that
  \begin{align*}
    g[n] &= f[n-3]\\
    h[n] &= g[n-3].
  \end{align*}
  From the time-shift property of the DTFT, we can get that
  \begin{align*}
    G(\omega) &= e^{-j3\omega}F(\omega)\\
    H(\omega) &= e^{-j3\omega}G(\omega).
  \end{align*}
  Plugging in $F(\omega)=1$, we get
  \begin{align*}
    G(\omega) &= e^{-j3\omega}\\
    H(\omega) &= e^{-j6\omega},
  \end{align*}
  which we already know to be the right answer!
\end{frame}

\begin{frame}
  \frametitle{Convolution Property and the Impulse}

  Notice that, if $F(\omega)=1$, then anything times $F(\omega)$ gives
  itself again.  In particular,
  \begin{align*}
    G(\omega) &= G(\omega)F(\omega)\\
    H(\omega) &= H(\omega)F(\omega)
  \end{align*}
  Since multiplication in frequency is the same as convolution in time, that must mean that
  when you convolve any signal with an impulse, you get the same signal back again:
  \begin{align*}
    g[n] &= g[n] \ast \delta[n]\\
    h[n] &= h[n]\ast \delta[n]
  \end{align*}
\end{frame}

\begin{frame}
  \frametitle{Convolution Property and the Impulse}

  \centerline{\includegraphics[height=3in]{exp/dconv.png}}
\end{frame}

\begin{frame}
  \frametitle{Convolution Property and the Delayed Impulse}

  Here's another interesting thing.  Notice that
  $G(\omega)=e^{-j3\omega}$, but $H(\omega)=e^{-j6\omega}$.  So
  \begin{align*}
    H(\omega) &= e^{-j3\omega}e^{-j3\omega}\\
    &= G(\omega)G(\omega)
  \end{align*}
  Does that mean that:
  \begin{align*}
    \delta[n-6] &= \delta[n-3]\ast \delta[n-3]
  \end{align*}
\end{frame}

\begin{frame}
  \frametitle{Convolution Property and the Delayed Impulse}

  \centerline{\includegraphics[height=3in]{exp/sdconv.png}}
\end{frame}

%%%%%%%%%%%%%%%%%%%%%%%%%%%%%%%%%%%%%%%%%%%%
\section[Summary]{Summary}
\setcounter{subsection}{1}

\begin{frame}
  \frametitle{Summary}

  The DTFT (discrete time Fourier transform) of any signal is
  $X(\omega)$, given by
  \begin{align*}
    X(\omega) &= \sum_{n=-\infty}^\infty x[n]e^{-j\omega n}\\
    x[n] &= \frac{1}{2\pi}\int_{-\pi}^\pi X(\omega)e^{j\omega n}d\omega
  \end{align*}
  Particular useful examples include:
  \begin{align*}
    f[n]=\delta[n] &\leftrightarrow F(\omega)=1\\
    g[n]=\delta[n-n_0] &\leftrightarrow G(\omega)=e^{-j\omega n_0}
  \end{align*}
\end{frame}

\begin{frame}
  \frametitle{Properties of the DTFT}

  Properties worth knowing  include:
  \begin{enumerate}
    \setcounter{enumi}{-1}
  \item Periodicity: $X(\omega+2\pi)=X(\omega)$
  \item Linearity:
    \[z[n]=ax[n]+by[n]\leftrightarrow Z(\omega)=aX(\omega)+bY(\omega)
    \]
  \item Time Shift: $x[n-n_0]\leftrightarrow e^{-j\omega n_0}X(\omega)$
  \item Frequency Shift: $e^{j\omega_0 n}x[n]\leftrightarrow X(\omega-\omega_0)$
  \item Filtering is Convolution:
    \[
    y[n]=h[n]\ast x[n]\leftrightarrow Y(\omega)=H(\omega)X(\omega)
    \]
  \end{enumerate}
\end{frame}

\end{document}
