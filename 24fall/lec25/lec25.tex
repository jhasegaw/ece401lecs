\documentclass{beamer}
\usepackage{tikz,amsmath,hyperref,graphicx,stackrel,animate,amssymb}
\usetikzlibrary{positioning,shadows,arrows,shapes,calc}
\newcommand{\argmax}{\operatornamewithlimits{argmax}}
\newcommand{\argmin}{\operatornamewithlimits{argmin}}
\mode<presentation>{\usetheme{Frankfurt}}
\AtBeginSection[]
{
  \begin{frame}<beamer>
    \frametitle{Outline}
    \tableofcontents[currentsection,currentsubsection]
  \end{frame}
}
\title{Lecture 25: Overlap-Add}
\author{Mark Hasegawa-Johnson}
\date{ECE 401: Signal and Image Analysis}
\begin{document}

% Title
\begin{frame}
  \maketitle
\end{frame}

% Title
\begin{frame}
  \tableofcontents
\end{frame}

%%%%%%%%%%%%%%%%%%%%%%%%%%%%%%%%%%%%%%%%%%%%
\section[Review]{Review: Circular Convolution}
\setcounter{subsection}{1}

\begin{frame}
  \frametitle{Review: Circular convolution}

  Multiplying the DFT means {\bf circular convolution} of the time-domain signals:
  \begin{displaymath}
    y[n]=h[n]\circledast x[n] \leftrightarrow Y[k] = H[k]X[k],
  \end{displaymath}
  
  Circular convolution ($h[n]\circledast x[n]$) is defined like this:
  \begin{displaymath}
    h[n]\circledast x[n] = \sum_{m=0}^{N-1}x[m]h\left[(\!(n-m)\!)_N\right]
    = \sum_{m=0}^{N-1}h[m]x\left[(\!(n-m)\!)_N\right]
  \end{displaymath}

  Circular convolution is the same as linear convolution if and only if $N\ge L+M-1$.
\end{frame}

%%%%%%%%%%%%%%%%%%%%%%%%%%%%%%%%%%%%%%%%%%%%
\section[FFT]{Fast Fourier Transform}
\setcounter{subsection}{1}

\begin{frame}
  \frametitle{Computational Complexity: Convolution and DFT}

  Convolution is an $O\{N^2\}$ operation: each of the $N$ samples of
  $y[n]$ is created by adding up $N$ samples of $x[m]h[n-m]$:
  \begin{displaymath}
    y[n] = \sum_m x[m] h[n-m]
  \end{displaymath}
  The way we've learned it so far, the DFT is {\bf also} an $O\{N^2\}$
  operation: each of the $N$ samples of $X[k]$ is created by adding up
  $N$ samples of $x[n]e^{j\omega_k n}$:
  \begin{displaymath}
    X[k] = \sum_n x[n] e^{-j\frac{2\pi kn}{N}}
  \end{displaymath}
  However\ldots
\end{frame}

\begin{frame}
  \frametitle{The Fast Fourier Transform}

  \begin{itemize}
    \item The {\bf fast Fourier transform} (FFT) is a clever
      divide-and-conquer algorithm that computes all of the $N$
      samples of $X[k]$, from $x[n]$, in only $N\log_2 N$
      multiplications.
    \item It does this by computing all $N$ of the $X[k]$, all at
      once.
    \item Multiplications ($x[n] \times w_{k,n}$, for some coefficient
      $w_{k,n}$) are grouped together, across different groups of $k$
      and $n$.
    \item On average, each of the $N$ samples of $X[k]$ can be
      computed using only $\log_2 N$ multiplications, for a total
      complexity of $N\log_2 N$.
  \end{itemize}
\end{frame}

\begin{frame}
  \frametitle{What's the difference between $N^2$ and $N\log_2 N$?}

  Consider filtering $N=1024$ samples of audio (about 1/40 second)
  with a filter, $h[n]$, that is 1024 samples long.
  \begin{itemize}
  \item Time-domain convolution requires $1024\times 1024\approx
    1,000,000$ multiplications.  If a GPU does 40 billion
    multiplications/second, then it will take an hour of GPU time to
    apply this operation to a 1000-hour audio database.
  \item FFT requires $1024\times \log_2 1024\approx 10,000$
    multiplications.  If a GPU does 40 billion multiplications/second,
    then it will take 36 seconds of GPU time to apply this operation
    to a 1000-hour audio database.
  \end{itemize}
\end{frame}

\begin{frame}
  \frametitle{How is it used?}

  Suppose we have a 1025-sample $h[n]$, and we want to filter a
  one-hour audio (144,000,000 samples).  Divide the audio into frames,
  $x[n]$, of length $M=1024$, zero-pad to $N=L+M-1=2048$, and take
  their FFTs.
  \begin{itemize}
  \item $H[k]=\text{FFT}\{h[n]\}$: total cost is trivial, because we
    only need to do this once.
  \item $X[k]=\text{FFT}\{x[n]\}$: total cost is $N\log N$ per $M$ samples.
  \item $Y[k]=X[k]H[k]$: total cost is $N$ multiplications per $M$ samples.
  \item $y[n]=\text{FFT}^{-1}\{Y[k]\}$: total cost is $N\log N$ per
    $M$ samples.
  \end{itemize}
  Grand total: $N\times\left(2\log_2N+1\right)=2048\times 23=47104$
  multiplications per 1024 audio samples, or 46 multiplications per
  sample.
\end{frame}
  
\begin{frame}
  \frametitle{How do we recombine the $y[n]$?}

  \begin{itemize}
    \item 
      The main topic of today's lecture: how do we recombine the $y[n]$?
    \item
      Remember: each frame of $x[n]$ was 1024 samples, but after
      zero-padding and convolution, each frame of $y[n]$ is 2048 samples.
    \item
      How do we recombine them?
  \end{itemize}
\end{frame}

%%%%%%%%%%%%%%%%%%%%%%%%%%%%%%%%%%%%%%%%%%%%
\section[Overlap-Add]{Overlap-Add}
\setcounter{subsection}{1}

\begin{frame}
  Let's look more closely at what convolution is.  Each sample of x[n]
  generates an impulse response.  Those impulse responses are added
  together to make the output.

  \centerline{\animategraphics[loop,controls,width=0.9\textwidth]{10}{exp/impulseresponses}{0}{127}}
\end{frame}

\begin{frame}
  First two lines show the first two frames (input on left, output on right).  Last line shows
  the total input (left) and output (right).

  \centerline{\animategraphics[loop,controls,width=0.9\textwidth]{10}{exp/overlapadd}{0}{255}}
\end{frame}

\begin{frame}
  \frametitle{The Overlap-Add Algorithm}

  \begin{enumerate}
    \item Divide $x[n]$ into frames
    \item Generate the output from each frame
    \item  Overlap the outputs, and add them together
  \end{enumerate}
\end{frame}

\begin{frame}
  \frametitle{The Overlap-Add Algorithm}

  \begin{enumerate}
    \item Divide $x[n]$ into frames ($w[n]$ is a length-$M$ rectangle).
      \begin{align*}
        x_t[n] &= x[n+tM]w[n]\\
        X_t[k] &= \text{FFT}\{x_t[n]\}
      \end{align*}
    \item Generate the output from each frame
      \begin{align*}
        Y_t[k] &= X_t[k]H[k]\\
        y_t[n] &= \text{FFT}^{-1}\{y_t[n]\}
      \end{align*}
    \item  Overlap the outputs, and add them together
      \begin{align*}
        y[n] &= \sum_t y_t[n-tM]
      \end{align*}
  \end{enumerate}
\end{frame}

\begin{frame}
  \frametitle{Quiz}
  Try the quiz!
\end{frame}

%%%%%%%%%%%%%%%%%%%%%%%%%%%%%%%%%%%%%%%%%%%%
\section[Conclusion]{Conclusion}
\setcounter{subsection}{1}

\begin{frame}
  \frametitle{The Overlap-Add Algorithm}

  \begin{enumerate}
    \item Divide $x[n]$ into frames ($w[n]$ is a length-$M$ rectangle).
      \begin{align*}
        x_t[n] &= x[n+tM]w[n]\\
        X_t[k] &= \text{FFT}\{x_t[n]\}
      \end{align*}
    \item Generate the output from each frame
      \begin{align*}
        Y_t[k] &= X_t[k]H[k]\\
        y_t[n] &= \text{FFT}^{-1}\{y_t[n]\}
      \end{align*}
    \item  Overlap the outputs, and add them together
      \begin{align*}
        y[n] &= \sum_t y_t[n-tM]
      \end{align*}
  \end{enumerate}
\end{frame}

\end{document}
