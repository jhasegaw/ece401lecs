\documentclass{beamer}
\usepackage{tikz,amsmath,hyperref,graphicx,stackrel,animate,media9}
\usetikzlibrary{positioning,shadows,arrows,shapes,calc}
\newcommand{\argmax}{\operatornamewithlimits{argmax}}
\newcommand{\argmin}{\operatornamewithlimits{argmin}}
\mode<presentation>{\usetheme{Frankfurt}}
\AtBeginSection[]
{
  \begin{frame}<beamer>
    \frametitle{Outline}
    \tableofcontents[currentsection,currentsubsection]
  \end{frame}
}
\title{Lecture 27: Z Transform}
\author{Mark Hasegawa-Johnson}
\date{ECE 401: Signal and Image Analysis}  
\begin{document}

% Title
\begin{frame}
  \maketitle
\end{frame}

% Title
\begin{frame}
  \tableofcontents
\end{frame}

%%%%%%%%%%%%%%%%%%%%%%%%%%%%%%%%%%%%%%%%%%%%
\section[DTFT]{Review: DTFT}
\setcounter{subsection}{1}

\begin{frame}
  \frametitle{DTFT}

  The DTFT (discrete time Fourier transform) of any signal is
  $X(\omega)$, given by
  \begin{align*}
    X(\omega) &= \sum_{n=-\infty}^\infty x[n]e^{-j\omega n}\\
    x[n] &= \frac{1}{2\pi}\int_{-\pi}^\pi X(\omega)e^{j\omega n}d\omega
  \end{align*}
  Particular useful examples include:
  \begin{align*}
    f[n]=\delta[n] &\leftrightarrow F(\omega)=1\\
    g[n]=\delta[n-n_0] &\leftrightarrow G(\omega)=e^{-j\omega n_0}
  \end{align*}
\end{frame}

\begin{frame}
  \frametitle{Ideal Filters}
  Ideal Lowpass Filter:
  \[
  H_{LPF}(\omega)
  = \begin{cases} 1& |\omega|\le\omega_L,\\
    0 & \omega_c<|\omega|\le\pi.
  \end{cases}~~~\leftrightarrow~~~
  h_{LPF}[m]=\frac{\omega_c}{\pi}\mbox{sinc}(\omega_c n)
  \]
\end{frame}

\begin{frame}
  \frametitle{Ideal Filters}
  Ideal Bandpass Filter:
  \begin{align*}
    &H_{BPF}(\omega)=H_{LPF}(\omega|\omega_2)-H_{LPF}(\omega|\omega_1)\\
    &\leftrightarrow~~~ h_{BPF}[n]=\frac{\omega_2}{\pi}\mbox{sinc}(\omega_2 n)-
    \frac{\omega_1}{\pi}\mbox{sinc}(\omega_1 n)
  \end{align*}
\end{frame}

\begin{frame}
  \frametitle{Ideal Filters}
  Delayed Ideal Highpass Filter:
  \begin{align*}
    H_{HPF}(\omega)&=e^{-j\omega n_0}\left(1-H_{LPF}(\omega)\right)\\
    &\leftrightarrow\\
    h_{HPF}[n]&=\begin{cases}
    \delta[n-n_0]-\frac{\omega_c}{\pi}\mbox{sinc}(\omega_c (n-n_0)) & n_0=\mbox{integer}\\
    \mbox{sinc}(\pi (n-n_0))-\frac{\omega_c}{\pi}\mbox{sinc}(\omega_c (n-n_0)) & \mbox{otherwise}
    \end{cases}
  \end{align*}
\end{frame}

%%%%%%%%%%%%%%%%%%%%%%%%%%%%%%%%%%%%%%%%%%%%
\section{Difference Equations}
\setcounter{subsection}{1}

\begin{frame}
  \frametitle{Linearity and Time-Shift Properties}

  The linearity property of the DTFT says that
  \[
  z[n]= ax[n]+by[n]\leftrightarrow Z(\omega) = aX(\omega)+bY(\omega).
  \]
  The time-shift property says that
  \[
  z[n]= x[n-n_0] \leftrightarrow Z(\omega) = e^{-j\omega n_0} X(\omega)
  \]
\end{frame}

\begin{frame}
  \frametitle{Difference Equation}

  A {\bf difference equation} is an equation in terms of time-shifted
  values of multiple signals.  For example,
  \[
  y[n] = x[n] - 2x[n-1] + 2x[n-2]
  \]
  By combining the linearity and time-shift properties of the DTFT, we
  can translate the whole difference equation into the frequency
  domain as
  \[
  Y(\omega)  = X(\omega) -2e^{-j\omega} X(\omega) + 2 e^{-2j\omega}X(\omega)
  \]
\end{frame}

\begin{frame}
  \frametitle{Impulse Response}

  A difference equation implements a discrete-time filter.  Therefore,
  it has an {\bf impulse response}.  You can find the impulse response
  by just putting in an impulse, $x[n]=\delta[n]$, and seeing how it
  responds.  Whatever value of $y[n]$ that comes out of the filter is
  the impulse response:
  \begin{align*}
    h[n] &= \delta[n]-2\delta[n-1]+2\delta[n-2]=\begin{cases}
    1 & n=0\\-2 & n=1\\2 & n=2\\0 & \mbox{otherwise}\end{cases}
  \end{align*}
  If you wanted to use {\tt np.convolve} to implement this filter, you
  now know what impulse response to use.
  \begin{align*}
    y[n] &= x[n] -2x[n-1]+2x[n-2]\\
    & = \sum_{m=0}^2 h[m]x[n-m]
  \end{align*}
\end{frame}  

\begin{frame}
  \frametitle{Frequency Response}

  The {\bf frequency response} of a filter is the function $H(\omega)$ such that
  $Y(\omega)=H(\omega)X(\omega)$, or in other words,
  \[
  H(\omega) = \frac{Y(\omega)}{X(\omega)}
  \]
  We can get this from the DTFT of the difference equation:
  \[
  Y(\omega)  = \left(1-2e^{-j\omega}+2e^{-2j\omega}\right) X(\omega)
  \]
  \[
  H(\omega) = \frac{Y(\omega)}{X(\omega)} = \left(1-2e^{-j\omega}+2e^{-2j\omega}\right)
  \]
\end{frame}

\begin{frame}
  \frametitle{Frequency Response is DTFT of the Impulse Response}

  The frequency response is the DTFT of the impulse response.
  \begin{align*}
    h[n] &= \delta[n]-2\delta[n-1]+2\delta[n-2]\\
    H(\omega) &= 1-2e^{-j\omega}+2e^{-2j\omega}
  \end{align*}
  It sounds like an accident, that frequency response is DTFT of the
  impulse response.  But actually, it's because the method for
  computing frequency response and the method for computing impulse
  response do the same two steps, in the opposite order (see next slide).
\end{frame}

\begin{frame}
  \begin{block}{How to compute the frequency response}
    \begin{enumerate}
    \item Take the DTFT of every term, so that $ax[n-n_0]$ is
      converted to $ae^{-j\omega n_0}X(\omega)$.
    \item Divide by $X(\omega)$.
    \end{enumerate}
  \end{block}

  \begin{block}{How to compute the DTFT of the impulse response}
    \begin{enumerate}
    \item Replace $x[n]$ by $\delta[n]$, so that $ax[n-n_0]$ is
      converted to $a\delta[n-n_0]$.
    \item Take the DTFT, so that $a\delta[n-n_0]$ becomes
      $ae^{-j\omega n_0}$.
    \end{enumerate}
  \end{block}
\end{frame}
  
\begin{frame}
  \begin{block}{How to compute the frequency response}
    \begin{enumerate}
    \item Take the DTFT of every term:
      \[
      Y(\omega)= X(\omega) -2e^{-j\omega}X(\omega)+2e^{-2j\omega}X(\omega)
      \]
    \item Divide by $X(\omega)$.
      \[
      H(\omega)= 1-2e^{-j\omega} +2e^{-2j\omega}
      \]
    \end{enumerate}
  \end{block}

  \begin{block}{How to compute the DTFT of the impulse response}
    \begin{enumerate}
    \item Replace $x[n]$ by $\delta[n]$:
      \[
      h[n]= \delta[n] -2\delta[n-1] +2\delta[n-2]
      \]      
    \item Take the DTFT:
      \[
      H(\omega)= 1-2e^{-j\omega} + 2e^{-2j\omega}
      \]      
    \end{enumerate}
  \end{block}
\end{frame}

%%%%%%%%%%%%%%%%%%%%%%%%%%%%%%%%%%%%%%%%%%%%
\section[Impulses]{Every Signal is a Weighted Sum of Impulses}
\setcounter{subsection}{1}

\begin{frame}
  \frametitle{Definition of the DTFT}

  The definition of the DTFT is
  \[
  X(\omega)  = \sum_{n=-\infty}^\infty x[n]e^{-j\omega n}
  \]
  \begin{itemize}
  \item We viewed this, before, as computing the phasor $X(\omega)$ using
    the orthogonality principle: multiply $x[n]$ by a pure tone at the
    corresponding frequency, and sum over all time.
  \item But there's another useful way to think about this.
  \end{itemize}
\end{frame}

\begin{frame}
  \frametitle{A Signal is a Weighted Sum of Impulses}

  Try writing the DTFT as
  \[
  X(\omega) = \ldots + x[-1]e^{j\omega} + x[0] + x[1]e^{-j\omega} + \ldots
  \]
  This looks like the DTFT of a difference equation.  The inverse DTFT would be
  \[
  x[n] = \ldots + x[-1]\delta[n+1] + x[0]\delta[n] + x[1]\delta[n-1] + \ldots
  \]
\end{frame}
  
\begin{frame}
  \frametitle{A Signal is a Weighted Sum of Impulses}
  
  \centerline{\animategraphics[loop,controls,width=4.5in]{5}{exp/impulses}{0}{59}}

\end{frame}

\begin{frame}
  \frametitle{A Signal is a Weighted Sum of Impulses}

  So we can use the DTFT formula, $X(\omega)=\sum x[n]e^{-j\omega n}$,
  to inspire us to think about $x[n]$ as just a weighted sum of impulses:

  \vspace*{1cm}
  
  \[
  X(\omega)=\sum_{m=-\infty}^\infty x[m]e^{-j\omega m}
  \leftrightarrow x[n]=\sum_{m=-\infty}^\infty x[m]\delta[n-m]
  \]
\end{frame}

%%%%%%%%%%%%%%%%%%%%%%%%%%%%%%%%%%%%%%%%%%%%
\section{Z Transform}
\setcounter{subsection}{1}

\begin{frame}
  \frametitle{Z: a Frequency Variable for Time Shifts}

  If we're going to be working a lot with delays, instead of pure
  tones, then it helps to change our frequency variable.
  Until now, we've been working in terms of $\omega$, the frequency of the pure tone:
  \[
  X(\omega) = \sum_{n=-\infty}^\infty x[n]e^{-j\omega n}
  \]
  A unit delay, $\delta[n-1]$, has the DTFT $e^{-j\omega}$.
  Just to reduce the notation a little, let's define the basic unit to be
  $z=e^{j\omega}$, which is the transform of $\delta[n+1]$ (a unit advance).  Then we  get:
  \[
  X(z)   = \sum_{n=-\infty}^\infty x[n]z^{-n}
  \]
\end{frame}

\begin{frame}
  \frametitle{Main Use of Z Transform: Difference Equations}
  
  The main purpose of the Z transform, for now, is just so that we have less to write.
  Instead of transforming
  \[
  y[n] = x[n] - 2x[n-1] + 2x[n-2]
  \]
  to get
  \[
  Y(\omega) = \left(1-2e^{-j\omega}+2e^{-2j\omega}\right)X(\omega)
  \]
  Now we can just write
  \[
  Y(z) = \left(1-2z^{-1}+2z^{-2}\right) X(z)
  \]
\end{frame}

\begin{frame}
  \frametitle{Main Use of Z Transform: Difference Equations}
  
  The longer the difference equation, the more you will appreciate
  writing $z$ instead of $e^{j\omega}$.
  \begin{align*}
    y[n] &= 0.2x[n+3]+0.3x[n+2]+0.5x[n+1]\\
    &-0.5x[n-1]-0.3x[n-2]-0.2x[n-2]\\
  \end{align*}
  \[
  H(z)=\frac{Y(z)}{X(z)} = 0.2z^{3}+0.3z^{2}+0.5z^{1}-0.5z^{-1}-0.3z^{-2}-0.2z^{-3}
  \]
\end{frame}

\begin{frame}
  \frametitle{A Signal is a Weighted Sum of Impulses}
  
  Remember that a signal is just a weighted sum of impulses?
  \[
  x[n] = \sum_{m=-\infty}^\infty x[m]\delta[n-m]
  \]
  Since the Z-transform of $\delta[n-m]$ is $z^{-m}$, we can transform
  the above difference equation to get
  \[
  X(z) = \sum_{m=-\infty}^\infty x[m]z^{-m}.
  \]
  It's like we've converted a weighted  sum of impulses ($x[n]$)  into a
  polynomial in $z$ ($X(z)$).
\end{frame}

%%%%%%%%%%%%%%%%%%%%%%%%%%%%%%%%%%%%%%%%%%%%
\section[Zeros]{Finding the Zeros of $H(z)$}
\setcounter{subsection}{1}

\begin{frame}
  \frametitle{Z Transform and Convolution}
  
  Here's a formula for convolution:
  \[
  y[n] = \sum_{m=-\infty}^\infty h[m]x[n-m]
  \]
  Since the Z-transform of $x[n-m]$ is $z^{-m}X(z)$, we can transform
  the above difference equation to get
  \begin{align*}
  Y(z) &= \sum_{m=-\infty}^\infty h[m]z^{-m}X(z)\\
  &= H(z)X(z)
  \end{align*}
  So we confirm that $x[n]\ast h[n]\leftrightarrow H(z)X(z)$.
\end{frame}

\begin{frame}
  \frametitle{Treating $H(z)$ as a Polynomial}

  Suppose we have
  \[
  h[n] = \delta[n]-2\delta[n-1] + 2\delta[n-2]
  \]
  \begin{itemize}
  \item Is this a low-pass filter, a high-pass
    filter, or something else?
  \item $H(z)$ provides a new way of thinking about the frequency response:
    not an ideal LPF or HPF or BPF, but instead, something with
    particular {\bf zeros} in the frequency domain.
  \end{itemize}
\end{frame}

\begin{frame}
  \frametitle{Treating $H(z)$ as a Polynomial}
  
  Here's the transfer function $H(z)$:
  \[
  H(z) = 1-2z^{-1}+2z^{-2}
  \]
  Notice that we can factor that, just like any other polynomial:
  \[
  H(z) = \frac{1}{z^2}\left(z^2 -2z+2\right)
  \]
  Using the quadratic formula, we can find its roots:
  \[
  z = \frac{2\pm\sqrt{(2)^2-4\times 2}}{2}= 1\pm j
  \]
\end{frame}

\begin{frame}
  \frametitle{Treating $H(z)$ as a Polynomial}
  
  We've discovered that $H(z)$ can be written as a product of factors:
  \[
  H(z) = 1-2z^{-1}+2z^{-2} = \frac{1}{z^2}(z-z_1)(z-z_2),
  \]
  where the roots of the polynomial are
  \begin{align*}
    z_1 &= 1+j =\sqrt{2}e^{j\pi/4}\\
    z_2 &= 1-j = \sqrt{2}e^{-j\pi/4}
  \end{align*}
\end{frame}

\begin{frame}
  \frametitle{The Zeros of $H(z)$}

  \begin{itemize}
  \item The roots, $z_1$ and $z_2$, are the values of $z$ for which
    $H(z)=0$.
  \item But what does that mean?  We know that for $z=e^{j\omega}$,
    $H(z)$ is just the frequency response:
    \[
    H(\omega) = H(z)\vert_{z=e^{j\omega}}
    \]
    but the roots do not have unit magnitude:
    \begin{align*}
      z_1 &= 1+j =\sqrt{2}e^{j\pi/4}\\
      z_2 &= 1-j = \sqrt{2}e^{-j\pi/4}
    \end{align*}
  \item What it means is that, when $\omega=\frac{\pi}{4}$ (so
    $z=e^{j\pi/4}$), then $|H(\omega)|$ is as close to a zero as it
    can possibly get.  So at that frequency, $|H(\omega)|$ is as low
    as it can get.
  \end{itemize}
\end{frame}

\begin{frame}
  \centerline{\animategraphics[loop,controls,width=4.5in]{10}{exp/magresponse}{0}{99}}
\end{frame}

\begin{frame}
  \centerline{\animategraphics[loop,controls,width=4.75in]{10}{exp/toneresponse}{0}{99}}
\end{frame}

\begin{frame}
  \frametitle{Vectors in the Complex Plane}

  Suppose we write $|H(z)|$ like this:
  \[
  \vert H(z)\vert = \frac{1}{|z|^2}\times\vert z-z_1\vert\times\vert z-z_2\vert
  = \vert z-z_1\vert\times\vert z-z_2\vert
  \]
  Now let's evaluate at $z=e^{j\omega}$:
  \[
  \vert H(\omega)\vert = 
  \vert e^{j\omega} -z_1\vert\times\vert e^{j\omega}-z_2\vert
  \]
  What we've discovered is that $|H(\omega)|$ is small when the vector
  distance $|e^{j\omega}-z_1|$ is small, in other words, when
  $z=e^{j\omega}$ is as close as possible to one of the zeros.
\end{frame}

\begin{frame}
  \centerline{\animategraphics[loop,controls,width=4.5in]{10}{exp/magresponse}{0}{99}}
\end{frame}

\begin{frame}
  \frametitle{Why This is Useful}

  Now we have another way of thinking about frequency response.
  \begin{itemize}
    \item Instead of just LPF, HPF, or BPF, we can design a filter to have
      zeros at particular frequencies, $\angle z_1$ and $\angle z_2$.
    \item The magnitude $|H(\omega)|$ at the zero frequency is
      proportional to $|e^{j\omega}-z_1|$.
    \item Using this trick, we can design filters that have much more
      subtle frequency responses than just an ideal LPF, BPF, or HPF.
  \end{itemize}
\end{frame}

%%%%%%%%%%%%%%%%%%%%%%%%%%%%%%%%%%%%%%%%%%%%
\section[Summary]{Summary}
\setcounter{subsection}{1}

\begin{frame}
  \frametitle{Summary: Z Transform}
  \begin{itemize}
  \item A {\bf difference equation} is an equation in terms of
    time-shifted copies of $x[n]$ and/or $y[n]$.
  \item We can find the frequency response
    $H(\omega)=Y(\omega)/X(\omega)$ by taking the DTFT of each term of
    the difference equation.  This will result in a lot of terms of
    the form $e^{j\omega n_0}$ for various $n_0$.
  \item We have less to write if we use a new frequency variable,
    $z=e^{j\omega}$.  This leads us to the Z transform:
    \[
    X(z) = \sum_{n=-\infty}^\infty x[n]z^{-n}
    \]
  \end{itemize}
\end{frame}

\begin{frame}
  \frametitle{Zeros of the Transfer Function}
  \begin{itemize}
  \item The {\bf transfer function}, $H(z)$, is a polynomial in $z$.
  \item The zeros of the transfer function are usually complex numbers, $z_k$.
  \item The frequency response, $H(\omega) = H(z)\vert_{z=e^{j\omega}}$, has a dip
    whenever $\omega$ equals the phase of any of the zeros, $\omega=\angle z_k$.
  \end{itemize}
\end{frame}

%%%%%%%%%%%%%%%%%%%%%%%%%%%%%%%%%%%%%%%%%%%%
\section[Example]{Written Example}
\setcounter{subsection}{1}

\begin{frame}
  \frametitle{Written Example}

  Use the Z-transform method to find the magnitude frequency response
  of
  \[
  y[n] = x[n]-0.9x[n-1]
  \]
  
\end{frame}

\end{document}
