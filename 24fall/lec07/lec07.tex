\documentclass{beamer}
\usepackage{tikz,amsmath,hyperref,graphicx,stackrel,animate,amssymb}
\usetikzlibrary{positioning,shadows,arrows,shapes,calc}
\newcommand{\argmax}{\operatornamewithlimits{argmax}}
\newcommand{\argmin}{\operatornamewithlimits{argmin}}
\mode<presentation>{\usetheme{Frankfurt}}
\AtBeginSection[]
{
  \begin{frame}<beamer>
    \frametitle{Outline}
    \tableofcontents[currentsection,currentsubsection]
  \end{frame}
}
\title{Lecture 7: Sampling and Aliasing}
\author{Mark Hasegawa-Johnson\\These slides are in the public domain.}
\date{ECE 401: Signal and Image Analysis}  
\begin{document}

% Title
\begin{frame}
  \maketitle
\end{frame}

% Title
\begin{frame}
  \tableofcontents
\end{frame}

%%%%%%%%%%%%%%%%%%%%%%%%%%%%%%%%%%%%%%%%%%%%
\section[Review]{Review: Spectrum of continuous-time signals}
\setcounter{subsection}{1}

\begin{frame}
  \frametitle{Two-sided spectrum}

  The {\bf spectrum} of $x(t)$ is the set of frequencies, and their
  associated phasors,
  \[
  \mbox{Spectrum}\left( x(t) \right) =
  \left\{ (f_{-N},a_{-N}), \ldots, (f_0,a_0), \ldots, (f_N,a_N) \right\}
  \]
  such that
  \[
  x(t) = \sum_{k=-N}^N a_ke^{j2\pi f_kt}
  \]
\end{frame}

\begin{frame}
  \frametitle{Fourier's theorem}

  One reason the spectrum is useful is that {\bf\em any} periodic
  signal can be written as a sum of cosines.  Fourier's theorem says that
  any $x(t)$ that is periodic, i.e.,
  \[
  x(t+T_0) = x(t)
  \]
  can be written as
  \[
  x(t) = \sum_{k=-\infty}^\infty X_k e^{j2\pi k F_0 t}
  \]
  which is a special case of the spectrum for periodic signals:
  $f_k=kF_0$, and $a_k=X_k$, and
  \[
  F_0 = \frac{1}{T_0}
  \]
\end{frame}

\begin{frame}
  \frametitle{Fourier Series}

  \begin{itemize}
  \item {\bf Analysis}  (finding the spectrum, given the waveform):
    \[
    X_k = \frac{1}{T_0}\int_0^{T_0} x(t)e^{-j2\pi kt/T_0}dt
    \]
  \item {\bf Synthesis} (finding the waveform, given the spectrum):
    \[
    x(t) = \sum_{k=-\infty}^\infty X_k e^{j2\pi kt/T_0}
    \]
  \end{itemize}
\end{frame}  

%%%%%%%%%%%%%%%%%%%%%%%%%%%%%%%%%%%%%%%%%%%%
\section[Sampling]{Sampling}
\setcounter{subsection}{1}

\begin{frame}
  \frametitle{How to sample a continuous-time signal}

  Suppose you have some continuous-time signal, $x(t)$, and you'd like
  to sample it, in order to store the sample values in a computer.
  The samples are collected once every $T_s=\frac{1}{F_s}$ seconds:
  \begin{displaymath}
    x[n] = x(t=nT_s)
  \end{displaymath}
\end{frame}

\begin{frame}
  \frametitle{Example: a 1kHz sine wave}

  For example, suppose $x(t)=\sin(2\pi 1000t)$.  By sampling at
  $F_s=16000$ samples/second, we get
  \begin{displaymath}
    x[n] = \sin\left(2\pi 1000\frac{n}{16000}\right) = \sin(\pi n/8)
  \end{displaymath}
  
  \centerline{\includegraphics[width=4.5in]{exp/sampled_sine1.png}}
\end{frame}

%%%%%%%%%%%%%%%%%%%%%%%%%%%%%%%%%%%%%%%%%%%%
\section[Aliasing]{Aliasing}
\setcounter{subsection}{1}

\begin{frame}
  \frametitle{Can every  sine wave be reconstructed from its samples?}

  The question immediately arises: can every sine wave be reconstructed from its samples?

  The answer, unfortunately, is ``no.''
\end{frame}
\begin{frame}
  \frametitle{Can every  sine wave be reconstructed from its samples?}
  
  For example, two signals $x_1(t)$ and $x_2(t)$, at 10kHz and 6kHz respectively:
  \begin{displaymath}
    x_1(t)=\cos(2\pi 10000t),~~~x_2(t)=\cos(2\pi 6000t)
  \end{displaymath}
  Let's sample them at $F_s=16,000$ samples/second:
  \begin{displaymath}
    x_1[n]=\cos\left(2\pi 10000\frac{n}{16000}\right),~~~x_2[n]=\cos\left(2\pi 6000\frac{n}{16000}\right)
  \end{displaymath}
  Simplifying a bit, we discover that $x_1[n]=x_2[n]$.  We say that
  the 10kHz tone has been ``aliased'' to 6kHz:
  \begin{align*}
    x_1[n]&=\cos\left(\frac{5\pi n}{4}\right)=\cos\left(\frac{3\pi n}{4}\right)\\
    x_2[n]&=\cos\left(\frac{3\pi n}{4}\right)=\cos\left(\frac{5\pi n}{4}\right)
  \end{align*}
\end{frame}

\begin{frame}
  \frametitle{Can every  sine wave be reconstructed from its samples?}

  \centerline{\includegraphics[width=4.5in]{exp/sampled_aliasing.png}}
\end{frame}

\begin{frame}
  \frametitle{What is the highest frequency that can be reconstructed?}

  The minimum sampling rate that avoids aliasing is called the {\bf
    Nyquist rate,} and it is $F_s=2f$.  Conversely, we talk about the
  the {\bf Nyquist frequency,} $F_N=F_S/2$, which is the highest
  frequency pure tone that can be reconstructed at sampling rate
  $F_s$. If $x(t)=\cos(2\pi F_Nt)$, then
  \begin{displaymath}
    x[n]=\cos\left(2\pi F_N\frac{n}{F_S}\right)=\cos(\pi n)=(-1)^n
  \end{displaymath}

  \centerline{\includegraphics[width=4.5in]{exp/sampled_nyquist.png}}
\end{frame}

\begin{frame}
  \frametitle{Nyquist rate vs. Nyquist frequency}

  Unfortunately, due to historical reasons, the terms ``Nyquist rate''
  and ``Nyquist frequency'' are sort of opposite in meaning:
  \begin{itemize}
  \item The Nyquist rate is the {\bf lowest sampling rate} at which
    you can sample a signal without aliasing.  If the highest
    frequency in a signal is $f$, then the Nyquist rate is $F_s=2f$.
  \item The Nyquist frequency is the {\bf highest frequency} that will
    be reproduced without aliasing, i.e., $F_N=F_s/2$.
  \end{itemize}
\end{frame}

  
\begin{frame}
  \frametitle{Sampling below Nyquist rate $\Rightarrow$ Aliasing to a
    frequency below the Nyquist frequency}

  If you try to sample below the {\bf Nyquist rate} ($F_s < 2f$, like
  the one shown on the left), then the tone gets aliased to a {\bf
    frequency alias} $f_a$ below the {\bf Nyquist frequency}
  ($f_a<F_N$, like the one shown on the right).
  \centerline{\includegraphics[width=4.5in]{exp/sampled_aliasing.png}}
\end{frame}

\begin{frame}
  \frametitle{When does aliasing happen?}

  Aliasing happens:
  \begin{itemize}
  \item When a continuous-time signal, $x(t)=\cos(2\pi ft)$, is
    sampled below the Nyquist rate: $F_s<2f$.
  \item When a tone has already been sampled at a high enough sampling
    rate, but then you {\bf downsample} to a rate below Nyquist.
  \end{itemize}
\end{frame}
\begin{frame}
    For example, suppose you have sampled at $F_s=2.88f$, so that you
    have
    \begin{align*}
      x[n] = \cos\left(\frac{2\pi f}{F_s}n\right)=\cos\left(\frac{2\pi}{2.88}n\right),
    \end{align*}
    but if you then {\bf downsample} by throwing away every second sample,
    \begin{align*}
      y[n] = x[2n],~~~\text{integer values of}~n,
    \end{align*}
    then you wind up with a new sampling rate of only $F_s=1.44f$,
    which means the signal can be aliased to a lower frequency below
    Nyquist:
    \begin{align*}
      y[n] = \cos\left(\frac{2\pi}{1.44}n\right) = \cos\left(\left(2\pi-\frac{2\pi}{1.44}\right)n\right)
      =\cos\left(\frac{0.88\pi}{1.44}n\right)
    \end{align*}
\end{frame}
    
%%%%%%%%%%%%%%%%%%%%%%%%%%%%%%%%%%%%%%%%%%%%
\section[Aliased Frequency]{Aliased Frequency}
\setcounter{subsection}{1}

\begin{frame}
  \frametitle{Aliased Frequency}

  Suppose you have a cosine at frequency $f$:
  \[
  x(t) = \cos(2\pi ft)
  \]
  Suppose you sample it at $F_s$ samples/second.  If $F_s$ is not high
  enough, it might get aliased to some other frequency, $f_a$.
  \[
  x[n] = \cos(2\pi f n/F_s) = \cos(2\pi f_a n/F_s)
  \]
  How can you predict what $f_a$ will be?
\end{frame}

\begin{frame}
  \frametitle{Aliased Frequency}

  \centerline{\includegraphics[height=1in]{exp/cosine_aliasing.png}}
  Aliasing comes from two sources:
  \begin{align*}
    \cos(\phi) &= \cos(2\pi n -\phi)\\
    \cos(\phi) &= \cos(\phi-2\pi n)
  \end{align*}
  The equations above are true for any integer $n$.
\end{frame}


\begin{frame}
  \frametitle{Aliased Frequency}

  \centerline{\includegraphics[height=1in]{exp/cosine_aliasing.png}}
  Let's plug in $\phi=\frac{2\pi fn}{F_s}$, and $2\pi = \frac{2\pi
    F_s}{F_s}$.  That gives us:
  \begin{align*}
    \cos\left(\frac{2\pi fn}{F_s}\right) &= \cos\left(\frac{2\pi n(F_s-f)}{F_s}\right)\\
    \cos\left(\frac{2\pi fn}{F_s}\right) &= \cos\left(\frac{2\pi (f-F_s)n}{F_s}\right)
  \end{align*}
  So a discrete-time cosine at frequency $f$ is also a cosine at
  frequency $F_s-f$, and it's also a cosine at $f-F_s$.
\end{frame}

\begin{frame}
  \frametitle{Spectrum of a Continuous-time Cosine}

  A continuous-time cosine is the sum of two complex exponentials:
  \begin{displaymath}
    \cos(0.75\pi n)=\frac{1}{2}e^{j0.75\pi n}+\frac{1}{2}e^{-j0.75\pi n}
  \end{displaymath}
  \centerline{\includegraphics[height=0.5\textheight]{exp/ct_cosine.png}}
\end{frame}

\begin{frame}
  \frametitle{Spectrum of a Discrete-time Cosine}

  A discrete-time cosine is {\bf still} just the sum of two complex
  exponentials, but each of those two complex exponentials is
  identical to an exponential at any other multiple of $2\pi$:
  \begin{displaymath}
    e^{j2.75\pi n}=e^{-j2\pi n}e^{j0.75\pi n}=e^{j0.75\pi n}
  \end{displaymath}
  \centerline{\includegraphics[height=0.5\textheight]{exp/dt_cosine0.png}}
\end{frame}

\begin{frame}
  \frametitle{Spectrum of an Aliased Discrete-time Cosine}

  Now consider what happens as we lower $F_s$.  As we lower $F_s$, the
  frequency $\omega=\frac{2\pi f}{F_s}$ gets higher and higher, until
  aliasing occurs:
  \begin{align*}
    \cos\left(\omega n\right) &= \cos\left((2\pi-\omega)n\right)
  \end{align*}
    \centerline{\animategraphics[loop,controls,height=0.5\textheight]{20}{exp/dt_cosine}{0}{45}}
\end{frame}

\begin{frame}
  \frametitle{Aliased Frequency}

  \begin{itemize}
    \item A discrete-time cosine at frequency $f$ is also a cosine at
      frequency $F_s-f$, and it's also a cosine at $f-F_s$.
    \item So which of those frequencies will we hear when we play
      the sinusoid back again?
    \item {\bf ANSWER:} any frequency that can be reconstructed by the
      analog-to-digital converter.  That means any frequency below the
      Nyquist frequency, $F_N=F_s/2$.
  \end{itemize}
  
\end{frame}

\begin{frame}
  \frametitle{Aliased Frequency}
  \centerline{\includegraphics[width=4.5in]{exp/sampled_cosine1.png}}
\end{frame}
\begin{frame}
  \frametitle{Aliased Frequency}
  \centerline{\includegraphics[width=4.5in]{exp/sampled_cosine2.png}}
\end{frame}

\begin{frame}
  \frametitle{Aliased Frequency}

  All of the following frequencies are actually {\bf the same
    frequency} when a cosine is sampled at $F_s$ samples/second.
  \[
  f_a \in\left\{f - \ell F_s, \ell F_s - f : \ell\in \mbox{any integer}\right\}
  \]
  The ``aliased frequency'' is whichever of those is below Nyquist
  ($F_s/2$).  Usually there's only one that's below Nyquist, so you
  can just look for
  \[
  f_a =\min\left(f - \ell F_s, \ell F_s - f : \ell\in \mbox{any integer}\right)
  \]
  
\end{frame}

%%%%%%%%%%%%%%%%%%%%%%%%%%%%%%%%%%%%%%%%%%%%
\section[Aliased Phase]{Aliased Phase}
\setcounter{subsection}{1}

\begin{frame}
  \frametitle{Sine is Different}

  \centerline{\includegraphics[height=1in]{exp/sine_aliasing.png}}
  Sine waves are different for the following reason:
  \begin{align*}
    \sin(\phi) &= -\sin(2\pi n -\phi)\\
    \sin(\phi) &= \sin(\phi-2\pi n)
  \end{align*}
\end{frame}

\begin{frame}
  \frametitle{Sine is Different}
  \centerline{\includegraphics[height=1in]{exp/sine_aliasing.png}}
  Therefore:
  \begin{align*}
    \sin\left(\frac{2\pi fn}{F_s}\right) &= -\sin\left(\frac{2\pi n(F_s-f)}{F_s}\right)\\
    \sin\left(\frac{2\pi fn}{F_s}\right) &= \sin\left(\frac{2\pi (f-F_s)n}{F_s}\right)
  \end{align*}
  So a discrete-time sine at frequency $f$ is also a {\bf negative} sine at
  frequency $F_s-f$, and a {\bf positive} sine at frequency $f-F_s$.
\end{frame}

\begin{frame}
  \frametitle{Spectrum of a Continuous-time Sine}

  A continuous-time sine is the {\bf difference} of two complex exponentials:
  \begin{displaymath}
    \sin(0.75\pi n)=\frac{1}{2j}e^{j0.75\pi n}-\frac{1}{2j}e^{-j0.75\pi n}
  \end{displaymath}
  \centerline{\includegraphics[height=0.5\textheight]{exp/ct_sine.png}}
\end{frame}

\begin{frame}
  \frametitle{Spectrum of a Discrete-time Sine}

  A discrete-time sine is still just the difference of two complex
  exponentials, but each of those two complex exponentials is
  identical to an exponential at any other multiple of $2\pi$:
  \begin{displaymath}
    e^{j2.75\pi n}=e^{-j2\pi n}e^{j0.75\pi n}=e^{j0.75\pi n}
  \end{displaymath}
  \centerline{\includegraphics[height=0.5\textheight]{exp/dt_sine0.png}}
\end{frame}

\begin{frame}
  \frametitle{Spectrum of an Aliased Discrete-time Sine}

  Now consider what happens as we lower $F_s$.  As we lower $F_s$, the
  frequency $\omega=\frac{2\pi f}{F_s}$ gets higher and higher, until
    aliasing occurs:
    \begin{align*}
      \sin\left(\omega n\right) &= -\sin\left((2\pi-\omega)n\right)
    \end{align*}
    \centerline{\animategraphics[loop,controls,height=0.5\textheight]{20}{exp/dt_sine}{0}{45}}
\end{frame}

%\begin{frame}
%  \frametitle{Sine is Different}
%  \centerline{\includegraphics[width=4.5in]{exp/sampled_sine2.png}}
%\end{frame}


\begin{frame}
  \frametitle{Aliased Phase of a General Phasor}

  For a general complex exponential, we get:
  \begin{align*}
    ze^{j\phi} = ze^{j(\phi-2\pi n)} = \left(z^* e^{j(2\pi n-\phi)}\right)^*
  \end{align*}
  Therefore:
  \begin{align*}
    \Re\left\{z e^{j\frac{2\pi fn}{F_s}} \right\} =
    \Re\left\{z e^{j\frac{2\pi (f-F_s)n}{F_s}} \right\} =
    \Re\left\{z^* e^{j\frac{2\pi (F_s-f)n}{F_s}} \right\}
  \end{align*}
\end{frame}


\begin{frame}
  \frametitle{Aliased Phase of a General Phasor}

  Suppose we have some frequency $f$, and we're trying to find its
  aliased frequency $f_a$.
  \begin{itemize}
  \item Among the several possibilities, if $f_a=F_s-f$ is below
    Nyquist, then that's the frequency we'll hear.  Its phasor will be
    the complex conjugate of the original phasor,
    \[
    z_a = z^*
    \]
  \item On the other hand, if $f_a=f-F_s$ is below Nyquist, then
    that's the frequency we'll hear.  Its phasor will be the same as
    the phasor of the original sinusoid:
    \[
    z_a = z
    \]    
  \end{itemize}
  
\end{frame}


\begin{frame}
  \frametitle{Aliased Phase of a  General Phasor}
  \centerline{\includegraphics[width=4.5in]{exp/sampled_quarter.png}}
\end{frame}

\begin{frame}
  \frametitle{Try the Quiz!}
  Go to the course webpage, and try today's quiz!
\end{frame}

%%%%%%%%%%%%%%%%%%%%%%%%%%%%%%%%%%%%%%%%%%%%
\section[Summary]{Summary}
\setcounter{subsection}{1}

\begin{frame}
  \frametitle{Summary}

  \begin{itemize}
  \item A sampled sinusoid can be reconstructed perfectly if the
    Nyquist criterion is met, $f < \frac{F_s}{2}$.
  \item If the Nyquist criterion is violated, then:
    \begin{itemize}
    \item If $\frac{F_s}{2}<f<F_s$, then it will be aliased to
      \begin{align*}
        f_a &= F_s-f\\
        z_a &= z^*
      \end{align*}
      i.e., the sign of all sines will be reversed.
    \item If $F_s < f < \frac{3F_s}{2}$, then it will be aliased to
      \begin{align*}
        f_a &= f-F_s\\
        z_a &= z
      \end{align*}
    \end{itemize}
  \end{itemize}
\end{frame}

%%%%%%%%%%%%%%%%%%%%%%%%%%%%%%%%%%%%%%%%%%%%
\section[Example]{Written Example}
\setcounter{subsection}{1}

\begin{frame}
  \frametitle{Written Example}

  Sketch a sinusoid with some arbitrary phase (say, $-\pi/4$).  Show
  where the samples are if it's sampled:
  \begin{itemize}
  \item more than twice per period
  \item more than once per period, but less than twice per period
  \item less than once per period
  \end{itemize}
\end{frame}
  

\end{document}
