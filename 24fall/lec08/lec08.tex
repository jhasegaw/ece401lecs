\documentclass{beamer}
\usepackage{tikz,amsmath,hyperref,graphicx,stackrel,animate,amssymb}
\usetikzlibrary{positioning,shadows,arrows,shapes,calc}
\newcommand{\argmax}{\operatornamewithlimits{argmax}}
\newcommand{\argmin}{\operatornamewithlimits{argmin}}
\mode<presentation>{\usetheme{Frankfurt}}
\AtBeginSection[]
{
  \begin{frame}<beamer>
    \frametitle{Outline}
    \tableofcontents[currentsection,currentsubsection]
  \end{frame}
}
\title{Lecture 8: Interpolation}
\author{Mark Hasegawa-Johnson\\These slides are in the public domain.}
\date{ECE 401: Signal and Image Analysis}
\begin{document}

% Title
\begin{frame}
  \maketitle
\end{frame}

% Title
\begin{frame}
  \tableofcontents
\end{frame}

%%%%%%%%%%%%%%%%%%%%%%%%%%%%%%%%%%%%%%%%%%%%
\section[Sampling]{Review: Sampling}
\setcounter{subsection}{1}

\begin{frame}
  \frametitle{How to sample a continuous-time signal}

  Suppose you have some continuous-time signal, $x(t)$, and you'd like
  to sample it, in order to store the sample values in a computer.
  The samples are collected once every $T_s=\frac{1}{F_s}$ seconds:
  \begin{displaymath}
    x[n] = x(t=nT_s)
  \end{displaymath}
\end{frame}

%%%%%%%%%%%%%%%%%%%%%%%%%%%%%%%%%%%%%%%%%%%%
\section[Interpolation]{Interpolation: Discrete-to-Continuous Conversion}
\setcounter{subsection}{1}

\begin{frame}
  \frametitle{How can we get $x(t)$ back again?}

  We've already seen one method of getting $x(t)$ back again: we can
  find all of the cosine components, and re-create the corresponding
  cosines in continuous time.

  There is a more general method, that we can use for any signal, even
  signals that are not composed of pure tones.  It involves
  multiplying each of the samples, $x[n]$, by a short-time pulse,
  $p(t)$, as follows:
  \begin{displaymath}
    y(t) = \sum_{n=-\infty}^\infty y[n]p(t-nT_s)
  \end{displaymath}
\end{frame}

\begin{frame}
  \frametitle{Rectangular pulses}

  For example, suppose that the pulse is  just a  rectangle,
  \begin{displaymath}
    p(t) = \begin{cases}
      1 & -\frac{T_S}{2}\le t<\frac{T_S}{2}\\
      0 & \mbox{otherwise}
    \end{cases}
  \end{displaymath}

  \centerline{\includegraphics[width=4.5in]{exp/pulse_rectangular.png}}  
\end{frame}

\begin{frame}
  \frametitle{Rectangular pulses = Piece-wise constant interpolation}

  The result is a  piece-wise constant interpolation of the digital signal:

  \centerline{\includegraphics[width=4.5in]{exp/interpolated_rectangular.png}}

  Although this has roughly the right values, it is discontinuous.
  The discontinuities make an image ``blocky,'' and add high-frequency
  noise to an audio signal.
\end{frame}

\begin{frame}
  \frametitle{Aliasing caused by imperfect interpolation}

  For example, suppose we start with
  $x[n]=\cos\left(\frac{n\pi}{8}\right)$, and interpolate using a
  rectangular pulse with $T_s=\frac{1}{8000}$ of a second.  We wind up
  with
  \begin{displaymath}
    y(t) = a\cos\left(2\pi 1000 n\right)+b\cos\left(2\pi 7000n\right)+c\cos\left(2\pi 9000n\right)+\cdots,
  \end{displaymath}
  where $a\approx 1$, and $b\approx 0$ and $c\approx 0$, but not
  exactly.  Since $b$ and $c$ are not zero, we can hear tones at those
  aliased frequencies.
\end{frame}

\begin{frame}
  \frametitle{Triangular pulses}

  The rectangular pulse has the disadvantage that $y(t)$ is discontinuous.
  We can eliminate the discontinuities by using a triangular pulse:
  \begin{displaymath}
    p(t) = \begin{cases}
      1-\frac{|t|}{T_S} & -T_S\le t<T_S\\
      0 & \mbox{otherwise}
    \end{cases}
  \end{displaymath}

  \centerline{\includegraphics[width=4.5in]{exp/pulse_triangular.png}}  
\end{frame}

\begin{frame}
  \frametitle{Triangular pulses = Piece-wise linear interpolation}

  The result is a  piece-wise linear interpolation of the digital signal:

  \centerline{\includegraphics[width=4.5in]{exp/interpolated_triangular.png}}

  Although the signal is continuous, its derivatives are
  discontinuous.  Images look pretty good like this, but audio signals
  still sound noisy (the ear is more sensitive to high frequencies
  than the eye).
\end{frame}

\begin{frame}
  \frametitle{Aliasing caused by imperfect interpolation}

  For example, suppose we start with
  $x[n]=\cos\left(\frac{n\pi}{8}\right)$, and interpolate using a
  triangular pulse with $T_s=\frac{1}{8000}$ of a second.  We wind up
  with
  \begin{displaymath}
    y(t) = a\cos\left(2\pi 1000 n\right)+b\cos\left(2\pi 7000n\right)+c\cos\left(2\pi 9000 n\right)+\cdots,
  \end{displaymath}
  where $b$ and $c$ are much smaller than they were for the rectangular pulse, but
  still not zero. 
\end{frame}

\begin{frame}
  \frametitle{Cubic spline pulses}

  The triangular pulse has the disadvantage that, although $y(t)$ is continuous, its
  first derivative is discontinuous.  We can eliminate discontinuities in the first derivative
  by using a cubic-spline pulse:
  \begin{displaymath}
    p(t) = \begin{cases}
      1-\frac{3}{2}\left(\frac{|t|}{T_S}\right)^2 +\frac{1}{2}\left(\frac{|t|}{T_s}\right)^3 & 0\le |t|\le T_S\\
      -\frac{3}{2}\left(\frac{|t|-2T_s}{T_S}\right)^2\left(\frac{|t|-T_s}{T_S}\right) & T_S\le |t|\le 2T_S\\
      0 & \mbox{otherwise}
    \end{cases}
  \end{displaymath}

\end{frame}

\begin{frame}
  \frametitle{Cubic spline pulses}

  The triangular pulse has the disadvantage that, although $y(t)$ is continuous, its
  first derivative is discontinuous.  We can eliminate discontinuities in the first derivative
  by using a cubic-spline pulse:
  \centerline{\includegraphics[width=4.5in]{exp/pulse_spline.png}}  
\end{frame}

\begin{frame}
  \frametitle{Cubic spline pulses = Piece-wise cubic interpolation}

  The result is a  piece-wise cubic interpolation of the digital signal:

  \centerline{\includegraphics[width=4.5in]{exp/interpolated_spline.png}}  
\end{frame}

\begin{frame}
  \frametitle{Aliasing caused by imperfect interpolation}

  For example, suppose we start with
  $x[n]=\cos\left(\frac{n\pi}{8}\right)$, and interpolate using a
  cubic-spline pulse with $T_s=\frac{1}{8000}$ of a second.  We wind up
  with
  \begin{displaymath}
    y(t) = a\cos\left(2\pi 1000 n\right)+b\cos\left(2\pi 7000n\right)+c\cos\left(2\pi 9000n\right)+\cdots,
  \end{displaymath}
  where $b$ and $c$ are much smaller than they were for the triangular pulse, but
  still not zero. 
\end{frame}

\begin{frame}
  \frametitle{Sinc pulses}

  The cubic spline has no discontinuities, and no slope  discontinuities, but it still has
  discontinuities in its second derivative and all higher derivatives.  Can we fix those?

  The answer: yes!  If we keep smoothing $p(t)$, we can find a signal such that:
  \begin{itemize}
  \item $p(0)=1$
  \item $p(nT_s)=0$ for all integers $n\ne 0$
  \item All of the derivatives of $p(t)$ are continuous everywhere
  \end{itemize}
  The resulting signal is called a ``sinc function.''
\end{frame}
  
\begin{frame}
  \frametitle{Sinc pulses}

  We can reconstruct a signal that has  no discontinuities in any of its derivatives
  by using an  ideal sinc pulse:
  \begin{displaymath}
    p(t) = \frac{\sin(\pi t/T_S)}{\pi t/T_S}
  \end{displaymath}

  \centerline{\includegraphics[width=4.5in]{exp/pulse_sinc.png}}

\end{frame}

\begin{frame}
  \frametitle{Sinc pulse = ideal bandlimited interpolation}

  The resulting signal has no high-frequency noise: it contains only
  frequencies below the Nyquist frequency. We call this an ``ideal
  bandlimited interpolation.''

  \centerline{\includegraphics[width=4.5in]{exp/interpolated_sinc.png}}  
\end{frame}

\begin{frame}
  \frametitle{Sinc interpolation = no aliasing!}

  For example, suppose we start with
  $x[n]=\cos\left(\frac{n\pi}{8}\right)$, and interpolate using a
  sinc function with $F_s=8000$ samples/second.  We wind up with exactly:
  \begin{displaymath}
    y(t) = \cos\left(1000\pi n\right)
  \end{displaymath}
  Perfect!
\end{frame}

\begin{frame}
  \frametitle{Try the quiz!}

  Go to the course web page, and try today's quiz!
\end{frame}

%%%%%%%%%%%%%%%%%%%%%%%%%%%%%%%%%%%%%%%%%%%%
\section[Interpolation]{Interpolation: Upsampling a signal}
\setcounter{subsection}{1}

\begin{frame}
  \frametitle{Changing the sampling rate of a signal}

  Suppose we have an audio signal ($x[n]$) sampled at $11025$
  samples/second, but we really want to play it back at $44100$
  samples second.  We can do that by creating a new signal, $y[n]$, at
  $M=4$ times the sampling rate of $x[n]$:
  \begin{displaymath}
    y[n] = \left\{\begin{array}{ll}
    x[n/M] & n=\text{integer multiple of}~M\\
    \text{interpolated value}&\text{otherwise}
    \end{array}\right.
  \end{displaymath}
\end{frame}

\begin{frame}
  \frametitle{Upsampling}

  We split this process into two steps.  First, {\bf upsampling} means
  that we just insert zeros between the samples of $x[n]$:
  \begin{displaymath}
    u[n] = \left\{\begin{array}{ll}
    x[n/M] & n=\text{integer multiple of}~M\\
    0&\text{otherwise}
    \end{array}\right.
  \end{displaymath}
\end{frame}

\begin{frame}
  \frametitle{Interpolation}

  Second, we generate the missing samples by interpolation:
  \begin{align*}
    y[n] &= \sum_{m=-\infty}^\infty u[m] p[n-m]\\
    &=\left\{\begin{array}{ll}
    x[n/M] & n=\text{integer multiple of}~M\\
    \text{interpolated value}&\text{otherwise}
    \end{array}\right.
  \end{align*}
  The second line of the equality holds if
  \begin{displaymath}
    p[n] =\left\{\begin{array}{ll}
    1 & n=0\\
    0 & n=\text{nonzero integer multiple of}~M\\
    \text{anything}&\text{otherwise}
    \end{array}\right.
  \end{displaymath}
\end{frame}
    
\begin{frame}
  \frametitle{Interpolation Kernels}
  All of these interpolation kernels satisfy the condition on the previous slide:
  
  \centerline{\includegraphics[height=0.8\textheight]{exp/dt_interpolators.png}}  
\end{frame}

%%%%%%%%%%%%%%%%%%%%%%%%%%%%%%%%%%%%%%%%%%%%
\section[Summary]{Summary}
\setcounter{subsection}{1}

\begin{frame}
  \frametitle{Summary}
  \begin{itemize}
  \item Piece-wise constant interpolation = interpolate using a rectangle
  \item Piece-wise linear interpolation = interpolate using a triangle
  \item Cubic-spline interpolation = interpolate using a spline
  \item Ideal interpolation = interpolate using a sinc
  \end{itemize}
\end{frame}

\end{document}
