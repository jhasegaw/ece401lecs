\documentclass{beamer}
\usepackage{tikz,amsmath,hyperref,graphicx,stackrel,animate,amssymb}
\usetikzlibrary{positioning,shadows,arrows,shapes,calc}
\newcommand{\argmax}{\operatornamewithlimits{argmax}}
\newcommand{\argmin}{\operatornamewithlimits{argmin}}
\mode<presentation>{\usetheme{Frankfurt}}
\AtBeginSection[]
{
  \begin{frame}<beamer>
    \frametitle{Outline}
    \tableofcontents[currentsection,currentsubsection]
  \end{frame}
}
\title{Lecture 23: Discrete Fourier Transform}
\author{Mark Hasegawa-Johnson}
\date{ECE 401: Signal and Image Analysis}
\begin{document}

% Title
\begin{frame}
  \maketitle
\end{frame}

% Title
\begin{frame}
  \tableofcontents
\end{frame}

%%%%%%%%%%%%%%%%%%%%%%%%%%%%%%%%%%%%%%%%%%%%
\section[DTFT]{Review: DTFT}
\setcounter{subsection}{1}

\begin{frame}
  \frametitle{Review: DTFT}

  The DTFT (discrete time Fourier transform) of any signal is
  $X(\omega)$, given by
  \begin{align*}
    X(\omega) &= \sum_{n=-\infty}^\infty x[n]e^{-j\omega n}\\
    x[n] &= \frac{1}{2\pi}\int_{-\pi}^\pi X(\omega)e^{j\omega n}d\omega
  \end{align*}
  Particular useful examples include:
  \begin{align*}
    f[n]=\delta[n] &\leftrightarrow F(\omega)=1\\
    g[n]=\delta[n-n_0] &\leftrightarrow G(\omega)=e^{-j\omega n_0}
  \end{align*}
\end{frame}

\begin{frame}
  \frametitle{Properties of the DTFT}

  Properties worth knowing  include:
  \begin{enumerate}
    \setcounter{enumi}{-1}
  \item Periodicity: $X(\omega+2\pi)=X(\omega)$
  \item Linearity:
    \[z[n]=ax[n]+by[n]\leftrightarrow Z(\omega)=aX(\omega)+bY(\omega)
    \]
  \item Time Shift: $x[n-n_0]\leftrightarrow e^{-j\omega n_0}X(\omega)$
  \item Frequency Shift: $e^{j\omega_0 n}x[n]\leftrightarrow X(\omega-\omega_0)$
  \item Filtering is Convolution:
    \[
    y[n]=h[n]\ast x[n]\leftrightarrow Y(\omega)=H(\omega)X(\omega)
    \]
  \end{enumerate}
\end{frame}

%%%%%%%%%%%%%%%%%%%%%%%%%%%%%%%%%%%%%%%%%%%%
\section[DFT]{DFT}
\setcounter{subsection}{1}

\begin{frame}
  \frametitle{How can we compute the DTFT?}

  \begin{itemize}
  \item
    The DTFT has a big problem: it requires an infinite-length
    summation, therefore you can't compute it on a computer.
  \item
    The DFT solves this problem by assuming a {\bf finite length}
    signal.
  \item
    ``$N$ equations in $N$ unknowns:'' if there are $N$ samples in the
    time domain ($x[n],~0\le n\le N-1$), then there are only $N$
    independent samples in the frequency domain ($X(\omega_k),~0\le
    k\le N-1$).
  \end{itemize}
\end{frame}

\begin{frame}
  \frametitle{Finite-length signal}

  First, assume that $x[n]$ is nonzero only for $0\le n\le N-1$.  Then
  the DTFT can be computed as:

  \begin{displaymath}
    X(\omega) = \sum_{n=0}^{N-1} x[n]e^{-j\omega n}
  \end{displaymath}
\end{frame}

\begin{frame}
  \frametitle{N equations in N  unknowns}

  Since there are only $N$ samples in the time domain, there are also
  only $N$ {\bf independent} samples in the frequency domain:

  \begin{displaymath}
    X[k] = X(\omega_k) = \sum_{n=0}^{N-1} x[n]e^{-j\omega_k n} = \sum_{n=0}^{N-1} x[n]e^{-j\frac{2\pi kn}{N}}
  \end{displaymath}
  where
  \begin{displaymath}
    \omega_k = \frac{2\pi k}{N},~~0\le k\le N-1
  \end{displaymath}
\end{frame}
    

\begin{frame}
  \frametitle{Discrete Fourier Transform}

  Putting it all together, we get the formula for the DFT:
  \begin{displaymath}
    X[k] = \sum_{n=0}^{N-1} x[n]e^{-j\frac{2\pi kn}{N}}
  \end{displaymath}
\end{frame}
    
\begin{frame}
  \frametitle{Inverse Discrete Fourier Transform}

  \begin{displaymath}
    X[k] = \sum_{n=0}^{N-1} x[n]e^{-j\frac{2\pi kn}{N}}
  \end{displaymath}
  Using  orthogonality, we can  also show that
  \begin{displaymath}
    x[n] = \frac{1}{N}\sum_{k=0}^{N-1} X[k]e^{j\frac{2\pi kn}{N}}
  \end{displaymath}
\end{frame}
    
%%%%%%%%%%%%%%%%%%%%%%%%%%%%%%%%%%%%%%%%%%%%
\section[Example]{Example}
\setcounter{subsection}{1}

\begin{frame}
  \frametitle{Example}

  \centerline{\includegraphics[width=\textwidth]{exp/simple_signal.png}}
  
  Consider the signal
  \begin{displaymath}
    x[n] = \begin{cases} 1&\mbox{n=0,1}\\
      0 & \mbox{n=2,3}\\
      \mbox{undefined} & \mbox{otherwise}
    \end{cases}
  \end{displaymath}
\end{frame}

\begin{frame}
  \frametitle{Example DFT}

  \begin{align*}
    X[k] &= \sum_{n=0}^3 x[n] e^{-j\frac{2\pi kn}{4}}\\
    &= 1 + e^{-j\frac{2\pi k}{4}}\\
    &= \begin{cases}
      2 & k=0\\
      1-j & k=1\\
      0 & k=2\\
      1+j & k=3
    \end{cases}
  \end{align*}
\end{frame}

\begin{frame}
  \frametitle{Example IDFT}
  
  \begin{align*}
    X[k] = [ 2, (1-j), 0, (1+j) ]
  \end{align*}

  \begin{align*}
    x[n] &= \frac{1}{4}\sum_{k=0}^3 X[k]e^{j\frac{2\pi kn}{4}}\\
    &= \frac{1}{4}\left(2+ (1-j)e^{j\frac{2\pi n}{4}}+(1+j)e^{j\frac{6\pi n}{4}}\right)\\
    &= \frac{1}{4}\left(2+ (1-j)j^n +(1+j)(-j)^n\right)\\
    &=\begin{cases}
    1 & n=0,1\\
    0 & n=2,3
    \end{cases}
  \end{align*}
\end{frame}


%%%%%%%%%%%%%%%%%%%%%%%%%%%%%%%%%%%%%%%%%%%%
\section[Delta]{Example: Shifted Delta Function}
\setcounter{subsection}{1}

\begin{frame}
  \frametitle{Shifted Delta Function}

  In many cases, we can find the DFT directly from the DTFT.  For example:
  \begin{displaymath}
    h[n]=\delta[n-n_0] ~~\leftrightarrow~~H(\omega)=e^{-j\omega n_0}
  \end{displaymath}

  {\bf If and only if the signal is less than length $N$,} we can just
  plug in $\omega_k=\frac{2\pi k}{N}$:
  \begin{displaymath}
    h[n]=\delta[n-n_0]~~\leftrightarrow~~
    H[k]=\begin{cases}
    e^{-j\frac{2\pi kn_0}{N}} & 0\le n_0\le N-1\\
    \mbox{undefined} &\mbox{otherwise}
    \end{cases}
  \end{displaymath}
\end{frame}

\begin{frame}
  \frametitle{Quiz}

  Go to the course webpage, and try today's quiz!
\end{frame}

%%%%%%%%%%%%%%%%%%%%%%%%%%%%%%%%%%%%%%%%%%%%
\section[Cosine]{Example: Cosine}
\setcounter{subsection}{1}

\begin{frame}
  \frametitle{Cosine}

  Finding the DFT of a cosine is possible, but harder than you might think.
  Consider:
  \begin{displaymath}
    x[n] = \cos(\omega_0 n)
  \end{displaymath}
  This signal violates the first requirement of a DFT:
  \begin{itemize}
  \item $x[n]$ must be finite length.
  \end{itemize}
\end{frame}

\begin{frame}
  \frametitle{Cosine}

  We can make $x[n]$ finite-length by windowing it, like this:
  \begin{displaymath}
    x[n] = \cos(\omega_0 n)w[n],
  \end{displaymath}

  where $w[n]$ is the rectangular window,
  \begin{displaymath}
    w[n] =\begin{cases}
    1 & 0\le n \le N-1\\
    0 & \mbox{otherwise}
    \end{cases}
  \end{displaymath}
\end{frame}

\begin{frame}
  \frametitle{Cosine}

  Now that $x[n]$ is finite length, we can just take its DTFT, and
  then sample at $\omega_k=\frac{2\pi k}{N}$:
  \begin{displaymath}
    X[k] = X(\omega_k) = \sum_{n=0}^{N-1} x[n]e^{-j\omega_k n}
  \end{displaymath}

\end{frame}

\begin{frame}
  \frametitle{Linearity and Frequency-Shift Properties of the DTFT}

  But how do we solve this equation?
  \begin{displaymath}
    X(\omega_k) = \sum_{n=0}^{N-1} \cos(\omega_0 n)w[n] e^{-j\omega_k n}
  \end{displaymath}
  The answer is, surprisingly, that we can use two properties of the
  DTFT:
  \begin{itemize}
  \item {\bf Linearity:} $x_1[n]+x_2[n]~~\leftrightarrow~~X_1(\omega)+X_2(\omega)$
  \item {\bf Frequency Shift:} $e^{j\omega_0 n}z[n]~~\leftrightarrow~~Z(\omega-\omega_0)$
  \end{itemize}

\end{frame}

\begin{frame}
  \frametitle{Linearity and Frequency-Shift Properties of the DTFT}

  \begin{itemize}
  \item {\bf Linearity:}
    \begin{displaymath}
      \cos(\omega_0 n)w[n] = \frac{1}{2}e^{j\omega_0 n}w[n] + \frac{1}{2}e^{-j\omega_0 n}w[n]
    \end{displaymath}
  \item {\bf Frequency Shift:}
    \begin{displaymath}
      e^{j\omega_0 n}w[n] \leftrightarrow W(\omega-\omega_0)
    \end{displaymath}
  \end{itemize}
  Putting them together, we have that
  \begin{displaymath}
    \cos(\omega_0 n)w[n]~~\leftrightarrow~~\frac{1}{2}W(\omega-\omega_0) + \frac{1}{2}W(\omega+\omega_0)
  \end{displaymath}
\end{frame}


\begin{frame}
  \frametitle{DFT of a Cosine}

  Putting it together,
  \begin{displaymath}
    x[n]=\cos(\omega_0 n)w[n] ~~\leftrightarrow~~
    X(\omega_k)=\frac{1}{2}W(\omega_k-\omega_0) + \frac{1}{2}W(\omega_k+\omega_0)
  \end{displaymath}
  where $W(\omega)$ is the Dirichlet form:
  \begin{align*}
    W(omega) &= e^{-j\omega\frac{N-1}{2}}\frac{\sin(\omega N/2)}{\sin(\omega/2)}
  \end{align*}
\end{frame}

\begin{frame}
  \frametitle{DFT of a Cosine}

  Here's the DFT of
  \[
  x[n] = \cos\left(\frac{2\pi 20.3}{N}n\right) w[n]
  \]
  
  \centerline{\includegraphics[width=\textwidth]{exp/dft_of_cosine1.png}}
\end{frame}

\begin{frame}
  \frametitle{DFT of a Cosine}

  Remember that $W(\omega)=0$ whenever $\omega$ is a multiple of
  $\frac{2\pi}{N}$.  But {\bf the DFT only samples at multiples
    of}~$\frac{2\pi}{N}$!  So if $\omega_0$ is {\bf also} a multiple
  of $\frac{2\pi}{N}$, then the DFT of a cosine is just a pair of
  impulses in frequency:
  \centerline{\includegraphics[width=\textwidth]{exp/dft_of_cosine2.png}}
\end{frame}

%%%%%%%%%%%%%%%%%%%%%%%%%%%%%%%%%%%%%%%%%%%%
\section[Properties of DFT]{Properties of the DFT}
\setcounter{subsection}{1}


\begin{frame}
  \frametitle{Periodic in Frequency}

  Just as $X(\omega)$ is periodic with period $2\pi$, in the same way,
  $X[k]$ is periodic with period $N$:

  \begin{align*}
    X[k+N] &= \sum_n x[n]e^{-j\frac{2\pi (k+N)n}{N}}\\
    &= \sum_n x[n]e^{-j\frac{2\pi kn}{N}}e^{-j\frac{2\pi Nn}{N}}\\    
    &= \sum_n x[n]e^{-j\frac{2\pi kn}{N}}\\
    &= X[k]
  \end{align*}
\end{frame}  
        
\begin{frame}
  \frametitle{Periodic in Time}

  The inverse DFT is also periodic in time!  $x[n]$ is undefined
  outside $0\le n\le N-1$, but if we accidentally try to compute
  $x[n]$ at any other times, we end up with:

  \begin{align*}
    x[n+N] &= \frac{1}{N}\sum_k X[k]e^{j\frac{2\pi k(n+N)}{N}}\\
    &= \frac{1}{N}\sum_k X[k]e^{j\frac{2\pi kn}{N}}e^{j\frac{2\pi kN}{N}}\\    
    &= \frac{1}{N}\sum_k X[k]e^{j\frac{2\pi kn}{N}}\\
    &= x[n]
  \end{align*}
\end{frame}  
        
\begin{frame}
  \frametitle{Linearity}

  \begin{displaymath}
    ax_1[n]+bx_2[n] ~~\leftrightarrow~~aX_1[k]+bX_2[k]
  \end{displaymath}
\end{frame}  
        
\begin{frame}
  \frametitle{Samples of the DTFT}

  If $x[n]$ is finite length, with length of at most $N$ samples, then
  \begin{displaymath}
    X[k] = X(\omega_k),~~\omega_k = \frac{2\pi k}{N}
  \end{displaymath}
\end{frame}  
        
\begin{frame}
  \frametitle{Conjugate Symmetry of the DTFT}

  Here's a property of the DTFT that we didn't talk about much.
  Suppose that $x[n]$ is real.  Then
  \begin{align*}
    X(-\omega) &= \sum_{n=-\infty}^\infty x[n]e^{-j(-\omega)n}\\
    &= \sum_{n=-\infty}^\infty x[n]e^{j\omega n}\\
    &= \left(\sum_{n=-\infty}^\infty x[n]e^{-j\omega n}\right)^*\\
    &= X^*(\omega)
  \end{align*}
  
\end{frame}  

\begin{frame}
  \frametitle{Conjugate Symmetry of the DFT}

  \begin{displaymath}
    X(\omega) = X^*(-\omega)
  \end{displaymath}
  Remember that the DFT, $X[k]$, is just the samples of the DTFT,
  sampled at $\omega_k=\frac{2\pi k}{N}$.  So that means that
  conjugate symmetry also applies to the DFT:
  \begin{displaymath}
    X[k] = X^*[-k]
  \end{displaymath}
  But remember that the DFT is periodic with a period of $N$, so
  \begin{displaymath}
    X[k] = X^*[-k] = X^*[N-k]
  \end{displaymath}
\end{frame}


\begin{frame}
  \frametitle{Frequency Shift}

  The frequency shift property of the DTFT also applies to the DFT:
  \begin{displaymath}
    w[n]e^{j\omega_0 n}~~\leftrightarrow~~W(\omega-\omega_0)
  \end{displaymath}
  If $\omega=\frac{2\pi k}{N}$, and if $\omega_0=\frac{2\pi k_0}{N}$,
  then we get
  \begin{displaymath}
    w[n]e^{j\frac{2\pi k_0 n}{N}} ~~\leftrightarrow~~W[k-k_0]
  \end{displaymath}
\end{frame}

\begin{frame}
  \frametitle{Time Shift}

  The time shift property of the DTFT was
  \begin{displaymath}
    x[n-n_0]~~\leftrightarrow~~e^{j\omega n_0}X(\omega)
  \end{displaymath}

  The same thing also applies to the DFT, except that {\bf the DFT is
    finite in time}.  Therefore we have to use what's called a ``circular shift:''
  \begin{displaymath}
    x\left[((n-n_0))_N\right]~~\leftrightarrow~~e^{-j\frac{2\pi kn_0}{N}}X[k]
  \end{displaymath}
  where $((n-n_0))_N$ means ``$n-n_0$, modulo $N$.''  We'll talk more
  about what that means in the next lecture.
\end{frame}


%%%%%%%%%%%%%%%%%%%%%%%%%%%%%%%%%%%%%%%%%%%%
\section[Summary]{Summary}
\setcounter{subsection}{1}


\begin{frame}
  \frametitle{DFT Examples}

  \begin{enumerate}
  \item
    \begin{displaymath}
      x[n]=[1,1,0,0]~~\leftrightarrow~~X[k]=[2,1-j,0,1+j]
    \end{displaymath}
  \item
    \begin{displaymath}
      x[n]=\delta[n-n_0]~~\leftrightarrow~~X[k]=\begin{cases}
      e^{-j\frac{2\pi kn_0}{N}} & 0\le n_0\le N-1\\
      \mbox{undefined} & \mbox{otherwise}
      \end{cases}
    \end{displaymath}
  \item 
    \begin{align*}
      &x[n]=w[n]\cos(\omega_0 n)\\
      &\leftrightarrow~~X[k]=
      \frac{1}{2}W\left[k-\frac{N\omega_0}{2\pi}\right]+
      \frac{1}{2}W\left[k+\frac{N\omega_0}{2\pi}\right]
    \end{align*}
  \end{enumerate}
\end{frame}

\begin{frame}
  \frametitle{DFT Properties}

  \begin{enumerate}
  \item {\bf Periodic in Time and Frequency:}
    \begin{displaymath}
      x[n]=x[n+N],~~~~X[k]=X[k+N]
    \end{displaymath}
  \item {\bf Linearity:}
    \begin{displaymath}
      ax_1[n]+bx_2[n]~~\leftrightarrow~~aX_1[k]+bX_2[k]
    \end{displaymath}
  \item {\bf Samples of the DTFT:} if $x[n]$ has length at most $N$ samples, then
    \begin{displaymath}
      X[k] = X(\omega_k),~~~\omega_k=\frac{2\pi k}{N}
    \end{displaymath}
  \item {\bf Time \& Frequency Shift:}
    \begin{align*}
      x[n]e^{j\frac{2\pi k_0 n}{N}} ~~&\leftrightarrow~~X[k-k_0]\\
      x\left[((n-n_0))_N\right] ~~&\leftrightarrow~~X[k]e^{-j\frac{2\pi kn_0}{N}}
    \end{align*}
  \end{enumerate}
\end{frame}

%%%%%%%%%%%%%%%%%%%%%%%%%%%%%%%%%%%%%%%%%%%%
\section[Written]{Written Example}
\setcounter{subsection}{1}


\begin{frame}
  \frametitle{Written  Example}

  Show that the signal $x[n]=\delta[n-n_0]$ obeys the conjugate
  symmetry properties of both the DFT and DTFT.
\end{frame}


\end{document}
