\documentclass{beamer}
\usepackage{tikz,amsmath,hyperref,graphicx,stackrel,animate}
\usetikzlibrary{positioning,shadows,arrows,shapes,calc,dsp,chains}
\DeclareMathOperator{\sinc}{sinc}
\newcommand{\argmax}{\operatornamewithlimits{argmax}}
\newcommand{\argmin}{\operatornamewithlimits{argmin}}
\mode<presentation>{\usetheme{Frankfurt}}
\AtBeginSection[]
{
  \begin{frame}<beamer>
    \frametitle{Outline}
    \tableofcontents[currentsection,currentsubsection]
  \end{frame}
}
\title{Lecture 14: DT Filtering of CT Signals}
\author{Mark Hasegawa-Johnson\\These slides are in the public domain.}
\date{ECE 401: Signal and Image Analysis}  
\begin{document}

% Title
\begin{frame}
  \maketitle
\end{frame}

% Title
\begin{frame}
  \tableofcontents
\end{frame}

%%%%%%%%%%%%%%%%%%%%%%%%%%%%%%%%%%%%%%%%%%%%
\section{Review}
\setcounter{subsection}{1}

\begin{frame}
  \frametitle{Sampling: Continuous Time $\rightarrow$ Discrete Time}

  A signal is sampled by measuring its value once every $T_s$ seconds:
  \begin{displaymath}
    x[n] = x(t=nT_s)
  \end{displaymath}
  The spectrum of the DT signal has components at $\omega=\frac{2\pi
    f}{F_s}$, and also at every $2\pi\ell+\omega$ and $2\pi\ell-\omega$, for
  every integer $\ell$.  Aliasing occurs unless $|\omega|\le\pi$.
\end{frame}


\begin{frame}
  \frametitle{Interpolation: Discrete Time $\rightarrow$ Continuous Time}

  A CT signal $y(t)$ can be created from a DT signal $y[n]$ by interpolation:
  \begin{displaymath}
    y(t) = \sum_{n=-\infty}^\infty y[n]p(t-nT_s)
  \end{displaymath}
  \begin{itemize}
  \item $p(t)=$rectangle $\Rightarrow$ PWC interpolation
  \item $p(t)=$triangle $\Rightarrow$ PWL interpolation
  \item $p(t)=\sinc\left(\frac{\pi t}{T_s}\right)$ $\Rightarrow$ perfectly bandlimited interpolation,
    $y(t)$ has no spectral components above $F_N$
  \end{itemize}
\end{frame}

\begin{frame}
  \frametitle{Convolution}

  Convolution (finite impulse response filtering) is a generalization of weighted local averaging:
  \begin{displaymath}
    y[n] = h[n]\ast x[n] \equiv \sum_m x[m] h[n-m] = \sum_m x[n-m] h[m]
  \end{displaymath}
  \begin{itemize}
  \item If all samples of $h[n]$ are positive, then it's a weighted
    local averaging filter
  \item If the samples of $h[n]$ are positive for $n>0$ and negative
    for $n<0$ (or vice versa), then it's a weighted local differencing
    filter
  \end{itemize}
\end{frame}

\begin{frame}
  \frametitle{Frequency Response}
  \begin{itemize}
  \item {\bf Tones in $\rightarrow$ Tones out}
    \begin{align*}
      x[n]=e^{j\omega n} &\rightarrow y[n]=H(\omega)e^{j\omega n}\\
      x[n]=\cos\left(\omega n\right)
      &\rightarrow y[n]=|H(\omega)|\cos\left(\omega n+\angle H(\omega)\right)\\
      x[n]=A\cos\left(\omega n+\theta\right)
      &\rightarrow y[n]=A|H(\omega)|\cos\left(\omega n+\theta+\angle H(\omega)\right)
    \end{align*}
  \item where the {\bf Frequency Response} is given by
    \[
    H(\omega) = \sum_m h[m]e^{-j\omega m}
    \]
  \end{itemize}
\end{frame}  

%%%%%%%%%%%%%%%%%%%%%%%%%%%%%%%%%%%%%%%%%%%%
\section{DT Filtering}
\setcounter{subsection}{1}

\begin{frame}
  \frametitle{DT Filtering of CT Signals}
  \begin{center}
    \begin{tikzpicture}
      \node[dspnodeopen,dsp/label=right] (y) at (8,0) {$y(t)$};
      \node[dspsquare] (da) at (7,0) {D/A} edge[dspconn](y);
      \node[dspsquare] (h) at (4,0) {$y[n]=h[n]\ast x[n]$};
      \draw (h) -- (da) node[midway,above]{$y[n]$};
      \node[dspsquare] (ad) at (1,0) {A/D};
      \draw (ad) -- (h) node[midway,above]{$x[n]$};
      \node[dspnodeopen,dsp/label=left] (x) at (0,0) {$x(t)$} edge[dspconn](ad);
    \end{tikzpicture}
  \end{center}
  Constraints:
  \begin{itemize}
  \item Assume that A/D and D/A use same $F_s$
  \item Assume $F_s\ge 2f_{\text{max}}$
  \item Assume sinc interpolation
  \end{itemize}
\end{frame}

\begin{frame}
  \frametitle{DT Filtering of CT Signals}

  \begin{itemize}
  \item If $h[n]$ is a local averager, what's the relationship of $y(t)$ to $x(t)$?
  \item If $h[n]$ is a local differencer, what's the relationship of $y(t)$ to $x(t)$?
  \end{itemize}
\end{frame}

%%%%%%%%%%%%%%%%%%%%%%%%%%%%%%%%%%%%%%%%%%%%
\section{Sampling}
\setcounter{subsection}{1}

\begin{frame}
  \frametitle{Fourier Series}

  To start with, let's assume $x(t)$ is periodic and bandlimited, so:
  \begin{align*}
    x(t)&=\sum_{k=-N}^{N}X_ke^{j2\pi kF_0t}\\
    x[n]&=\sum_{k=-N}^{N}X_ke^{jk\omega_0n},
  \end{align*}
  where $\omega_0=\frac{2\pi F_0}{F_s}$, and
  $N=\lfloor\frac{F_s/2}{F_0}\rfloor$ is number of harmonics between 0
  and the Nyquist frequency.
\end{frame}

\begin{frame}
  \frametitle{Spectrum of $x(t)$}

  \centerline{\includegraphics[width=\textwidth]{exp/spectrum_xt.png}}
\end{frame}

\begin{frame}
  \frametitle{Spectrum of $x[n]$}

  \centerline{\includegraphics[width=\textwidth]{exp/spectrum_xn.png}}
\end{frame}

\begin{frame}
  \frametitle{Sampling of Periodic Bandlimited Signals}
  Now we know how to get from $x(t)$ to $x[n]$, assuming $x(t)$ is
  periodic and bandlimited:
  \begin{align*}
    x(t)&=\sum_{k=-N}^{N}X_ke^{j2\pi kF_0t}\\
    x[n]&=\sum_{k=-N}^{N}X_ke^{jk\omega_0n}
  \end{align*}
  Next: What is $y[n]$?
\end{frame}

%%%%%%%%%%%%%%%%%%%%%%%%%%%%%%%%%%%%%%%%%%%%
\section{Filtering}
\setcounter{subsection}{1}

\begin{frame}
  \frametitle{DT Filtering of a Pure Tone}
  
  \begin{center}
    \begin{tikzpicture}
      \node[dspnodeopen,dsp/label=right] (y) at (8,0) {$y(t)$};
      \node[dspsquare] (da) at (7,0) {D/A} edge[dspconn](y);
      \node[dspsquare] (h) at (4,0) {$y[n]=h[n]\ast x[n]$};
      \draw (h) -- (da) node[midway,above]{$y[n]$};
      \node[dspsquare] (ad) at (1,0) {A/D};
      \draw (ad) -- (h) node[midway,above]{$x[n]$};
      \node[dspnodeopen,dsp/label=left] (x) at (0,0) {$x(t)$} edge[dspconn](ad);
    \end{tikzpicture}
  \end{center}
  Let's focus on just one of the harmonics of $x(t)$:
  \begin{align*}
    x(t)&=Xe^{j2\pi ft}\\
    x[n]&=Xe^{j\omega n}\\
    y[n]&=Ye^{j\omega n}\\
    y(t)&=Ye^{j2\pi ft}
  \end{align*}
  Can we find the relationship between the two phasors $Y$ and $X$?
\end{frame}

\begin{frame}
  \frametitle{Relationship Between $Y$ and $X$}

  Remember that $y[n]=x[n]\ast h[n]$, i.e.,
  \begin{displaymath}
    y[n] = \sum_{m=-\infty}^\infty h[m]x[n-m]
  \end{displaymath}
  What happens if we plug in $x[n]=Xe^{j\omega n}$?
  \begin{align*}
    y[n] &= \sum_{m=-\infty}^\infty h[m]Xe^{j\omega(n-m)}\\
    &= Xe^{j\omega n}\sum_{m=-\infty}^\infty h[m]e^{-j\omega m}\\
    &= X e^{j\omega n}H(\omega) 
  \end{align*}
  So $Y=H(\omega)X$, where $H(\omega)$ is the frequency response!
\end{frame}

\begin{frame}
  \frametitle{Filtering Periodic Signals}

  Now let's generalize to the case when $x(t)$ is any periodic
  bandlimited signal
  \begin{align*}
    x(t)&=\sum_{k=-N}^{N}X_ke^{j2\pi kF_0t}\\
    x[n]&=\sum_{k=-N}^{N}X_ke^{jk\omega_0n}
  \end{align*}
  Since the system is linear, adding inputs $\Rightarrow$ add outputs.
  So the output is
  \begin{align*}
    y[n]&=\sum_{k=-N}^{N}Y_ke^{jk\omega_0n},
    &=\sum_{k=-N}^{N}H(k\omega_0)X_ke^{jk\omega_0n},
  \end{align*}
  \ldots where $H(k\omega_0)$ is the frequency response of the system
  at frequency $k\omega_0$.
\end{frame}

%%%%%%%%%%%%%%%%%%%%%%%%%%%%%%%%%%%%%%%%%%%%
\section{Interpolation}
\setcounter{subsection}{1}

\begin{frame}
  \frametitle{DT Filtering of CT Signals}

  Remember, here's the overall system:

  \begin{center}
    \begin{tikzpicture}
      \node[dspnodeopen,dsp/label=right] (y) at (8,0) {$y(t)$};
      \node[dspsquare] (da) at (7,0) {D/A} edge[dspconn](y);
      \node[dspsquare] (h) at (4,0) {$y[n]=h[n]\ast x[n]$};
      \draw (h) -- (da) node[midway,above]{$y[n]$};
      \node[dspsquare] (ad) at (1,0) {A/D};
      \draw (ad) -- (h) node[midway,above]{$x[n]$};
      \node[dspnodeopen,dsp/label=left] (x) at (0,0) {$x(t)$} edge[dspconn](ad);
    \end{tikzpicture}
  \end{center}
  \ldots and remember that the D/A uses ideal bandlimited sinc interpolation.
\end{frame}

\begin{frame}
  \frametitle{Sinc Interpolation}

  Remember that the sinc is the function that perfectly reconstructs
  any signal below the Nyquist frequency:
  \begin{align*}
    y(t) &= \sum_{n=-\infty}^{\infty}y[n] \text{sinc}\left(\pi(t-nT_s)\right)\\
    &=\begin{cases}y[n]&t=nT_s\\
    \text{bandlimited interpolation}&\text{otherwise}\end{cases}
  \end{align*}
  \ldots where ``bandlimited interpolation'' means that $y[n]$'s
  spectrum for $-\pi<\omega<\pi$ is perfectly reconstructed by
  $y(t)$'s spectrum for $-\frac{F_s}{2}<f<\frac{F_s}{2}$, and that
  $y(t)$ has no spectrum outside that range.
\end{frame}

\begin{frame}
  \frametitle{Ideal Bandlimited Sinc Interpolation of a Periodic Signal}

  Putting it all together, we have:
  \begin{align*}
    x(t)&=\sum_{k=-N}^{N}X_ke^{j2\pi kF_0t}\\
    x[n]&=\sum_{k=-N}^{N}X_ke^{jk\omega_0n},\\
    y[n]&=\sum_{k=-N}^{N}Y_ke^{jk\omega_0n},\\
    y(t)&=\sum_{k=-N}^{N}Y_ke^{j2\pi kF_0n},
  \end{align*}
  where $\omega_0=\frac{2\pi F_0}{F_s}$, and
  $N=\lfloor\frac{F_s/2}{F_0}\rfloor$, and $Y_k=H(k\omega_0)X_k$.
\end{frame}

\begin{frame}
  \frametitle{Quiz}

  Go to the course webpage, and try the quiz!
\end{frame}

%%%%%%%%%%%%%%%%%%%%%%%%%%%%%%%%%%%%%%%%%%%%
\section{Rectangular Averager}
\setcounter{subsection}{1}

\begin{frame}
  \frametitle{Rectangular Averager}

  \begin{displaymath}
    y[n]=\sum_m h[m]x[n-m]
  \end{displaymath}
  Consider the case of the rectangular averager:
  \begin{displaymath}
    h[n]=\left\{\begin{array}{ll}
    \frac{1}{N} & -\left(\frac{N-1}{2}\right)\le n\le \left(\frac{N-1}{2}\right)\\
    0 & \mbox{otherwise}
    \end{array}\right.
  \end{displaymath}
\end{frame}

\begin{frame}
  \frametitle{Rectangular Averaging: Low-Frequency Cosine}

  When the input is low-frequency, the output of an averager is almost
  the same as the input:

  \centerline{\animategraphics[loop,controls,width=\linewidth]{20}{exp/averager_lowfreq}{0}{34}}
\end{frame}

\begin{frame}
  \frametitle{Rectangular Averaging: High-Frequency Cosine}

  When the input is high-frequency, the system averages over almost
  one complete period, so the output is close to zero:

  \centerline{\animategraphics[loop,controls,width=\linewidth]{20}{exp/averager_highfreq}{0}{34}}
\end{frame}

\begin{frame}
  \frametitle{Rectangular Averaging: General Case}

  Remember the general form for the frequency response:
  \begin{align*}
    Y &= X\sum_{m=-\infty}^\infty h[m]e^{-j\omega m}\\
    &= \frac{X}{N}\sum_{m=-(N-1)/2}^{(N-1)/2} e^{-j\omega m}\\
    &= \frac{X}{N}\left(1+2\sum_{m=1}^{(N-1)/2}\cos(\omega m)\right)
  \end{align*}
  \begin{itemize}
  \item If $\omega$ is very small, all terms are positive, so the
    output is large.
  \item If $\omega$ is larger, then the summation includes both
    positive and negative terms, so the output is small.
  \end{itemize}
\end{frame}

\begin{frame}
  \frametitle{Spectral Plots: Rectangular Averager}

  The averager retains low-frequency components, but reduces
  high-frequency components:
  \centerline{\includegraphics[width=\textwidth]{exp/averager_spectra.png}}
\end{frame}

\begin{frame}
  \frametitle{Waveforms: Rectangular Averager}

  The averager tends to smooth out the waveform:
  \centerline{\includegraphics[width=\textwidth]{exp/averager_waveforms.png}}
\end{frame}


%%%%%%%%%%%%%%%%%%%%%%%%%%%%%%%%%%%%%%%%%%%%
\section{Binary Differencer}
\setcounter{subsection}{1}


\begin{frame}
  \frametitle{Binary Differencer}

  \begin{displaymath}
    y[n]=\sum_m h[m]x[n-m]
  \end{displaymath}
  Consider the case of the binary differencer:
  \begin{displaymath}
    h[n]=\left\{\begin{array}{ll}
    1 & n=0\\
    -1 & n=1\\
    0 & \mbox{otherwise}
    \end{array}\right.
  \end{displaymath}
  \ldots so that $y[n]=x[n]-x[n-1]$.
\end{frame}

\begin{frame}
  \frametitle{Binary Differencer: Low-Frequency Cosine}

  When the input is low-frequency, the difference between neighboring
  samples is nearly zero:
  \centerline{\animategraphics[loop,controls,width=\linewidth]{20}{exp/differencer_lowfreq}{0}{34}}
\end{frame}

\begin{frame}
  \frametitle{Binary Differencer: High-Frequency Cosine}

  When the input is high-frequency, the difference between neighboring
  samples is large:
  \centerline{\animategraphics[loop,controls,width=\linewidth]{20}{exp/differencer_highfreq}{0}{34}}
\end{frame}

\begin{frame}
  \frametitle{Binary Differencer: General Case}

  Remember the general form for the frequency response:
  \begin{align*}
    Y &= X\sum_{m=-\infty}^\infty h[m]e^{-j\omega m}\\
    &= X\left(1-e^{-j\omega}\right)\\
    &= X\left(e^{j\omega/2}-e^{-j\omega/2}\right)e^{-j\omega/2}\\
    &= 2jX\sin\left(\frac{\omega}{2}\right)e^{-j\omega/2}
  \end{align*}
  \begin{itemize}
  \item If $\omega$ is very small, $\sin(\omega/2)$ is very small
  \item As $\omega\rightarrow\pi$ (high frequencies), $\sin(\omega/2)\rightarrow 1$
  \end{itemize}
\end{frame}

\begin{frame}
  \frametitle{Spectral Plots: Binary Differencer}

  The binary differencer removes the 0Hz component, but keeps high frequencies:
  \centerline{\includegraphics[width=\textwidth]{exp/differencer_spectra.png}}
\end{frame}

\begin{frame}
  \frametitle{Waveforms: Binary Differencer}

  The binary differencer removes the 0Hz component, and tends to
  emphasize ``edges'' in the waveform:
  \centerline{\includegraphics[width=\textwidth]{exp/differencer_waveforms.png}}
\end{frame}


%%%%%%%%%%%%%%%%%%%%%%%%%%%%%%%%%%%%%%%%%%%%
\section{Conclusions}
\setcounter{subsection}{1}
\begin{frame}
  \frametitle{Conclusions}

  \begin{center}
    \begin{tikzpicture}
      \node[dspnodeopen,dsp/label=right] (y) at (8,0) {$y(t)$};
      \node[dspsquare] (da) at (7,0) {D/A} edge[dspconn](y);
      \node[dspsquare] (h) at (4,0) {$y[n]=h[n]\ast x[n]$};
      \draw (h) -- (da) node[midway,above]{$y[n]$};
      \node[dspsquare] (ad) at (1,0) {A/D};
      \draw (ad) -- (h) node[midway,above]{$x[n]$};
      \node[dspnodeopen,dsp/label=left] (x) at (0,0) {$x(t)$} edge[dspconn](ad);
    \end{tikzpicture}
  \end{center}
  If $x(t)$ is periodic, then $y(t)$ is also periodic with the same
  period but different Fourier Series coefficients:
  \begin{align*}
    x(t)=\sum_{k=-N}^{N}X_ke^{j2\pi kF_0t},~~~~~
    &x[n]=\sum_{k=-N}^{N}X_ke^{jk\omega_0n}\\
    y[n]=\sum_{k=-N}^{N}Y_ke^{jk\omega_0n},~~~~~
    &y(t)=\sum_{k=-N}^{N}Y_ke^{j2\pi kF_0t}
  \end{align*}  
\end{frame}

\begin{frame}
  \frametitle{Conclusions}

  The relationship between the Fourier series coefficients is given by
  the frequency response of the system:
  \begin{displaymath}
    Y = X \sum_m h[m]e^{-j\omega m}
  \end{displaymath}
  \begin{itemize}
  \item A rectangular averager is a low-pass filter: low-frequency
    signals pass through, but high-frequency signals are averaged out.
  \item A binary differencer is a high-pass filter: high-frequency
    signals pass through, but low-frequency signals are differenced
    out, especially the 0Hz component.
  \end{itemize}
\end{frame}

\end{document}
