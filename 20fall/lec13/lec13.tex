\documentclass{beamer}
\usepackage{tikz,amsmath,hyperref,graphicx,stackrel,animate,media9}
\usetikzlibrary{positioning,shadows,arrows,shapes,calc,dsp,chains}
\newcommand{\argmax}{\operatornamewithlimits{argmax}}
\newcommand{\argmin}{\operatornamewithlimits{argmin}}
\mode<presentation>{\usetheme{Frankfurt}}
\AtBeginSection[]
{
  \begin{frame}<beamer>
    \frametitle{Outline}
    \tableofcontents[currentsection,currentsubsection]
  \end{frame}
}
\title{Lecture 13: Block Diagrams and the Inverse Z Transform}
\author{Mark Hasegawa-Johnson}
\date{ECE 401: Signal and Image Analysis, Fall 2020}  
\begin{document}

% Title
\begin{frame}
  \maketitle
\end{frame}

% Title
\begin{frame}
  \tableofcontents
\end{frame}

%%%%%%%%%%%%%%%%%%%%%%%%%%%%%%%%%%%%%%%%%%%%
\section[Review]{Review: FIR and IIR Filters, and System Functions}
\setcounter{subsection}{1}

\begin{frame}
  \frametitle{Review: FIR and IIR Filters}
  \begin{itemize}
  \item An autoregressive filter is also called {\bf infinite impulse response (IIR)},
    because $h[n]$ has infinite length.
  \item A filter with only feedforward coefficients, and no feedback coefficients, is called
    {\bf finite impulse response (FIR)}, because $h[n]$ has finite length (its length is
    just the number of feedforward terms in the difference equation).
  \end{itemize}
\end{frame}
\begin{frame}
  \frametitle{Summary: Poles and Zeros}
  A first-order autoregressive filter,
  \[
  y[n] = x[n]+bx[n-1]+ay[n-1],
  \]
  has the impulse response and transfer function
  \[
  h[n]=a^n u[n]+ba^{n-1}u[n-1] \leftrightarrow H(z)  = \frac{1+bz^{-1}}{1-az^{-1}},
  \]
  where $a$ is called the {\bf pole} of the filter, and $-b$ is called
  its {\bf zero}.
\end{frame}

%%%%%%%%%%%%%%%%%%%%%%%%%%%%%%%%%%%%%%%%%%%%
\section[Block Diagrams]{The System Function and Block Diagrams}
\setcounter{subsection}{1}

\begin{frame}
  \frametitle{Why use block diagrams?}

  A first-order difference equation looks like
  \[
  y[n] = b_0x[n]+b_1x[n-1] +ay[n-1]
  \]
  \begin{itemize}
  \item It's pretty easy to understand what computation is taking
    place in a first-order difference equation.
  \item As we get to higher-order systems, though, the equations for
    implementing them will be kind of complicated.
  \item In order to make the complicated equations very easy, we
    represent the equations using block diagrams.
  \end{itemize}
\end{frame}

\begin{frame}
  \frametitle{Elements of a block diagram}

  A block diagram has just three main element types:
  \begin{enumerate}
  \item {\bf Multiplier:} the following element means
    $y[n]=b_0x[n]$:
    \vspace*{5mm}
    
    \centerline{\begin{tikzpicture}
        \node[dspnodeopen,dsp/label=right] (m02) at (2,0) {$y[n]$};
        \node[dspmixer,dsp/label=above] (m01) at (1,0) {$b_0$} edge[dspflow] (m02);
        \node[dspnodeopen,dsp/label=left] (m00) at (0,0) {$x[n]$} edge[dspflow](m01);
    \end{tikzpicture}}
  \item {\bf Unit Delay:} the following element means $y[n]=x[n-1]$
    (i.e., $Y(z)=z^{-1}X(z)$):
    \vspace*{3mm}
    
    \centerline{\begin{tikzpicture}
        \node[dspnodeopen,dsp/label=right] (m02) at (2,0) {$y[n]$};
        \node[dspsquare] (m01) at (1,0) {$z^{-1}$}  edge[dspflow] (m02);
        \node[dspnodeopen,dsp/label=left] (m00) at (0,0) {$x[n]$} edge[dspflow](m01);
    \end{tikzpicture}}          
  \item {\bf Adder:} the following element means $z[n]=x[n]+y[n]$:
    \vspace*{3mm}
    
    \centerline{\begin{tikzpicture}
        \node[dspnodeopen,dsp/label=right] (m12) at (2,-1) {$z[n]$};
        \node[dspadder](m11) at (1,-1) {}  edge[dspflow] (m12);
        \node[dspnodeopen,dsp/label=left] (m10) at (0,-1) {$y[n]$} edge[dspflow](m11);
        \node[coordinate](m01) at (1,0) {}  edge[dspflow](m11);
        \node[dspnodeopen,dsp/label=left] (m00) at (0,0) {$x[n]$} edge[dspflow](m01);
    \end{tikzpicture}}
  \end{enumerate}
\end{frame}

\begin{frame}
  \frametitle{Example: Time Domain}

  Here's an example of a complete block diagram:
  \centerline{\begin{tikzpicture}
      \node[dspnodeopen,dsp/label=right] (y) at (6,2) {$y[n]$};
      \node[dspsquare] (ydelay) at (5,1) {$z^{-1}$};
      \node[dspadder] (adder) at (3,2) {};
      \node[dspnodefull](ysplit) at (5,2){} edge[dspflow](y) edge[dspconn](ydelay) edge[dspline](adder);
      \node[coordinate] (ycenter) at (3,0) {} edge[dspconn] (adder);
      \node[dspmixer,dsp/label=above] (ymix) at (4,0) {$a$} edge[dspline] (ycenter);
      \node[coordinate] (ycorner)  at (5,0) {} edge[dspconn] (ymix) edge[dspline](ydelay);
      \node[dspnodeopen,dsp/label=left] (x) at (2,2) {$x[n]$} edge[dspconn](adder);
  \end{tikzpicture}}

  This block diagram is equivalent to the following equation:
  \begin{displaymath}
    y[n] = x[n]+ay[n-1]
  \end{displaymath}
  Notice that we can read it, also, as
  \begin{displaymath}
    Y(z)  = X(z) + az^{-1}Y(z)~~~\Rightarrow~~~H(z)=\frac{1}{1-az^{-1}}
  \end{displaymath}
\end{frame}

%\begin{frame}
%  \frametitle{Example: $z$ Domain}
%
%  Here's an example of a complete block diagram, but with the signals written in the $z$ domain:
%  \centerline{\begin{tikzpicture}
%      \node[dspnodeopen,dsp/label=right] (y) at (6,2) {$Y(z)$};
%      \node[dspsquare] (ydelay) at (5,1) {$z^{-1}$};
%      \node[dspadder] (adder) at (3,2) {};
%      \node[dspnodefull](ysplit) at (5,2){} edge[dspflow](y) edge[dspconn](ydelay) edge[dspline](adder);
%      \node[coordinate] (ycenter) at (3,0) {} edge[dspconn] (adder);
%      \node[dspmixer,dsp/label=above] (ymix) at (4,0) {$a$} edge[dspline] (ycenter);
%      \node[coordinate] (ycorner)  at (5,0) {} edge[dspconn] (ymix) edge[dspline](ydelay);
%      \node[dspnodeopen,dsp/label=left] (x) at (2,2) {$X(z)$} edge[dspconn](adder);
%  \end{tikzpicture}}
%
%  It means exactly the same thing as on the previous slide.  It means that
%  \begin{align*}
%    y[n] &= x[n]+ay[n-1]\\
%    Y(z) &= X(z) + az^{-1}Y(z)
%  \end{align*}
%\end{frame}

\begin{frame}
  \frametitle{A Complete First-Order IIR Filter}

  Now consider how we can represent a complete first-order IIR filter, including both
  the pole and the zero.  Here's its system function:
  \begin{displaymath}
    Y(z)  = b_0X(z) +b_1z^{-1}X(z)+ a_1z^{-1}Y(z).
  \end{displaymath}
  When we implement it, we would write a line of python that does this:
  \begin{displaymath}
    y[n] = b_0x[n]+b_1x[n-1]+a_1y[n-1],
  \end{displaymath}
  which is exactly this block diagram:
  \vspace*{3mm}
  
  \centerline{\begin{tikzpicture}
      \node[dspnodeopen,dsp/label=right] (y) at (6,2) {$y[n]$};
      \node[dspsquare] (ydelay) at (5,1) {$z^{-1}$};
      \node[dspsquare] (xdelay) at (1,1) {$z^{-1}$};
      \node[dspadder] (adder) at (3,2) {};
      \node[dspmixer,dsp/label=below] (b0) at (2,2) {$b_0$} edge[dspconn] (adder);
      \node[dspnodefull](ysplit) at (5,2){} edge[dspflow](y) edge[dspconn](ydelay) edge[dspline](adder);
      \node[dspnodefull] (xsplit) at (1,2) {} edge[dspconn] (xdelay) edge[dspline](b0);
      \node[coordinate] (ycenter) at (3.1,0) {} edge[dspconn] (adder);
      \node[coordinate] (xcenter) at (2.9,0) {} edge[dspconn] (adder);
      \node[dspmixer,dsp/label=above] (ymix) at (4,0) {$a_1$} edge[dspline] (ycenter);
      \node[dspmixer,dsp/label=above] (xmix) at (2,0) {$b_1$} edge[dspline] (xcenter);
      \node[coordinate] (ycorner)  at (5,0) {} edge[dspconn] (ymix) edge[dspline](ydelay);
      \node[coordinate] (xcorner)  at (1,0) {} edge[dspconn] (xmix) edge[dspline](xdelay);
      \node[dspnodeopen,dsp/label=left] (x) at (0,2) {$x[n]$} edge[dspline](xsplit);
  \end{tikzpicture}}
\end{frame}

\begin{frame}
  \frametitle{Series and Parallel Combinations}
  Now let's talk about how to combine systems.
  \begin{itemize}
  \item {\bf Series combination}: passing the signal through two
    systems {\bf in series} is like multiplying the system functions:
    \[
    H(z)=H_2(z)H_1(z)
    \]
  \item {\bf Parallel combination}: passing the signal through two
    systems in {\bf parallel}, then adding the outputs, is like adding
    the system functions:
    \[
    H(z) = H_1(z)+H_2(z)
    \]
  \end{itemize}
\end{frame}
  
\begin{frame}
  \frametitle{One Block for Each System}

  Suppose that one of the two systems, $H_1(z)$, looks like this:
  \vspace*{3mm}

  \centerline{\begin{tikzpicture}
      \node[dspnodeopen,dsp/label=right] (y) at (6,2) {$y[n]$};
      \node[dspsquare] (ydelay) at (5,1) {$z^{-1}$};
      \node[dspadder] (adder) at (3,2) {};
      \node[dspnodefull](ysplit) at (5,2){} edge[dspflow](y) edge[dspconn](ydelay) edge[dspline](adder);
      \node[coordinate] (ycenter) at (3,0) {} edge[dspconn] (adder);
      \node[dspmixer,dsp/label=above] (ymix) at (4,0) {$p_1$} edge[dspline] (ycenter);
      \node[coordinate] (ycorner)  at (5,0) {} edge[dspconn] (ymix) edge[dspline](ydelay);
      \node[dspnodeopen,dsp/label=left] (x) at (2,2) {$x[n]$} edge[dspconn](adder);
  \end{tikzpicture}}
  and has the system function
  \begin{displaymath}
    H_1(z) = \frac{1}{1-p_1z^{-1}}
  \end{displaymath}
  Let's represent the whole system using a single box:
  \vspace*{3mm}

  \centerline{\begin{tikzpicture}
      \node[dspnodeopen,dsp/label=right] (m02) at (2,0) {$y[n]$};
      \node[dspsquare] (m01) at (1,0) {$H_1(z)$}  edge[dspflow] (m02);
      \node[dspnodeopen,dsp/label=left] (m00) at (0,0) {$x[n]$} edge[dspflow](m01);
  \end{tikzpicture}}          
\end{frame}

\begin{frame}
  \frametitle{Series Combination}

  The series combination, then, looks like this:
  \vspace*{3mm}
  
  \centerline{\begin{tikzpicture}
      \node[dspnodeopen,dsp/label=right] (y2) at (2,0) {$y_2[n]$};
      \node[dspsquare] (h2) at (1,0) {$H_2(z)$}  edge[dspconn] (y2);
      \node[dspnodefull,dsp/label=above] (y1) at (0,0) {$y_1[n]$} edge[dspconn] (h2);
      \node[dspsquare] (h1) at (-1,0) {$H_1(z)$}  edge[dspline] (y1);
      \node[dspnodeopen,dsp/label=left] (x) at (-2,0) {$x[n]$} edge[dspconn] (h1);
  \end{tikzpicture}}
  This means that
  \[
  Y_2(z) = H_2(z)Y_1(z) = H_2(z)H_1(z)X(z)
  \]
  and therefore
  \[
  H(z) = \frac{Y(z)}{X(z)} = H_1(z)H_2(z)
  \]
\end{frame}

\begin{frame}
  \frametitle{Series Combination}

  The series combination, then, looks like this:
  \vspace*{3mm}

  \centerline{\begin{tikzpicture}
      \node[dspnodeopen,dsp/label=right] (y2) at (2,0) {$y_2[n]$};
      \node[dspsquare] (h2) at (1,0) {$H_2(z)$}  edge[dspconn] (y2);
      \node[coordinate,dsp/label=above] (y1) at (0,0) {$y_1[n]$} edge[dspconn] (h2);
      \node[dspsquare] (h1) at (-1,0) {$H_1(z)$}  edge[dspline] (y1);
      \node[dspnodeopen,dsp/label=left] (x) at (-2,0) {$x[n]$} edge[dspconn] (h1);
  \end{tikzpicture}}
  Suppose that we know each of the systems separately:
  \[
  H_1(z)=\frac{1}{1-p_1z^{-1}},~~~~~
  H_2(z)=\frac{1}{1-p_2z^{-1}}
  \]
  Then, to get $H(z)$, we just  have to multiply:
  \[
  H(z) = \frac{1}{(1-p_1z^{-1})(1-p_2z^{-1})} =
  \frac{1}{1-(p_1+p_2)z^{-1}+p_1p_2z^{-2}}
  \]
\end{frame}

\begin{frame}
  \frametitle{Parallel Combination}

  Parallel combination of two systems looks like this:
  \vspace*{3mm}

  \centerline{\begin{tikzpicture}
      \node[dspnodeopen,dsp/label=right] (y) at (2,0) {$y[n]$};
      \node[dspadder] (adder) at (1,0) {}  edge[dspflow] (y);
      \node[coordinate] (y1) at (1,1) {}  edge[dspline] (adder);
      \node[coordinate] (y2) at (1,-1) {}  edge[dspline] (adder);
      \node[dspsquare] (h1) at (0,1) {$H_1(z)$}  edge[dspline] (y1);
      \node[dspsquare] (h2) at (0,-1) {$H_2(z)$}  edge[dspline] (y2);
      \node[coordinate] (x1) at (-1,1) {}  edge[dspconn] (h1);
      \node[coordinate] (x2) at (-1,-1) {}  edge[dspconn] (h2);
      \node[dspnodefull] (xsplit) at (-1,0) {} edge[dspline](x1) edge[dspline](x2);
      \node[dspnodeopen,dsp/label=left] (x) at (-2,0) {$x[n]$} edge[dspline] (xsplit);
  \end{tikzpicture}}
  This means that
  \[
  Y(z) = H_1(z)X(z)+H_2(z)X(z)
  \]
  and therefore
  \[
  H(z) = \frac{Y(z)}{X(z)} = H_1(z)  + H_2(z)
  \]
\end{frame}

\begin{frame}
  \frametitle{Parallel Combination}

  Parallel combination of two systems looks like this:
  \vspace*{3mm}

  \centerline{\begin{tikzpicture}
      \node[dspnodeopen,dsp/label=right] (y) at (2,0) {$y[n]$};
      \node[dspadder] (adder) at (1,0) {}  edge[dspflow] (y);
      \node[coordinate] (y1) at (1,1) {}  edge[dspline] (adder);
      \node[coordinate] (y2) at (1,-1) {}  edge[dspline] (adder);
      \node[dspsquare] (h1) at (0,1) {$H_1(z)$}  edge[dspline] (y1);
      \node[dspsquare] (h2) at (0,-1) {$H_2(z)$}  edge[dspline] (y2);
      \node[coordinate] (x1) at (-1,1) {}  edge[dspconn] (h1);
      \node[coordinate] (x2) at (-1,-1) {}  edge[dspconn] (h2);
      \node[dspnodefull] (xsplit) at (-1,0) {} edge[dspline](x1) edge[dspline](x2);
      \node[dspnodeopen,dsp/label=left] (x) at (-2,0) {$x[n]$} edge[dspline] (xsplit);
  \end{tikzpicture}}
  Suppose that we know each of the systems separately:
  \[
  H_1(z)=\frac{1}{1-p_1z^{-1}},~~~~~
  H_2(z)=\frac{1}{1-p_2z^{-1}}
  \]
  Then, to get $H(z)$, we just  have to add:
  \[
  H(z) = \frac{1}{1-p_1z^{-1}}+\frac{1}{1-p_2z^{-1}}
  \]
\end{frame}

\begin{frame}
  \frametitle{Parallel Combination}

  Parallel combination of two systems looks like this:
  \vspace*{3mm}

  \centerline{\begin{tikzpicture}
      \node[dspnodeopen,dsp/label=above] (y) at (2,0) {$y[n]$};
      \node[dspadder] (adder) at (1,0) {}  edge[dspflow] (y);
      \node[coordinate] (y1) at (1,1) {}  edge[dspline] (adder);
      \node[coordinate] (y2) at (1,-1) {}  edge[dspline] (adder);
      \node[dspsquare] (h1) at (0,1) {$H_1(z)$}  edge[dspline] (y1);
      \node[dspsquare] (h2) at (0,-1) {$H_2(z)$}  edge[dspline] (y2);
      \node[coordinate] (x1) at (-1,1) {}  edge[dspconn] (h1);
      \node[coordinate] (x2) at (-1,-1) {}  edge[dspconn] (h2);
      \node[dspnodefull] (xsplit) at (-1,0) {} edge[dspline](x1) edge[dspline](x2);
      \node[dspnodeopen,dsp/label=above] (x) at (-2,0) {$x[n]$} edge[dspline](xsplit);
  \end{tikzpicture}}
  \begin{align*}
  H(z) &= \frac{1}{1-p_1z^{-1}}+\frac{1}{1-p_2z^{-1}}\\
  &= \frac{1-p_2z^{-1}}{(1-p_1z^{-1})(1-p_2z^{-1})}+\frac{1-p_1z^{-1}}{(1-p_1z^{-1})(1-p_2z^{-1})}\\
  &= \frac{2-(p_1+p_2)z^{-1}}{1-(p_1+p_2)z^{-1}+p_1p_2z^{-2}}
  \end{align*}
\end{frame}

%%%%%%%%%%%%%%%%%%%%%%%%%%%%%%%%%%%%%%%%%%%%
\section[Inverse Z]{Inverse Z Transform}
\setcounter{subsection}{1}

\begin{frame}
  \frametitle{Inverse Z transform}

  Suppose you know $H(z)$, and you want to find $h[n]$.  How can you
  do that?
\end{frame}
      
\begin{frame}
  \frametitle{How to find the inverse Z transform}

  Any IIR filter $H(z)$ can be written as\ldots
  \begin{itemize}
  \item a {\bf sum} of {\bf exponential} terms, each with this form:
    \begin{displaymath}
      G_\ell(z)=\frac{1}{1-az^{-1}}~~~\leftrightarrow~~~g_\ell[n]= a^nu[n],
    \end{displaymath}
  \item each possibly {\bf multiplied} by a {\bf delay} term, like this one:
    \begin{displaymath}
      D_k(z)=b_kz^{-k}~~~\leftrightarrow~~~d_k[n]=b_k\delta[n-k].
    \end{displaymath}
  \end{itemize}
\end{frame}

\begin{frame}
  \frametitle{Step \#1: The Products}

  Consider one that you already know:
  \begin{displaymath}
    H(z)=\frac{1+bz^{-1}}{1-az^{-1}}
    =\left(\frac{1}{1-az^{-1}}\right)+bz^{-1}\left(\frac{1}{1-az^{-1}}\right)
  \end{displaymath}
  and therefore
  \begin{displaymath}
    h[n] = \left(a^nu[n]\right) + b\left(a^{n-1}u[n-1]\right)
  \end{displaymath}
\end{frame}

\begin{frame}
  \frametitle{Step \#1: The Products}

  So here is the inverse transform of $H(z)=\frac{1+0.5z^{-1}}{1-0.85z^{-1}}$:
  \centerline{\includegraphics[width=4.5in]{exp/numsum.png}}
\end{frame}

\begin{frame}
  \frametitle{Step \#1: The Products}

  In general, if 
  \begin{displaymath}
    G(z) = \frac{1}{A(z)}
  \end{displaymath}
  for any polynomial $A(z)$, and
  \begin{displaymath}
    H(z) = \frac{\sum_{k=0}^M b_kz^{-k}}{A(z)}
  \end{displaymath}
  then
  \begin{displaymath}
    h[n] = b_0 g[n]+b_1g[n-1]+\cdots+b_M g[n-M]
  \end{displaymath}
\end{frame}

\begin{frame}
  \frametitle{Step \#2: The Sum}

  Now we need to figure out the inverse transform of
  \begin{displaymath}
    G(z) = \frac{1}{A(z)}
  \end{displaymath}
\end{frame}

\begin{frame}
  \frametitle{Step \#2: The Sum}
  The method is this:
  \begin{enumerate}
  \item Factor $A(z)$:
    \begin{displaymath}
      G(z) = \frac{1}{\prod_{\ell=1}^N \left(1-p_\ell z^{-1}\right)}
    \end{displaymath}
  \item Assume that $G(z)$ is the result of a parallel system
    combination:
    \begin{displaymath}
      G(z) = \frac{C_1}{1-p_1z^{-1}} + \frac{C_2}{1-p_2z^{-1}} + \cdots
    \end{displaymath}
  \item Find the constants, $C_\ell$, that make the equation true.
  \end{enumerate}
\end{frame}

\begin{frame}
  \frametitle{Example}
  Step \# 1:  Factor it:
  \begin{displaymath}
    \frac{1}{1-1.2z^{-1}+0.72z^{-2}}=
    \frac{1}{\left(1-(0.6+j0.6)z^{-1}\right)\left(1-(0.6-j0.6)z^{-1}\right)}
  \end{displaymath}
  Step \#2: Express it as a  sum:
  \begin{displaymath}
    \frac{1}{1-1.2z^{-1}+0.72z^{-2}}=
    \frac{C_1}{1-(0.6+j0.6)z^{-1}}+\frac{C_2}{1-(0.6-j0.6)z^{-1}}
  \end{displaymath}
  Step \#3: Find the constants.  The algebra is annoying, but it turns out that:
  \begin{displaymath}
    C_1=\frac{1}{2}-j\frac{1}{2},~~~
    C_2=\frac{1}{2}+j\frac{1}{2}
  \end{displaymath}
\end{frame}

\begin{frame}
  \frametitle{Example: All Done!}
  The system function is:
  \begin{align*}
    G(z) &= \frac{1}{1-1.2z^{-1}+0.72z^{-2}}\\
    &=\frac{0.5-0.5j}{1-(0.6+j0.6)z^{-1}}+\frac{0.5+0.5j}{1-(0.6-j0.6)z^{-1}}
  \end{align*}
  and therefore the impulse response is:
  \begin{align*}
    g[n] &= (0.5-0.5j)(0.6+0.6j)^nu[n]+(0.5+0.5j)(0.6-j0.6)^nu[n]\\
    &= \left(0.5\sqrt{2}e^{-j\frac{\pi}{4}}\left(0.6\sqrt{2}e^{j\frac{\pi}{4}}\right)^n+
    0.5\sqrt{2}e^{j\frac{\pi}{4}}\left(0.6\sqrt{2}e^{-j\frac{\pi}{4}}\right)^n\right)u[n]\\
    &= \sqrt{2}(0.6\sqrt{2})^n \cos\left(\frac{\pi}{4}(n-1)\right)u[n]
  \end{align*}
\end{frame}

\begin{frame}
  \centerline{\includegraphics[width=4.5in]{exp/densum.png}}
\end{frame}

\begin{frame}
  \frametitle{How to find the inverse Z transform}

  Any IIR filter $H(z)$ can be written as\ldots
  \begin{itemize}
  \item a {\bf sum} of {\bf exponential} terms, each with this form:
    \begin{displaymath}
      G_\ell(z)=\frac{1}{1-az^{-1}}~~~\leftrightarrow~~~g_\ell[n]= a^nu[n],
    \end{displaymath}
  \item each possibly {\bf multiplied} by a {\bf delay} term, like this one:
    \begin{displaymath}
      D_k(z)=b_kz^{-k}~~~\leftrightarrow~~~d_k[n]=b_k\delta[n-k].
    \end{displaymath}
  \end{itemize}
\end{frame}

%%%%%%%%%%%%%%%%%%%%%%%%%%%%%%%%%%%%%%%%%%%%
\section[Summary]{Summary}
\setcounter{subsection}{1}

\begin{frame}
  \frametitle{Summary: Block Diagrams}
  \begin{itemize}
  \item A {\bf block diagram} shows the delays, additions, and
    multiplications necessary to compute output from input.
  \item {\bf Series combination}: passing the signal through two
    systems {\bf in series} is like multiplying the system functions:
    \[
    H(z)=H_2(z)H_1(z)
    \]
  \item {\bf Parallel combination}: passing the signal through two
    systems in {\bf parallel}, then adding the outputs, is like adding
    the system functions:
    \[
    H(z) = H_1(z)+H_2(z)
    \]
  \end{itemize}
\end{frame}
  
\begin{frame}
  \frametitle{Summary: Inverse Z Transform}

  Any IIR filter $H(z)$ can be written as\ldots
  \begin{itemize}
  \item a {\bf sum} of {\bf exponential} terms, each with this form:
    \begin{displaymath}
      G_\ell(z)=\frac{1}{1-az^{-1}}~~~\leftrightarrow~~~g_\ell[n]= a^nu[n],
    \end{displaymath}
  \item each possibly {\bf multiplied} by a {\bf delay} term, like this one:
    \begin{displaymath}
      D_k(z)=b_kz^{-k}~~~\leftrightarrow~~~d_k[n]=b_k\delta[n-k].
    \end{displaymath}
  \end{itemize}
\end{frame}

\begin{frame}
  \frametitle{Next Time}

  Next time:
  \begin{itemize}
  \item How to  design second-order notch filters, to get rid of 60Hz line noise, and\ldots
  \item more about the frequency response and impulse response of second-order filters.
  \end{itemize}
\end{frame}


\end{document}
