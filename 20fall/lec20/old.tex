\begin{frame}
  \frametitle{Use Parseval's Theorem!}

  In order to turn the convolutions into multiplications, let's use
  Parseval's theorem!
  \begin{align*}
    {\mathcal E}&=\sum_{n=-\infty}^\infty \left(s[n]-y[n]\right)^2\\
    &=\frac{1}{2\pi}\int_{-\pi}^\pi \left|S(\omega)-H(\omega)X(\omega)\right|^2 d\omega\\
    &=\frac{1}{2\pi}\int_{-\pi}^\pi \left(S(\omega)S^*(\omega)-H(\omega)X(\omega)S^*(\omega)\right.\\
    &\left.-S(\omega)H^*(\omega)X^*(\omega)+H(\omega)X(\omega)X^*(\omega)H^*(\omega)\right)d\omega
  \end{align*}
  Now let's try to find the minimum, by setting
  \begin{align*}
    \frac{d{\mathcal E}}{dH(\omega)}=&0
  \end{align*}
  What does that become?  Let's find out on the next slide!
\end{frame}
  

\begin{frame}
  \frametitle{Differentiate and Solve!}
  
  Differentiating by $H(\omega)$ (and pretending that $H^*(\omega)$
  stays constant---which works, even though the reasons it works are
  way too complicated to talk about), we get
  \begin{displaymath}
    \frac{d{\mathcal E}}{dH(\omega)}
    =\frac{1}{2\pi}\int_{-\pi}^\pi \left(-X(\omega)S^*(\omega)+
    X(\omega)X^*(\omega)H^*(\omega)\right)d\omega
  \end{displaymath}
  So we can set $\frac{d{\mathcal E}}{dH(\omega)}=0$ if we choose
  \begin{align*}
    H^*(\omega)&=\frac{X(\omega)S^*(\omega)}{|X(\omega)|^2}
  \end{align*}
  or, equivalently, 
  \begin{align*}
    H(\omega)&=\frac{S(\omega)X^*(\omega)}{|X(\omega)|^2}
  \end{align*}
\end{frame}

