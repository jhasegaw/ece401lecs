\documentclass{beamer}
\usepackage{tikz,amsmath,hyperref,graphicx,stackrel,animate,amssymb}
\usetikzlibrary{positioning,shadows,arrows,shapes,calc}
\newcommand{\argmax}{\operatornamewithlimits{argmax}}
\newcommand{\argmin}{\operatornamewithlimits{argmin}}
\mode<presentation>{\usetheme{Frankfurt}}
\AtBeginSection[]
{
  \begin{frame}<beamer>
    \frametitle{Outline}
    \tableofcontents[currentsection,currentsubsection]
  \end{frame}
}
\title{Lecture 10: Exam 1 Sample Problems}
\author{Mark Hasegawa-Johnson\\All content~\href{https://creativecommons.org/licenses/by-sa/4.0/}{CC-SA 4.0} unless otherwise specified.}
\date{ECE 401: Signal and Image Analysis, Fall 2021}  
\begin{document}

% Title
\begin{frame}
  \maketitle
\end{frame}

% Title
\begin{frame}
  \tableofcontents
\end{frame}

%%%%%%%%%%%%%%%%%%%%%%%%%%%%%%%%%%%%%%%%%%%%
\section[11x1p1]{Fall 2011 Exam 1 Problem 1}
\setcounter{subsection}{1}

\begin{frame}
  \frametitle{Question}

  Calculate the Fourier series coefficients $X_0$ and $X_k$ for the
  periodic signal $x(t)=x(t+8)$:
  \[
  x(t)=\left\{\begin{array}{ll}
  1,& 0\le t < 1\\
  -1,& 1\le t\le 3\\
  0,& 3<t<8 \\
  \end{array}\right.
  \]
\end{frame}

\begin{frame}
  \frametitle{Answer Part 1}

  Calculate the Fourier series coefficients $X_0$ and $X_k$ for the
  periodic signal $x(t)=x(t+8)$:
  \begin{align*}
    X_0 &= \frac{1}{8}\int_0^8 x(t)dt\\
    &= \frac{1}{8}\left(\int_0^1 dt-\int_1^3 dt\right)\\
    &=-\frac{1}{8}
  \end{align*}
\end{frame}

\begin{frame}
  \frametitle{Answer Part 2}

  Calculate the Fourier series coefficients $X_0$ and $X_k$ for the
  periodic signal $x(t)=x(t+8)$:
  \begin{align*}
    X_k &= \frac{1}{8}\int_0^8 x(t)e^{-j2\pi kt/8}dt\\
    &= \frac{1}{8}\left(\int_0^1 e^{-j2\pi kt/8}dt-\int_1^3 e^{-j2\pi kt/8}dt\right)\\
    &=\frac{1}{8}\left(\frac{1}{-j2\pi k/8}\right)\left(\left[e^{-j2\pi kt/8}\right]_0^1-\left[e^{-j2\pi kt/8}\right]_1^3\right)\\
    &=\left(\frac{1}{-j2\pi k}\right)\left(2e^{-j2\pi k/8}-1-e^{-j6\pi k/8}\right)
  \end{align*}
\end{frame}

%%%%%%%%%%%%%%%%%%%%%%%%%%%%%%%%%%%%%%%%%%%%
\section[11x1p4a]{Fall 2011 Exam 1 Problem 4(a)}
\setcounter{subsection}{1}

\begin{frame}
  \begin{block}{Question}
  Suppose that we have a signal bandlimited to 5kHz.  
  What is the minimum $F_s$ necessary to avoid aliasing?
  \end{block}
  \begin{block}{Answer}
    10kHz
  \end{block}
\end{frame}

%%%%%%%%%%%%%%%%%%%%%%%%%%%%%%%%%%%%%%%%%%%%
\section[11x1p5b]{Fall 2011 Exam 1 Problem 5(b)}
\setcounter{subsection}{1}

\begin{frame}
  \begin{block}{Question}
    Assume that $x[n]=x_c(nT)$, where $1/T=10,000$
    samples/second. Find $x[n]$ and its spectrum if
    \[
    x_c(t) = \cos(7000\pi t)
    \]
  \end{block}
  \begin{block}{Answer}
    \[
    x[n] = \cos\left(\frac{7000\pi n}{10,000}\right)
    \]
    \ldots and the spectrum is
    \[
    \left\{(-\frac{7000\pi}{10000},\frac{1}{2}),(\frac{7000\pi}{10000},\frac{1}{2})\right\}
    \]
  \end{block}
\end{frame}

%%%%%%%%%%%%%%%%%%%%%%%%%%%%%%%%%%%%%%%%%%%%
\section[11x3p6]{Fall 2011 Exam 3 Problem 6(a-e)}
\setcounter{subsection}{1}

\begin{frame}
  \begin{block}{Part (a)}
    Consider the signal $x(t)=-2+\sin(40\pi t)$.
    Determine and list all of the analog
    frequencies in the signal $x(t)$.  Include negative frequencies.
  \end{block}
  \begin{block}{Answer}
    \{-20,0,20\}
  \end{block}
\end{frame}

\begin{frame}
  \begin{block}{Part (b)}
    $x(t)=-2+\sin(40\pi t)$.
    What is the lowest possible sampling frequency
    that would avoid aliasing?
  \end{block}
  \begin{block}{Answer}
    $F_s > 2f = 40$
  \end{block}
\end{frame}

\begin{frame}
  \begin{block}{Part (c)}
    What is the corresponding Nyquist frequency
    for the sampling rate you found in part (b)?
  \end{block}
  \begin{block}{Answer}
    $F_N = \frac{F_s}{2} > 20$
  \end{block}
\end{frame}

\begin{frame}
  \begin{block}{Part (d)}
    $x(t)=-2+\sin(40\pi t)$.
    For a sampling frequency of $F_s=100$Hz, find
    $x[n]$.
  \end{block}
  \begin{block}{Answer}
    \[
    x[n] = -2 + \sin\left(\frac{40\pi n}{100}\right)
    \]
  \end{block}
\end{frame}

\begin{frame}
  \begin{block}{Part (e)}
    $x(t)=-2+\sin(40\pi t)$,  $F_s=100$Hz.
    Determine and list all of the frequencies
    $\omega$, $-\pi<\omega\le\pi$, present in the discrete-time signal
    $x[n]$.  Include negative frequencies.
  \end{block}
  \begin{block}{Answer}
    \[
    \left\{ -\frac{40\pi}{100},0,\frac{40\pi}{100}\right\}
    \]
  \end{block}
\end{frame}

%%%%%%%%%%%%%%%%%%%%%%%%%%%%%%%%%%%%%%%%%%%%
\section[13x1p1]{Fall 2013 Exam 1 Problem 1}
\setcounter{subsection}{1}

\begin{frame}
  \frametitle{Question}
  \[
  \cos(\omega t)+\cos(\omega t+\frac{\pi}{3})=m\cos(\omega t+\theta)
  \]
  Find $x$ and $y$ such that $m=\sqrt{x^2+y^2}$ and
  $\theta=\mbox{atan2}(x,y)$, the two-argument arctangent of $x$ and $y$.
\end{frame}

\begin{frame}
  \frametitle{Answer}
  \begin{align*}
    \cos(\omega t)+\cos(\omega t+\frac{\pi}{3}) &= \Re\left\{(1+e^{j\pi/3})e^{j\omega t}\right\}\\
    &= \Re\left\{(1+\cos(\pi/3)+j\sin(\pi/3)))e^{j\omega t}\right\}\\
  \end{align*}
  So
  \[
  x=1+\cos(\pi/3),~~~y=\sin(\pi/3)
  \]
\end{frame}

%%%%%%%%%%%%%%%%%%%%%%%%%%%%%%%%%%%%%%%%%%%%
\section[13x1p2]{Fall 2013 Exam 1 Problem 2}
\setcounter{subsection}{1}

\begin{frame}
  \frametitle{Question}
  A signal $x(t)=\cos(2\pi 6000t)$ is sampled at $F_s=8000$
  samples/second to create $y[n]$.  The digital signal $y[n]$ is then
  played back through an ideal D/A at the same sampling rate, $F_s=8000$
  samples/second, to generate a signal $z(t)$.  Find $z(t)$.
\end{frame}

\begin{frame}
  \frametitle{Answer}
  \begin{align*}
    x(t) &=\cos(2\pi 6000t)\\
    y[n] &=\cos\left(\frac{2\pi 6000n}{8000}\right)\\
    &=\cos\left(\frac{3\pi n}{2}\right)\\
    &=\cos\left(\frac{\pi n}{2}\right)\\
    z(t) &=\cos\left(\frac{\pi}{2}8000t\right)=\cos(4000\pi t)
  \end{align*}
\end{frame}

%%%%%%%%%%%%%%%%%%%%%%%%%%%%%%%%%%%%%%%%%%%%
\section[13x1p3]{Fall 2013 Exam 1 Problem 3}
\setcounter{subsection}{1}

\begin{frame}
  \frametitle{Question}
  The signal $x[n]$ is periodic with period $N_0=4$.  Its values in each
  period are
  \[
  x[n]=\left\{\begin{array}{ll}
  1 & n=0\\
  -1 & n=1,2,3\\
  \end{array}\right.
  \]
  Find the Fourier series coefficients.
\end{frame}

\begin{frame}
  \frametitle{Answer}
  \begin{align*}
    X_k &= \frac{1}{4}\sum_{n=0}^3 x[n]e^{-j2\pi kn/4}\\
    &= \frac{1}{4}\sum_{n=0}^3 x[n]e^{-j\pi kn/2}\\
    &= \frac{1}{4}\left(1 - e^{-j\pi k/2}-e^{-j\pi k}-e^{-j\pi 3k/2}\right)
  \end{align*}
\end{frame}

%%%%%%%%%%%%%%%%%%%%%%%%%%%%%%%%%%%%%%%%%%%%
\section[13x3p1]{Fall 2013 Exam 3 Problem 1}
\setcounter{subsection}{1}

\begin{frame}
  \frametitle{Question}
  \begin{align*}
    &6\cos\left(2\pi 1000\left(t-\frac{1}{4000}\right)\right)+6\sin\left(2\pi 1000\left(t-\frac{1}{4000}\right)\right)\\
    &=A\cos(\Omega t+\phi)
  \end{align*}
  Find $A$, $\Omega$, and $\phi$.
\end{frame}


\begin{frame}
  \frametitle{Answer}
  \begin{align*}
    &
    6\cos\left(2\pi 1000\left(t-\frac{1}{4000}\right)\right)+6\sin\left(2\pi 1000\left(t-\frac{1}{4000}\right)\right)\\
    &=6\cos\left(2\pi 1000\left(t-\frac{1}{4000}\right)\right)+6\cos\left(2\pi 1000\left(t-\frac{1}{4000}\right)-\frac{\pi}{2}\right)\\
    &=
    6\cos\left(2\pi 1000t-\frac{\pi}{2}\right)+6\cos\left(2\pi 1000t-\frac{\pi}{2}-\frac{\pi}{2}\right)\\
    &=\Re\left\{6(e^{-j\pi/2}+e^{-j\pi})e^{j2000\pi t}\right\}\\
    &=\Re\left\{6(-j-1)e^{j2000\pi t}\right\}\\
    &=\Re\left\{6\sqrt{2}e^{-j3\pi/4}e^{j2000\pi t}\right\}
  \end{align*}
  So $A=6\sqrt{2}$, $\Omega=2000\pi$, $\phi=-\frac{3\pi}{4}$.
\end{frame}

%%%%%%%%%%%%%%%%%%%%%%%%%%%%%%%%%%%%%%%%%%%%
\section[13x3p2]{Fall 2013 Exam 3 Problem 2}
\setcounter{subsection}{1}

\begin{frame}
  \frametitle{Question}
  A periodic signal $x(t)$, with period $T_0$, is given by
  \[
  x(t)=\left\{\begin{array}{ll}1 &0\le t\le \frac{3T_0}{4}\\
  0 & \frac{3T_0}{4}<t<T_0\end{array}\right.
  \]
  The same signal can be expressed as a Fourier series:
  \[
  x(t)=\sum_{k=-\infty}^\infty X_ke^{j2\pi kt/T_0}
  \]
  Find $|X_2|$, the amplitude of the second harmonic.
\end{frame}


\begin{frame}
  \frametitle{Answer}
  \begin{align*}
    X_2 &= \frac{1}{T_0}\int_0^{T_0}x(t)e^{-j2\pi 2t/T_0}dt\\
    &= \frac{1}{T_0}\int_0^{3T_0/4}e^{-j4\pi t/T_0}dt\\
    &= \frac{1}{T_0}\left(\frac{1}{-j4\pi/T_0}\right)\left[e^{-j4\pi t/T_0}\right]_0^{3T_0/4}\\
    &= \left(\frac{1}{-j4\pi}\right)\left(e^{-j3\pi}-1\right)
    = \left(\frac{-2}{-j4\pi}\right)
  \end{align*}
  So $|X_2|=1/2\pi$.
\end{frame}

%%%%%%%%%%%%%%%%%%%%%%%%%%%%%%%%%%%%%%%%%%%%
\section[13x3p8]{Fall 2013 Exam 3 Problem 8}
\setcounter{subsection}{1}

\begin{frame}
  \begin{block}{Question}
  An 8000Hz tone, $x(t)=\cos(2\pi 8000t)$, is sampled at
  $F_s=\frac{1}{T}=10,000$ samples/second in order to create
  $x[n]=x(nT)$.  Sketch $X(\omega)$ for $0\le\omega\le 2\pi$ ({\bf note
    the domain!!}). Specify the frequencies at which $X(\omega)\ne 0$.
  \end{block}
  \begin{block}{Answer}
    Answer should be a spectrum plot with spikes at $\omega=8\pi/5$ and
    $\omega=2\pi/5$, each labeled with a phasor of $1/2$.
  \end{block}
\end{frame}

%%%%%%%%%%%%%%%%%%%%%%%%%%%%%%%%%%%%%%%%%%%%
\section[14x1p1]{Fall 2014 Exam 1 Problem 1}
\setcounter{subsection}{1}

\begin{frame}
  \begin{block}{Question, Part 1}
  Each of the following is sampled at $F_s=10000$ samples/second,
  producing either $x[n]=$constant, or $x[n]=\cos\omega n$ for some
  value of $\omega$.  Specify the constant if possible; otherwise,
  specify $\omega$ such that $-\pi\le\omega <\pi$.
  \[x(t)=\cos\left(2\pi 900t\right)\]
  \end{block}
  \begin{block}{Answer}
    {\bf Solution:}~~$ \omega = \frac{1800\pi}{10,000}$
  \end{block}
\end{frame}


\begin{frame}
  \begin{block}{Question, Part 2}
  Each of the following is sampled at $F_s=10000$ samples/second,
  producing either $x[n]=$constant, or $x[n]=\cos\omega n$ for some
  value of $\omega$.  Specify the constant if possible; otherwise,
  specify $\omega$ such that $-\pi\le\omega <\pi$.
  \[x(t)=\cos\left(2\pi 10000t\right)\]
  \end{block}
  \begin{block}{Answer}
    {\bf Solution:}~~$ x[n]=1$
  \end{block}
\end{frame}
\begin{frame}
  \begin{block}{Question, Part 3}
  Each of the following is sampled at $F_s=10000$ samples/second,
  producing either $x[n]=$constant, or $x[n]=\cos\omega n$ for some
  value of $\omega$.  Specify the constant if possible; otherwise,
  specify $\omega$ such that $-\pi\le\omega <\pi$.
  \[x(t)=\cos\left(2\pi 11000t\right)\]
  \end{block}
  \begin{block}{Answer}
    {\bf Solution:}~~$ \omega = \frac{22000\pi}{10000}-2\pi =\frac{2000\pi}{10000}$
  \end{block}
\end{frame}

%%%%%%%%%%%%%%%%%%%%%%%%%%%%%%%%%%%%%%%%%%%%
\section[14x1p2]{Fall 2014 Exam 1 Problem 2}
\setcounter{subsection}{1}

\begin{frame}
  \frametitle{Question}
  Consider the signal 
  \[
  x(t) = 2\cos\left(2\pi 440t\right)-3\sin\left(2\pi 440t\right)
  \]
  This signal can also be written as $x(t)=A\cos\left(\omega
  t+\theta\right)$ for some $A=\sqrt{M}$, $\omega$, and
  $\theta=\mbox{atan}(R)$.  Find $M$, $\omega$, and $R$.
\end{frame}


\begin{frame}
  \frametitle{Answer}
  \begin{align*}
    x(t) &= 2\cos\left(2\pi 440t\right)-3\sin\left(2\pi 440t\right)\\
    &= 2\cos\left(2\pi 440t\right)-3\cos\left(2\pi 440t-\frac{\pi}{2}\right)\\
    &= \Re\left\{(2-3e^{-j\pi/2})e^{j2\pi 440t}\right\}\\
    &= \Re\left\{(2+3j)e^{j2\pi 440t}\right\}\\
    &= \Re\left\{\sqrt{5}e^{j\mbox{atan}(3/2)}e^{j2\pi 440t}\right\}
  \end{align*}
  So $A=\sqrt{13}$, $\omega=2\pi 440$, and $\theta=\mbox{atan}(3/2)$.
\end{frame}

%%%%%%%%%%%%%%%%%%%%%%%%%%%%%%%%%%%%%%%%%%%%
\section[14x1p3]{Fall 2014 Exam 1 Problem 3}
\setcounter{subsection}{1}

\begin{frame}
  \begin{block}{Question, Part 1}
  A signal $x(t)$ is periodic with $T_0=0.02$ seconds, and its values
  are specified by
  \[
  x(t)=\left\{\begin{array}{ll}
  -1 & 0\le t\le 0.01\\
  0 & 0.01<t<0.02
  \end{array}\right.
  \]
  Sketch $x(t)$ as a function of $t$ for $0\le t\le 0.02$ seconds.
  Label at least one important tic mark, each, on the horizontal and
  vertical axes.
  \end{block}
  \begin{block}{Answer}
    Sketch should show $x(t)=-1$ between 0 and 0.01, then $x(t)=0$ between 0.01 and 0.02.
  \end{block}
\end{frame}

\begin{frame}
  \begin{block}{Question, Part 2}
    A signal $x(t)$ is periodic with $T_0=0.02$ seconds, and its values
    are specified by
    \[
    x(t)=\left\{\begin{array}{ll}
    -1 & 0\le t\le 0.01\\
    0 & 0.01<t<0.02
    \end{array}\right.
    \]
    What is $F_0$?
  \end{block}
  \begin{block}{Answer}
    \[
    F_0 = \frac{1}{T_0} = \frac{1}{0.02}
    \]
  \end{block}
\end{frame}

\begin{frame}
  \begin{block}{Question, Part 3}
    A signal $x(t)$ is periodic with $T_0=0.02$ seconds, and its values
    are specified by
    \[
    x(t)=\left\{\begin{array}{ll}
    -1 & 0\le t\le 0.01\\
    0 & 0.01<t<0.02
    \end{array}\right.
    \]
    Find $X_0$ without doing any integral.
  \end{block}
  \begin{block}{Answer}
    $x(t)$ is -1 for half a period, and 0 for half a period, so its average value is $X_0=-\frac{1}{2}$.
  \end{block}
\end{frame}

\begin{frame}
  \frametitle{Question, Part 3}
    A signal $x(t)$ is periodic with $T_0=0.02$ seconds, and its values
    are specified by
    \[
    x(t)=\left\{\begin{array}{ll}
    -1 & 0\le t\le 0.01\\
    0 & 0.01<t<0.02
    \end{array}\right.
    \]
    Find $X_k$ for all the other values of $k$, i.e., for $k\ne 0$.
    Simplify; your answer should have no exponentials in it.
\end{frame}


\begin{frame}
  \frametitle{Answer}
  \begin{align*}
    X_k &= \frac{1}{0.02}\int_0^{0.02} x(t) e^{-j2\pi kt/0.02}dt\\
    &= \frac{1}{0.02}\int_0^{0.01} e^{-j2\pi kt/0.02}dt\\
    &= \frac{1}{0.02}\left(\frac{1}{-j2\pi k/0.02}\right)\left[e^{-j2\pi kt/0.02}\right]_0^{0.01}\\
    &= \left(\frac{1}{-j2\pi k}\right)\left(e^{-j2\pi k0.01/0.02}-1\right)\\
    &= \left(\frac{1}{-j2\pi k}\right)\left(e^{-j\pi k}-1\right)\\
    &= \left(\frac{1}{-j2\pi k}\right)\left((-1)^k-1\right)
  \end{align*}
\end{frame}

%%%%%%%%%%%%%%%%%%%%%%%%%%%%%%%%%%%%%%%%%%%%
\section[14x3p3]{Fall 2014 Exam 3 Problem 3}
\setcounter{subsection}{1}

\begin{frame}
  \frametitle{Question}
  In order to become a billionaire, you've decided you need to know what
  was the total value of the U.S. GDP every day of every year since
  1901.  Unfortunately, GDP figures are only published once per year
  (once per 365 days), so you need to interpolate them.
  
  Consider the following system: 
  \begin{equation}
    d[n] = \sum_{m=-\infty}^\infty y[m] g[n-365m]
    \label{eq:gdp}
  \end{equation}
  where $y[m]$ is the GDP in the $m^{\textrm{th}}$ year, and $d[n]$ is
  the estimated GDP in the $n^{\textrm{th}}$ day.
  
  Design the filter $g[n]$ so that Eq.~\ref{eq:gdp} implements {\bf
    PIECE-WISE LINEAR} interpolation.  (Draw a sketch of $g[n]$ that
  specifies the values of all of its samples, or write a formula that
  does so).
\end{frame}


\begin{frame}
  \frametitle{Answer}
  \[
  g[n] = \begin{cases}
    1 - \frac{|n|}{365} & -365\le n\le 365\\
    0 & \mbox{otherwise}
  \end{cases}
  \]
\end{frame}



\end{document}
