\documentclass{beamer}
\usepackage{tikz,amsmath,hyperref,graphicx,stackrel,animate,amssymb}
\usetikzlibrary{positioning,shadows,arrows,shapes,calc}
\newcommand{\argmax}{\operatornamewithlimits{argmax}}
\newcommand{\argmin}{\operatornamewithlimits{argmin}}
\mode<presentation>{\usetheme{Frankfurt}}
\AtBeginSection[]
{
  \begin{frame}<beamer>
    \frametitle{Outline}
    \tableofcontents[currentsection,currentsubsection]
  \end{frame}
}
\title{Lecture 19: Exam 2 Review}
\author{Mark Hasegawa-Johnson\\All content~\href{https://creativecommons.org/licenses/by-sa/4.0/}{CC-SA 4.0} unless otherwise specified.}
\date{ECE 401: Signal and Image Analysis, Fall 2021}  
\begin{document}

% Title
\begin{frame}
  \maketitle
\end{frame}

% Title
\begin{frame}
  \tableofcontents
\end{frame}

%%%%%%%%%%%%%%%%%%%%%%%%%%%%%%%%%%%%%%%%%%%%
\section[Topics]{Topics Covered}
\setcounter{subsection}{1}

\begin{frame}
  \frametitle{Topics Covered}

  {\em DSP First}, chapters 5-7:
  \begin{enumerate}
  \item Chapter 5 (MP3; lec.~11-12): FIR Filters
  \item Chapter 6 (HW3, HW4; lec.~13-14): Frequency Response
  \item Chapter 7 (MP4; lec.~15-18): DTFT
  \end{enumerate}
\end{frame}

%%%%%%%%%%%%%%%%%%%%%%%%%%%%%%%%%%%%%%%%%%%%
\section[Convolution]{Convolution}
\setcounter{subsection}{1}

\begin{frame}
  \frametitle{Linearity and Shift-Invariance}
  \begin{itemize}
  \item A system is {\bf linear} if and only if, for any two inputs
    $x_1[n]$ and $x_2[n]$ that produce outputs $y_1[n]$ and $y_2[n]$,
    \[
    x[n]=x_1[n]+x_2[n] \stackrel{\mathcal H}{\longrightarrow}  y[n]=y_1[n]+y_2[n]
    \]
  \item A system is {\bf shift-invariant} if and only if, for any input
    $x_1[n]$ that produces output $y_1[n]$,
    \[
    x[n]=x_1[n-n_0] \stackrel{\mathcal H}{\longrightarrow}  y[n]=y_1[n-n_0]
    \]
  \item If a system is {\bf linear and shift-invariant} (LSI), then it
    can be implemented using convolution:
    \[
    y[n] = h[n]\ast x[n]=\sum_m h[m] x[n-m] = \sum_m h[n-m] x[m]
    \]
    where $h[n]$ is the impulse response:
    \[
    \delta[n] \stackrel{\mathcal H}{\longrightarrow}  h[n]
    \]
  \end{itemize}
\end{frame}

\begin{frame}
  \frametitle{Finite Impulse Response}

  \begin{itemize}
    \item A {\bf finite impulse response (FIR)} filter is one whose
      impulse response has finite length.
    \item If $h[n]$ has finite length, then we can implement the filter
      using an explicit summation:
      \[
      y[n] = h[n]\ast x[n]=\sum_{m=0}^{N-1} h[m] x[n-m]
      \]
  \end{itemize}
\end{frame}

\begin{frame}
  \frametitle{Example: First Difference}

  The first difference operator is:
  \begin{align*}
  x[n] &\stackrel{\mathcal H}{\longrightarrow} y[n]=x[n]-x[n-1]
  \end{align*}
  Its impulse response is:
  \begin{align*}
    h[n] = \delta[n]-\delta[n-1]
  \end{align*}
\end{frame}
  
\begin{frame}
  \frametitle{Example: Simple Delay}

  The delay operator is:
  \begin{align*}
  x[n] &\stackrel{\mathcal H}{\longrightarrow} y[n]=x[n-n_0]
  \end{align*}
  Its impulse response is:
  \begin{align*}
    h[n] = \delta[n-n_0]
  \end{align*}
\end{frame}
  
\begin{frame}
  \frametitle{Example: Local Adder}

  The local adder is:
  \begin{align*}
  x[n] &\stackrel{\mathcal H}{\longrightarrow} y[n]=\sum_{m=0}^{N-1} x[n-m]
  \end{align*}
  Its impulse response is:
  \begin{align*}
    h[n] = \begin{cases}1 & 0\le n\le N-1\\0&\mbox{otherwise}\end{cases}
  \end{align*}
\end{frame}


%%%%%%%%%%%%%%%%%%%%%%%%%%%%%%%%%%%%%%%%%%%%
\section[Freq.~Response]{Frequency Response}
\setcounter{subsection}{1}


\begin{frame}
  \frametitle{Frequency Response}
  \begin{itemize}
  \item {\bf Tones in $\rightarrow$ Tones out}
    \begin{align*}
      x[n]=e^{j\omega n} &\rightarrow y[n]=H(\omega)e^{j\omega n}\\
      x[n]=\cos\left(\omega n\right)
      &\rightarrow y[n]=|H(\omega)|\cos\left(\omega n+\angle H(\omega)\right)\\
      x[n]=A\cos\left(\omega n+\theta\right)
      &\rightarrow y[n]=A|H(\omega)|\cos\left(\omega n+\theta+\angle H(\omega)\right)
    \end{align*}
  \item where the {\bf Frequency Response} is given by
    \[
    H(\omega) = \sum_m h[m]e^{-j\omega m}
    \]
  \end{itemize}
\end{frame}  
        
\begin{frame}
  \frametitle{Example: First Difference}

  The first difference impulse response is:
  \begin{align*}
    h[n] = \delta[n]-\delta[n-1]
  \end{align*}
  Its frequency response is:
  \begin{align*}
    H(\omega) &= 1-e^{-j\omega}\\
    &= e^{j\left(\frac{\pi-\omega}{2}\right)} 2\sin(\omega/2)
  \end{align*}
\end{frame}
  
\begin{frame}
  \frametitle{Example: Simple Delay}

  The delay operator is:
  \begin{align*}
    h[n] = \delta[n-n_0]
  \end{align*}
  Its frequency response is:
  \begin{align*}
    H(\omega) = e^{-j\omega n_0}
  \end{align*}
\end{frame}
  

\begin{frame}
  \frametitle{Example: Local Adder}

  The delayed local adder's impulse response is:
  \begin{align*}
    h[n] = \begin{cases}1 & 0\le n\le N-1\\0&\mbox{otherwise}\end{cases}
  \end{align*}
  Its frequency response is:
  \begin{align*}
    H(\omega) &= \sum_{n=0}^{N-1} e^{-j\omega n}\\
    &= e^{-\frac{j\omega (N-1)}{2}} D_N(\omega),
  \end{align*}
  where $D_N(\omega)$ is the ``Dirichlet form,'' sometimes called the
  ``digital sinc:''
  \begin{align*}
    D_N(\omega) = \frac{\sin(\omega N/2)}{\sin(\omega/2)}
  \end{align*}
\end{frame}

\begin{frame}
  \frametitle{Response of an LSI System to a Periodic Input}
  If the input of an LSI system is periodic,
  \[
  x[n] =\sum_{k=-N_0/2}^{(N_0-1)/2} X_k e^{j2\pi kn/N_0}
  \]
  \ldots then the output is
  \[
  y[n] = \sum_{k=-N_0/2}^{(N_0-1)/2} X_k H(k\omega_0) e^{j2\pi kn/N_0}
  \]
\end{frame}

\begin{frame}
  \frametitle{Cascaded LSI Systems}

  Cascaded LSI Systems convolve their impulse responses, equivalently, they
  multiply their frequency responses:
  \[
  y[n]=h[n]\ast g[n]\ast x[n]~~\leftrightarrow~~Y_k=H(k\omega_0)G(k\omega_0)X_k
  \]
\end{frame}




%%%%%%%%%%%%%%%%%%%%%%%%%%%%%%%%%%%%%%%%%%%%
\section[DTFT]{Discrete Time Fourier Transform}
\setcounter{subsection}{1}


\begin{frame}
  \frametitle{Summary}

  The DTFT (discrete time Fourier transform) of any signal is
  $X(\omega)$, given by
  \begin{align*}
    X(\omega) &= \sum_{n=-\infty}^\infty x[n]e^{-j\omega n}\\
    x[n] &= \frac{1}{2\pi}\int_{-\pi}^\pi X(\omega)e^{j\omega n}d\omega
  \end{align*}
  Particular useful examples include:
  \begin{align*}
    f[n]=\delta[n] &\leftrightarrow F(\omega)=1\\
    g[n]=\delta[n-n_0] &\leftrightarrow G(\omega)=e^{-j\omega n_0}
  \end{align*}
\end{frame}

\begin{frame}
  \frametitle{Properties of the DTFT}

  Properties worth knowing  include:
  \begin{enumerate}
    \setcounter{enumi}{-1}
  \item Periodicity: $X(\omega+2\pi)=X(\omega)$
  \item Linearity:
    \[z[n]=ax[n]+by[n]\leftrightarrow Z(\omega)=aX(\omega)+bY(\omega)
    \]
  \item Time Shift: $x[n-n_0]\leftrightarrow e^{-j\omega n_0}X(\omega)$
  \item Frequency Shift: $e^{j\omega_0 n}x[n]\leftrightarrow X(\omega-\omega_0)$
  \item Filtering is Convolution:
    \[
    y[n]=h[n]\ast x[n]\leftrightarrow Y(\omega)=H(\omega)X(\omega)
    \]
  \end{enumerate}
\end{frame}

\begin{frame}
  \frametitle{Ideal Filters}
  \begin{itemize}
  \item Ideal Lowpass Filter:
    \[
    H_{LP}(\omega)
    = \begin{cases} 1& |\omega|\le\omega_c,\\
      0 & \omega_c<|\omega|\le\pi.
    \end{cases}~~~\leftrightarrow~~~
    h_{LP}[m]=\frac{\omega_c}{\pi}\mbox{sinc}(\omega_c n)
    \]
  \item Ideal Bandpass Filter:
    \begin{align*}
      H_{BP}(\omega)&=H_{LP,\omega_2}(\omega)-H_{LP,\omega_1}(\omega)\\
      \leftrightarrow
      &h_{BP}[n]=\frac{\omega_2}{\pi}\mbox{sinc}(\omega_2 n)-\frac{\omega_1}{\pi}\mbox{sinc}(\omega_1 n)
    \end{align*}
  \item Ideal Highpass Filter:
    \[
    H_{HP}(\omega)=1-H_{LP}(\omega)~~~\leftrightarrow~~~
    h_{HP}[n]=\mbox{sinc}(\pi n)-\frac{\omega_c}{\pi}\mbox{sinc}(\omega_c n)
    \]
  \end{itemize}
\end{frame}

\begin{frame}
  \frametitle{Practical Filters}
  \begin{itemize}
  \item Even-symmetric in time (odd length only):
    \[
    h[n] = \begin{cases}
      h_{\mbox{ideal}}[n]w[n] & -\frac{N-1}{2}\le n\le \frac{N-1}{2}\\
      0 & \mbox{otherwise}
    \end{cases}
    \]
  \item Right-sided in time (odd or even length):
    \[
    h[n] = \begin{cases}
      h_{\mbox{ideal}}\left[n-\left(\frac{N-1}{2}\right)\right]w[n] & 0\le n\le N-1\\
      0 & \mbox{otherwise}
    \end{cases}
    \]
  \end{itemize}
  where $w[n]$ is a finite-length windowing function.
\end{frame}

\begin{frame}
  \frametitle{Windows}

  You need to know these two windowing functions, presented here in
  their right-sided forms:
  \begin{itemize}
  \item Rectangular Window:
    \begin{displaymath}
      w[n]=\begin{cases}1&0\le n\le N-1\\0&\mbox{otherwise}\end{cases}~~\leftrightarrow~~
      W(\omega) = e^{-\frac{j\omega (N-1)}{2}}D_N(\omega)
    \end{displaymath}
    \begin{itemize}
    \item Main lobe halfwidth (first null): $\frac{2\pi}{N}$
    \item First sidelobe level: -11dB
    \end{itemize}
  \item Hamming Window:
    \begin{displaymath}
      w[n]=\begin{cases}
      0.54-0.46\cos\left(\frac{2\pi n}{N-1}\right)&0\le n\le N-1\\0&\mbox{otherwise}
      \end{cases}
    \end{displaymath}
    \begin{itemize}
    \item Main lobe halfwidth (first null): $\frac{4\pi}{N}$
    \item First sidelobe level: -44dB
    \end{itemize}
  \end{itemize}
\end{frame}

%%%%%%%%%%%%%%%%%%%%%%%%%%%%%%%%%%%%%%%%%%%%
\section[Summary]{Summary}
\setcounter{subsection}{1}

\begin{frame}
  \frametitle{Summary: Topics Covered}

  {\em DSP First}, chapters 5-7:
  \begin{enumerate}
  \item Chapter 5: FIR Filters
    \begin{itemize}
    \item LSI systems, impulse response, convolution
    \item first difference, pure delay, local sum
    \end{itemize}
  \item Chapter 6: Frequency Response
    \begin{itemize}
    \item complex exponentials, cosines, periodic signals
    \item cascaded systems
    \item first difference, pure delay, local sum
    \end{itemize}
  \item Chapter 7: DTFT
    \begin{itemize}
    \item DTFT \& frequency response
    \item Ideal filters
    \item Practical filters; rectangular \& Hamming windows
    \end{itemize}
  \end{enumerate}
\end{frame}
\end{document}
