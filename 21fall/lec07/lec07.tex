\documentclass{beamer}
\usepackage{tikz,amsmath,hyperref,graphicx,stackrel,animate,amssymb}
\usetikzlibrary{positioning,shadows,arrows,shapes,calc}
\newcommand{\argmax}{\operatornamewithlimits{argmax}}
\newcommand{\argmin}{\operatornamewithlimits{argmin}}
\mode<presentation>{\usetheme{Frankfurt}}
\AtBeginSection[]
{
  \begin{frame}<beamer>
    \frametitle{Outline}
    \tableofcontents[currentsection,currentsubsection]
  \end{frame}
}
\title{Lecture 8: Interpolation}
\author{Mark Hasegawa-Johnson\\All content~\href{https://creativecommons.org/licenses/by-sa/4.0/}{CC-SA 4.0} unless otherwise specified.}
\date{ECE 401: Signal and Image Analysis, Fall 2021}  
\begin{document}

% Title
\begin{frame}
  \maketitle
\end{frame}

% Title
\begin{frame}
  \tableofcontents
\end{frame}

%%%%%%%%%%%%%%%%%%%%%%%%%%%%%%%%%%%%%%%%%%%%
\section[Sampling]{Review: Sampling}
\setcounter{subsection}{1}

\begin{frame}
  \frametitle{How to sample a continuous-time signal}

  Suppose you have some continuous-time signal, $x(t)$, and you'd like
  to sample it, in order to store the sample values in a computer.
  The samples are collected once every $T_s=\frac{1}{F_s}$ seconds:
  \begin{displaymath}
    x[n] = x(t=nT_s)
  \end{displaymath}
\end{frame}

%%%%%%%%%%%%%%%%%%%%%%%%%%%%%%%%%%%%%%%%%%%%
\section[Interpolation]{Interpolation: Discrete-to-Continuous Conversion}
\setcounter{subsection}{1}

\begin{frame}
  \frametitle{How can we get $x(t)$ back again?}

  We've already seen one method of getting $x(t)$ back again: we can
  find all of the cosine components, and re-create the corresponding
  cosines in continuous time.

  There is an easier way.  It involves multiplying each of the
  samples, $x[n]$, by a short-time pulse, $p(t)$, as follows:
  \begin{displaymath}
    y(t) = \sum_{n=-\infty}^\infty y[n]p(t-nT_s)
  \end{displaymath}
\end{frame}

\begin{frame}
  \frametitle{Rectangular pulses}

  For example, suppose that the pulse is  just a  rectangle,
  \begin{displaymath}
    p(t) = \begin{cases}
      1 & -\frac{T_S}{2}\le t<\frac{T_S}{2}\\
      0 & \mbox{otherwise}
    \end{cases}
  \end{displaymath}

  \centerline{\includegraphics[width=4.5in]{exp/pulse_rectangular.png}}  
\end{frame}

\begin{frame}
  \frametitle{Rectangular pulses = Piece-wise constant interpolation}

  The result is a  piece-wise constant interpolation of the digital signal:

  \centerline{\includegraphics[width=4.5in]{exp/interpolated_rectangular.png}}  
\end{frame}

\begin{frame}
  \frametitle{Triangular pulses}

  The rectangular pulse has the disadvantage that $y(t)$ is discontinuous.
  We can eliminate the discontinuities by using a triangular pulse:
  \begin{displaymath}
    p(t) = \begin{cases}
      1-\frac{|t|}{T_S} & -T_S\le t<T_S\\
      0 & \mbox{otherwise}
    \end{cases}
  \end{displaymath}

  \centerline{\includegraphics[width=4.5in]{exp/pulse_triangular.png}}  
\end{frame}

\begin{frame}
  \frametitle{Triangular pulses = Piece-wise linear interpolation}

  The result is a  piece-wise linear interpolation of the digital signal:

  \centerline{\includegraphics[width=4.5in]{exp/interpolated_triangular.png}}  
\end{frame}

\begin{frame}
  \frametitle{Cubic spline pulses}

  The triangular pulse has the disadvantage that, although $y(t)$ is continuous, its
  first derivative is discontinuous.  We can eliminate discontinuities in the first derivative
  by using a cubic-spline pulse:
  \begin{displaymath}
    p(t) = \begin{cases}
      1-\frac{3}{2}\left(\frac{|t|}{T_S}\right)^2 +\frac{1}{2}\left(\frac{|t|}{T_s}\right)^3 & 0\le |t|\le T_S\\
      -\frac{3}{2}\left(\frac{|t|-2T_s}{T_S}\right)^2\left(\frac{|t|-T_s}{T_S}\right) & T_S\le |t|\le 2T_S\\
      0 & \mbox{otherwise}
    \end{cases}
  \end{displaymath}

\end{frame}

\begin{frame}
  \frametitle{Cubic spline pulses}

  The triangular pulse has the disadvantage that, although $y(t)$ is continuous, its
  first derivative is discontinuous.  We can eliminate discontinuities in the first derivative
  by using a cubic-spline pulse:
  \centerline{\includegraphics[width=4.5in]{exp/pulse_spline.png}}  
\end{frame}

\begin{frame}
  \frametitle{Cubic spline pulses = Piece-wise cubic interpolation}

  The result is a  piece-wise cubic interpolation of the digital signal:

  \centerline{\includegraphics[width=4.5in]{exp/interpolated_spline.png}}  
\end{frame}

\begin{frame}
  \frametitle{Sinc pulses}

  The cubic spline has no discontinuities, and no slope  discontinuities, but it still has
  discontinuities in its second derivative and all higher derivatives.  Can we fix those?

  The answer: yes!  The pulse we need is the inverse transform of an
  ideal lowpass filter, the sinc.
\end{frame}
  
\begin{frame}
  \frametitle{Sinc pulses}

  We can reconstruct a signal that has  no discontinuities in any of its derivatives
  by using an  ideal sinc pulse:
  \begin{displaymath}
    p(t) = \frac{\sin(\pi t/T_S)}{\pi t/T_S}
  \end{displaymath}

  \centerline{\includegraphics[width=4.5in]{exp/pulse_sinc.png}}  
\end{frame}

\begin{frame}
  \frametitle{Sinc pulse = ideal bandlimited interpolation}

  The result is an ideal bandlimited interpolation:

  \centerline{\includegraphics[width=4.5in]{exp/interpolated_sinc.png}}  
\end{frame}

%%%%%%%%%%%%%%%%%%%%%%%%%%%%%%%%%%%%%%%%%%%%
\section[Summary]{Summary}
\setcounter{subsection}{1}

\begin{frame}
  \frametitle{Summary}
  \begin{itemize}
  \item Piece-wise constant interpolation = interpolate using a rectangle
  \item Piece-wise linear interpolation = interpolate using a triangle
  \item Cubic-spline interpolation = interpolate using a spline
  \item Ideal interpolation = interpolate using a sinc
  \end{itemize}
\end{frame}

\end{document}
